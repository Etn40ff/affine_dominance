\documentclass{amsart}
\usepackage{amsmath,amssymb,latexsym,color,ulem,mathabx,pdflscape}
\usepackage{geometry}
\usepackage{tikz}
\usepackage[bookmarks=true, bookmarksopen=true, bookmarksdepth=3,bookmarksopenlevel=2, colorlinks=true, linkcolor=blue, citecolor=blue, filecolor=blue, menucolor=blue, urlcolor=blue]{hyperref}
\usepackage{cancel}

\usepackage[draft]{say}
\newcommand{\sayNR}[1]{\say[NR]{\color{purple}\;#1}}
\newcommand{\sayDR}[1]{\say[DR]{\color{red}\;#1}}
\newcommand{\saySS}[1]{\say[SS]{\color{blue}\;#1}}

\newtheorem{theorem}{Theorem}
\newtheorem{corollary}[theorem]{Corollary}
\newtheorem{definition}[theorem]{Definition}
\newtheorem{example}[theorem]{Example}
\newtheorem{lemma}[theorem]{Lemma}
\newtheorem{proposition}[theorem]{Proposition}
\newtheorem{remark}[theorem]{Remark}
\newtheorem{question}{Question}

\numberwithin{theorem}{section}

\newcommand{\bfb}{\boldsymbol{b}}
\newcommand{\bfc}{\boldsymbol{c}}
\newcommand{\bfd}{\boldsymbol{d}}
\newcommand{\bfe}{\boldsymbol{e}}
\newcommand{\bfg}{\boldsymbol{g}}
\newcommand{\bfk}{{\boldsymbol{k}}}
\newcommand{\bfn}{\boldsymbol{n}}
\newcommand{\bfr}{\boldsymbol{r}}
\newcommand{\bfu}{\boldsymbol{u}}
\newcommand{\bfx}{\boldsymbol{x}}
\newcommand{\bfy}{\boldsymbol{y}}

\newcommand{\cA}{\mathcal{A}}
\newcommand{\cC}{\mathcal{C}}
\newcommand{\cD}{\mathcal{D}}
\newcommand{\cE}{\mathcal{E}}
\newcommand{\cF}{\mathcal{F}}
\newcommand{\cI}{\mathcal{I}}
\newcommand{\cP}{\mathcal{P}}
\newcommand{\cQ}{\mathcal{Q}}

\newcommand{\fp}{\mathfrak{p}}

\newcommand{\CC}{\mathbb{C}}
\newcommand{\QQ}{\mathbb{Q}}
\newcommand{\RR}{\mathbb{R}}
\newcommand{\TT}{\mathbb{T}}
\newcommand{\ZZ}{\mathbb{Z}}

\newcommand{\ol}[1]{{\overline{#1}}}
\newcommand{\vv}[1]{{{}^\vee \! #1}}

\newcommand{\Aut}{\operatorname{Aut}}
\newcommand{\beps}{{\boldsymbol{\varepsilon}}}
\newcommand{\eps}{{\varepsilon}}
\newcommand{\Col}{\operatorname{Col}}
\newcommand{\diag}{\operatorname{diag}}
\newcommand{\dpt}{\operatorname{dp}}
\newcommand{\hgt}{\operatorname{ht}}
\newcommand{\Id}{\operatorname{Id}}
\newcommand{\into}{\hookrightarrow}
\newcommand{\obeta}{{\overline{\beta}}}
\newcommand{\oi}{{\overline{\imath}}}
\newcommand{\ot}{{\overline{t}}}
\newcommand{\rep}{\mathrm{rep}}
\newcommand{\sgn}{{\operatorname{sgn}}}
\newcommand{\Span}{{\operatorname{Span}}}
\newcommand{\WA}{{W\!\!A}}
\newcommand{\wt}{{\operatorname{wt}}}

\title{Dominance Regions for Affine Cluster Algebras}

\author{Nathan Reading}
\author{Dylan Rupel}
\author{Salvatore Stella}

\begin{document}
  \begin{abstract}
    We study the dominance order for $\bfg$-vectors in affine cluster algebras.
  \end{abstract}
  \maketitle

  %%%%%%%%%%%%%%%%%%%%%%
  \section{Introduction}
  Cluster algebras are recursively defined commutative rings.
  Since their discovery by Fomin and Zelevinsky through an intensive study of dual canonical bases \cite{...}, cluster algebras have found application throughout mathematics, including Lie theory \cite{...}, representation theory \cite{...}, Teichm\"uller theory \cite{...}, and mathematical physics \cite{...}.
  See \cite{...} for a more exhaustive description of the deep connections found to cluster algebras.

  A guiding question in the theory has always been to understand possible bases of a cluster algebra.
  Qin put bounds on how the pointed bases can be related.
  


  %%%%%%%%%%%%%%%%%%%%%%%%%%%%%%%%%%%%%
  \section{Mutation Maps and Dominance}
  Fix $m\ge n$.
  Let $\tilde B=(b_{ij})$ be an $m\times n$ exchange matrix with principal $n\times n$ submatrix $B$.
  Note that our exchange matrices are tall, which matches the convention of \cite{qin}.
  Then $B$ is skew-symmetrizable with $DB$ skew-symmetric for some diagonal integer matrix $D=\diag(d_1,\ldots,d_n)$.
 
  For $b\in\RR$, write $[b]_+=\max(b,0)$.
  %Similarly, given a vector $\bfb\in\RR^m$, let $[\bfb]_+$ denote the vector in $\RR_{\ge0}^n$ obtained by applying $[-]_+$ to each entry.
  Given a sign $\varepsilon\in\{\pm1\}$ and $1\le k\le n$, define an $m\times m$ matrix $\tilde E_{k,\varepsilon}=(e_{ij})$ with
  \begin{equation}
    \label{eq:left mutation matrix}
    e_{ij}=\begin{cases} 1 & \text{if $i=j\ne k$;}\\ -1 & \text{if $i=j=k$;}\\ [\varepsilon b_{ik}]_+ & \text{if $i\ne j=k$;}\\ 0 & \text{otherwise;} \end{cases}
  \end{equation}
  and an $n\times n$ matrix $F_{k,\varepsilon}=(f_{ij})$ with
  \begin{equation}
    \label{eq:right mutation matrix}
    f_{ij}=\begin{cases} 1 & \text{if $k\ne i=j$;}\\ -1 & \text{if $k=i=j$;}\\ [-\varepsilon b_{kj}]_+ & \text{if $k=i\ne j$;}\\ 0 & \text{otherwise.} \end{cases}
  \end{equation}
  Observe that $\tilde E^2_{k,\varepsilon}=\Id$ and $F^2_{k,\varepsilon}=\Id$ for any choice of $\varepsilon$.
  Then define $\mu_k\tilde B=\tilde E_{k,\varepsilon} \tilde B F_{k,\varepsilon}$.
  Using the identity $b_{ij}=[b_{ij}]_+-[-b_{ij}]_+$ it is easy to see that $\mu_k\tilde B$ doesn't depend on the choice of sign $\varepsilon$.
  Moreover, the principal part $\mu_k B$ of $\mu_k\tilde B$ is again skew-symmetrizable using the same matrix $D$.

  Given a matrix $M$, denote by $M_{\bullet k}$ (resp. $M_{k \bullet}$) the square matrix whose $k$-th column (resp. $k$-th row) matches that of $M$ with all other entries being zero.
  \begin{lemma}
    For $\varepsilon\in\{\pm1\}$ and $1\le k \le n$, we have 
    \begin{enumerate}
      \item $\tilde E_{k,-\varepsilon} \tilde E_{k,\varepsilon} = I_m + \varepsilon \tilde B_{\bullet k}$, $\tilde E_{k,\varepsilon} - \tilde E_{k,-\varepsilon} = \varepsilon \tilde B_{\bullet k}$, and $\tilde E_{k,\varepsilon} \tilde B_{\bullet k} = \tilde B_{\bullet k}$;
      \item $F_{k,-\varepsilon} F_{k,\varepsilon} = I_n + \varepsilon B_{k \bullet}$, $F_{k,\varepsilon} - F_{k,-\varepsilon} = -\varepsilon B_{k \bullet}$, and $F_{k,\varepsilon} B_{k \bullet} = -B_{k \bullet}$.
    \end{enumerate}
  \end{lemma}
  \begin{proof}
    This is immediate from the equality $\varepsilon b_{ij}=[\varepsilon b_{ij}]_+-[-\varepsilon b_{ij}]_+$.
  \end{proof}

  To record sequences of these matrix mutations, we introduce the $n$-regular rooted tree $\TT_n$ with root vertex $t_0$ and edges labeled by $\{1,\ldots,n\}$.
  Associate $m\times n$ matrices $\tilde{B}^t$ with principal part $B^t$ to the vertices $t\in\TT_n$ so that:
  \begin{itemize}
    \item $\tilde{B}^{t_0}=\tilde{B}$;
    \item if $t,t'\in\TT_n$ are joined by an edge labeled $k$, then $\tilde{B}^{t'}=\mu_k \tilde{B}^t$.
  \end{itemize}
  Given a sequence $\bfk=(k_N,\ldots,k_1)$ with $k_i\in\{1,\ldots,n\}$ write $\mu_\bfk$ for the iterated matrix mutation $\mu_{k_N}\circ\cdots\circ\mu_{k_1}$.
  Then more directly, when $t'$ is obtained from $t$ by following edges labeled by $\bfk=(k_N,\ldots,k_1)$, we have $\tilde B^{t'}=\mu_\bfk \tilde B$.

  A skew-symmetric matrix $B=(b_{ij})$ is \emph{acyclic} if there is no sequence $i_1,\ldots,i_r,i_{r+1}=i_1$ so that $b_{i_\ell i_{\ell+1}}>0$ for $1\le\ell\le r$.
  In the case when $B$ is acyclic, there exists a permutation $\sigma$ of $\{1,\ldots,n\}$ so that $r<r'$ implies $b_{\sigma_r \sigma_{r'}}\ge 0$.
  We also associate to $B$ a Cartan matrix $A=(a_{ij})$ with $a_{ii}=2$ and $a_{ij}=-|b_{ij}|$i for $i\ne j$.
  We say that the mutation pattern is of \emph{affine type} if there exists acyclic $B^t$ whose associated Cartan matrix gives rise to an affine Dynkin diagram.
  
  Assume there exists $t_+\in\TT_n$ so that $B^{t_+}=(b^{t_+}_{ij})$ is acyclic with $b^{t_+}_{ij}\ge 0$ for $i<j$.
  This provides a Coxeter element $c=s_1\cdots s_n$ in the Weyl group associated to $A$.
  
  Given $\bfk=(k_N,\ldots,k_1)$ and any $\tilde B$, define the piecewise-linear \emph{mutation map} $\eta^{\tilde B}_\bfk:\RR^m\to\RR^m$ where $\eta^{\tilde B}_\bfk(\nu)$ is the last column of $\mu_\bfk([\tilde{B}\;\nu])$.
  Note that $\left(\eta^{\tilde{B}}_\bfk\right)^{-1}=\eta^{\mu_\bfk\tilde{B}}_{\bfk^{op}}$, where $\bfk^{op}:=(k_1,\ldots,k_N)$.

  For $\tilde\lambda\in\RR^m$ and $\bfk=(k_N,\ldots,k_1)$, define $S_{\bfk,\tilde\lambda}=\left\{\left(\eta^{\tilde{B}}_\bfk\right)^{-1}\big(\eta^{\tilde B^{t_+}}_\bfk(\tilde\lambda)+\tilde{B}^{t_+}\alpha\big):\alpha\in\RR^n_{\ge0}\right\}$.
  
  \begin{definition}
    For $\tilde\lambda\in\RR^m$, define the \emph{dominance region} 
    \[
      \cP(\tilde\lambda) = \bigcap_\bfk S_{\bfk,\tilde\lambda}.
    \]
    When $\tilde\mu\in\cP(\tilde\lambda)$, we say \emph{$\tilde\lambda$ dominates $\tilde\mu$}.
    Write $\cP_\ZZ(\tilde\lambda):=\cP(\tilde\lambda)\cap \tilde B^+ \cdot \ZZ_{\ge0}^n$. 
  \end{definition}

  \begin{theorem}
    \cite{qin}
    $\cP_\ZZ(\lambda)$ controls the deformations of a basis element pointed at $\tilde\lambda$.
  \end{theorem}


  %%%%%%%%%%%%%%%%%%%%%%%%%%%%%%%%%%%%%%%%%%%%%%%
  \section{Weyl group combinatorics of mutations}
  Let $A^+=(a_{ij})$ denote the Cartan companion of $B^{t^+}$.
  Fix an $n$-dimensional vector space $V$ with basis $\alpha^\vee_1,\ldots,\alpha^\vee_n$, called the \emph{simple coroots}.
  Define \emph{simple roots} $\alpha_i:=d_i\alpha^\vee_i$ which provide another basis of $V$.
  We will use the bilinear pairing $K$ on $V$ defined by $K(\alpha^\vee_i,\alpha_j)=a_{ij}$.
  This defines the \emph{simple reflections} $s_i(\beta)=\beta-K(\alpha^\vee_i,\beta)\alpha_i$ and the corresponding Weyl group $W=\langle s_1,\ldots,s_n\rangle$ acting linearly on $V$.
  Write $S=\{s_1,\ldots,s_n\}$ for the collection of simple reflections.
  Given $I\subseteq S$, let $W_I=\langle s:s\in I\rangle$ denote the \emph{parabolic subgroup} generated by $I$.
  For $s\in S$, write $[s]=S\setminus\{s\}$.

  Let $c=s_1\cdots s_n$ be the \emph{Coxeter element} of $W$ associated to $B^{t^+}$.
  More generally, any element of $W$ that can be obtained as the product of all elements of $S$ in some order is a \emph{Coxeter element}.
  An element $s\in S$ is \emph{initial} in $c$ if $\ell(sc)<\ell(c)$, in this case $sc$ is a Coxeter element of $W_{[s]}$.
  When $s_k$ is initial in $c$, $s_k c s_k$ is the Coxeter element associated to $\mu_k B^{t^+}$.
  An element $w\in W$ is \emph{$c$-sortable} if the following recursive definition holds
  \begin{itemize}
    \item the identity element of $W$ is $c$-sortable;
    \item if $s$ is initial in $c$ and $\ell(sw)<\ell(w)$, then $w$ is $c$-sortable if and only if $sw$ is $scs$-sortable;
    \item if $s$ is initial in $c$ and $\ell(sw)>\ell(w)$, then $w$ is $c$-sortable if and only if $w\in W_{[s]}$ is $sc$-sortable.
  \end{itemize}


  
  






  Define the $m\times m$ matrix $\tilde G^t$ and the $n\times n$ matrix $C^t$ recursively from the initial seed $t_+$.
  \begin{remark}
    Since we only mutate in directions $k\in[1,n]$, $\tilde G^t$ has the following block form:
    \[\left[\begin{array}{cc} G^t & 0\\ * & 1\end{array}\right]\]
    where $G^t$ is the $n\times n$ $\bfg$-matrix for the coefficient-free case.
  \end{remark}

  The following is immediate by manipulating the recursion and using that it is independent of the choice of sign.
  \begin{lemma}
    For any $t\in\TT_n$, we have
    \[G^t\tilde B^t=\tilde B^0 C^t.\]
  \end{lemma}

  \begin{definition}
    We say $\nu,\lambda\in\RR^m$ are in the same \emph{$\tilde B^+$-class} if they lie in the same domain of linearity for $\eta^{\tilde B^+}_\bfk$ for all sequences $\bfk$.
    Define the \emph{mutation fan} $\cF_{\tilde B^+}$ whose maximal cones are the closures of the $\tilde B^+$-classes.
  \end{definition}
  \begin{remark}
    $\cF_{\tilde B^+}=\cF_{B^+}\times \RR^{m-n}$ since the linearity domain in which $\nu$ lies depends only on its first $n$ coordinates.
  \end{remark}

  Cite these results for affine types:
  \begin{itemize}
    \item $\cF_{B^+}$ is simplicial and complete 
    \item $\cF_{B^+}$ is equivalent to the (transposed) scattering diagram fan and the associahedron(?) fan
    \item A maximal (by inclusion) cone $K$ in $\cF_{B^+}$ is \emph{real} if there exists $t\in\TT_n$ such that $K=K_t:=G^t\cdot\RR^n_{\ge0}$, otherwise it is \emph{imaginary}.
    \item There is a finite of imaginary cones.
    \item Any maximal imaginary cone $K$ is $n-1$ dimensional and is contained in $\delta^\perp$ (need to define $\delta$ and $\perp$)
    \item There exists $N>0$ so that $c^N$ (as it acts on weights) fixes $K$ pointwise
    \item $-\frac12 B^+\cdot\delta$ is one of the primitive vectors spanning a ray of $K$
    \item Let $\eta_1,\ldots,\eta_{n-2}$ be the other $n-2$ primitive vectors spanning rays of $K$.  
    \item Then there exists a real cone $H$ in $\cF_{B^+}$ such that
      \begin{itemize}
        \item $\eta_1,\ldots,\eta^{n-2}$ span rays of $H$
        \item $c^{\ell N} H$ is a real cone for all $\ell\ge0$
        \item $\lim\limits_{\ell\to\infty} c^{\ell N} H = K$
      \end{itemize}
      The second condition above defines a sequence $t_\ell\in\TT_n$ that \emph{converges} to $K$.
    \item Let $\eta_{n-1},\eta_n$ be the remaining primitive vectors spanning rays of $H$.  Then, in the appropriate scaling limit, both of these vectors limit to $-\frac12 B^+\cdot\delta$ under the action of $c^{\ell N}$ as $\ell\to\infty$.
  \end{itemize}

  Part of $\cF_{B^+}$ is the Cambrian fan
  
  c-sortable elements
  
  after Prop. 5.4 in \cite{combinatorial frameworks} gives C-matrices explicitly
  
  -> quasi-Cartan companions, kernels of those

  \begin{theorem}
    Let $\tilde\lambda$ be imaginary with $s$ minimal such that $\tilde\lambda+s\tilde B^+\cdot\delta$ is real.
    Then the dominance region $\cP_{\tilde\lambda}$ is the segment $\{\tilde\lambda+r\tilde B^+\cdot\delta:0\le r\le s\}$.
  \end{theorem}
  \begin{remark}
    When $\tilde B^+$ is full rank, $\cP_\ZZ(\tilde\lambda)$ contains points of the form $\tilde\lambda+r\tilde B^+\cdot\delta$ with $r\in\ZZ_{\ge0}$.
    Without the full rank assumption, this no longer has to hold.
  \end{remark}
  \begin{proof}
    \begin{itemize}
      \item $\cP(\tilde\lambda)$ is contained in the imaginary cone.
        $\cP(\tilde\lambda)$ is contained in $\tilde\lambda + \tilde B^+ \cdot \RR_{\ge0}^n$ which is a proper cone since $\tilde B^+$ is full rank.
        In particular, for sufficiently large $r$, $\cP(\tilde\lambda)$ doesn't contain $-r \tilde B^+\cdot \delta$.
        Then for any point $\tilde\mu$ outside the closure of the imaginary cone, for sufficiently large $\ell$ the vector $c^\ell \tilde\mu$ has large magnitude and is arbitrarily close the imaginary ray.
      \item $\cP(\tilde\lambda)$ contains $\{\tilde\lambda+r\tilde B^+\cdot\delta:0\le r\le s\}$.
        \begin{itemize}
          \item $\delta$ spans the kernel of the Cartan companion $A$ of $B^+$ as well as any number of Coxeter mutations away
          \item after sufficiently many Coxeter mutations, the seed $t$ lies entirely on one side of the imaginary hyperplane and so the kernel $\kappa_t$ of the quasi-Cartan companion is sign coherent

            \emph{proof}: Since $A^t:=(C^{\vee,t})^T A_0 C^t$ then $G^t A^t (G^{\vee,t})^T:=A_0$. Since $G^t $ is invertible $A_0\cdot \delta = 0$ implies $A^t (G^{\vee,t})^T \cdot \delta = 0$ i.e. $(G^{\vee,t})^T \cdot \delta$ spans the kernel of $A^t$. After sufficiently many coxeter all the g-vectors at $t$ are on the positive side of $\delta^\perp$ and the result follows.
          \item for $t'$ connected to $t$ by mutation in direction $k$, we want $\eta_k^{\tilde B^t}(\tilde B^t\cdot\kappa_t)=\tilde B^{t'}\cdot\kappa_{t'}$
          \item $\eta_k^{\tilde B^t}(\tilde B^t\cdot\kappa_t)=\tilde E^t_{k,\varepsilon}\tilde B^t\cdot\kappa_t=\tilde E^t_{k,\varepsilon}\tilde B^t F^t_{k,\varepsilon} F^t_{k,\varepsilon}\cdot\kappa_t = \tilde B^{t'} \cdot F^t_{k,\varepsilon} \kappa_t $ but $\kappa_{t'}= F^t_{k,\varepsilon_{trop}} \kappa_t$
          \item For Cambrian mutations, $\varepsilon_{trop}$ is always positive.
            This follows from the explicit formula for $\bfc$-vectors in \cite{cambrian frameworks} after Prop. 5.4.
          \item The $k$-th row of $F^t_{k,+}+F^t_{k,-}$ is the negative of the $k$-th row of the associated Cartan companion (not quasi)
          \item The $k$-th row of $F^t_{k,-}-F^t_{k,+}$ is the $k$-th row of $\tilde B^t$
          \item The $k$-th entries of $\tilde B^{t'} \cdot F^t_{k,\varepsilon} \kappa_t$ and $\tilde B^{t'} \cdot \kappa_{t'}$ are the same
          \item $A^t:=(C^{\vee,t})^T A_0 C^t$
          \item $A^{t'}=F^{t,\vee}_{k,\varepsilon_{trop}} A_t F^t_{k,\varepsilon_{trop}}$
            \saySS{I think here we mean $A^{t'}=(F^{t,\vee}_{k,\varepsilon_{trop}})^T A_t F^t_{k,\varepsilon_{trop}}$}
        \end{itemize}
      \item $\cP(\tilde\lambda)$ is contained in the ray $\{\tilde\lambda+r\tilde B^+\cdot\delta:r\ge 0\}$.
        \begin{itemize}
          \item Define green and red regions for real $\tilde\lambda$.
          \item Compute the green and red regions as $\tilde\lambda \pm \tilde B^+\cdot C^t\cdot \RR_{\ge0}^n$.
            Use $\tilde G^t\tilde B^t = \tilde B^+ C^t$.
          \item Limits of regions make sense because of continuity after intersecting with domains of linearity.
          \item Use limits of $C^t$ to draw the conclusion.
        \end{itemize}
      \item For (sufficiently) Cambrian seeds $t$, the sign of the $k$-th entry of $\tilde B^t (G^{\vee,t})^T \delta$ is the same as the sign of the $k$-th $\bfc$-vector since $\tilde B^t (G^{\vee,t})^T \delta=(C^{\vee,t})^T B^+ \delta = -2 (C^{\vee,t})^T \nu_c(\delta)$ and $\nu_c$ is given by the negative of the Euler matrix.
    \end{itemize}

  \end{proof}


  %%%%%%%%%%%%%%%%%%%%%%%%
  \section{$\bfc$-vectors}

  Definition $\bfc$-matrices $C^t$, for $t\in\TT_n$, recursively as follows:
  \begin{itemize}
    \item $C^{t_0} = \boldsymbol{1}_n$ is the $n\times n$ identity matrix;
    \item when $t$ and $t'$ are joined by an edge labeled $k$, $C^{t'}=(c^{t'}_{ij})$ is related to $C^t=(c^t_{ij})$ by
      \[ c^{t'}_{ij} = \begin{cases} -c^t_{ij} & \text{if $i=k$ or $j=k$;}\\ c^t_{ij} + [-\varepsilon c^t_{ik}]_+ b^t_{kj} + c^t_{ik} [\varepsilon b^t_{kj}]_+ & \text{otherwise;} \end{cases} \]
        for any choice of sign $\varepsilon\in\{\pm1\}$.
  \end{itemize}

  $C^t$ is $n\times n$
  \begin{lemma}
    $C^t$ is a product of matrices $F^t_{k,\varepsilon}$ using tropical signs
  \end{lemma}

  Sign coherence?



  %%%%%%%%%%%%%%%%%%%%%%%%
  \section{$\bfg$-vectors}
  Definitions $G^t$ is $m\times m$
  \begin{lemma}
    $G^t$ is a product of matrices $E^t_{k,\varepsilon}$ using tropical signs
  \end{lemma}

  There exists a rank 2 cluster algebra with frozen variables $\eta_i$, with $\bfg$-vectors $\bfg_k$ for $k\in\ZZ$.
  This gives real clusters $X_k:=\{\eta_1,\ldots,\eta_{n-2},\bfg_k,\bfg_{k+1}\}$ such that $\lim_{j\to\infty} c^{j\ell}\bfg_k=\nu_c(\delta)$.

  \begin{lemma}
    There exist $\lambda_k\in\Span_{\ge0} X_k$ so that $\lim_{k\to\infty} \lambda_k=\lambda$.
  \end{lemma}


  %%%%%%%%%%%%%%%%%%%%%%%%%%%%%%%
  \section{Green and Red Regions}

  \begin{definition}
    Consider real $\tilde \lambda$, say $\tilde \lambda\in \tilde K_t:= K_t\times \RR^{m-n}$.
    Let $t$ be connected to $t_+$ by a sequence of edges labeled by $\bfk^+=(k_N,\ldots k_1)$. 
    Define the \emph{green cone} $S_\lambda^+:=S_{\bfk^+,\tilde\lambda}\cap\tilde K_t$.
    Similarly, let $\bfk^-=(n,\ldots,1,k_N,\ldots,k_1)$ and define the \emph{red cone} $S_\lambda^-:=S_{\bfk^-,\tilde\lambda}\cap\tilde K_t$.
  \end{definition}

  Define $S^\pm_\lambda:=\lim_{k\to\infty} S^\pm_{\lambda_k}$.

  \begin{lemma}
    $\cP(\lambda) \subseteq \Lambda^\pm_\lambda$
  \end{lemma}

  \begin{lemma}
    Consider $\lambda$ inside the $\bfg$-vector fan, say $\lambda$ lies inside the cone spanned by $G_t$.
    Then the intersection of $\phi_t^{-1} \cC_t(\phi_t \lambda)$ with this $\bfg$-cone is spanned by the columns of $\tilde B_0 C_t$.
  \end{lemma}

  \begin{lemma}
    Similar statement for the red region
  \end{lemma}

  \begin{corollary}
    Let $\lambda\in\ZZ^m$ correspond to a cluster monomial
    \sayDR{Possibly multiplied by a Laurent monomial in coefficients}. 
    Then $\cP(\lambda)=\{\lambda\}$.
  \end{corollary}
  \begin{remark}
    This uses full rank assumption.
  \end{remark}

  \begin{lemma}
    For imaginary $\lambda$, $\Lambda^+_\lambda\cap\Lambda^-_\lambda$ is the line through $\lambda$ in direction $\nu_c(\delta)$.
  \end{lemma}

  \begin{corollary}
    Assume $\tilde B$ is affine.
    For imaginary $\lambda$, $\cP(\lambda)$ is contained in the line through $\lambda$ in direction $\nu_c(\delta)$. 
  \end{corollary}


  %%%%%%%%%%%%%%%%%%%%%%%%%%%
  \section{Quasi-Cartan Companions}

  \begin{lemma}
    For any $t\in\TT_n$, there exists a quasi-Cartan companion $A^t$ of corank 1 for $B^t$ such that $\ker A_t$ is spanned by a positive primitive vector $\delta_t$.
  \end{lemma}
  \begin{proof}
    Move from $t$ to $t'$ with $B^t=B^{t'}$ such that $G^{t'}\cdots\RR^n$ lies entirely on one side of the imaginary hyperplane.
  \end{proof}

  %%%%%%%%%%%%%%%%%%%%%
  \section{Affine Type}
  
  \begin{proposition}
    The dominance region for an imaginary $\bfg$-vector $\lambda$ does not intersect the interior of the $\bfg$-vector fan.
  \end{proposition}
  \begin{proof}
    Anything in the interior of the $\bfg$-vector fan is a part of a single cluster.
    Moreover, it contains a preprojective or postinjective summand.
    It follows that a sufficiently high power of $\tau$ or $\tau^{-1}$ will move the $\bfg$-vector for that summand far away from the origin and close to the boundary of the $\bfg$-vector fan.
    Thus it cannot lie in the $B$-cone pointed at $\lambda$.
  \end{proof}

  \begin{proposition}
    For acyclic $\tilde B$, the first $n$ entries of $-\tilde B\cdot\delta$ is twice the $\bfg$-vector of $\delta$.
  \end{proposition}
  \begin{proof}
    $B+C$ is twice the Euler matrix, multiplying by $\delta$ gives the result.
  \end{proof}


  Let $B$ be an acyclic exchange matrix of affine type with $DB$ skew-symmetric.
  Write $A$ for the Cartan companion of $B$ and note that $A$ has corank 1.
  Consider $B^t$ mutation equivalent to $B$ and $C^t$ the associated $\bfc$-matrix.

  Let $A_0$ denote the Cartan companion of $B$ and write $\delta_0$ for the positive vector spanning the kernel of $A_0$.
  For $t\in\TT_n$, let $A^t:=(C^{\vee,t})^T A_0 C^t$ denote the Reading-Speyer quasi-Cartan companion of $B^t$ \cite[Cor. 3.29]{Combinatorial Frameworks for Cluster Algebras}.
  \begin{theorem}
    Let $\delta^t$ be the absolute value of the kernel of $A^t$.
    Then the primitive purely imaginary $\bfg$-vector direction at the seed $t$ is $-B^t\delta^t/2$.
  \end{theorem}

  \begin{lemma}
    The matrix $\cE^t A^t \cE^t$ is an admissible quasi-Cartan companion of $B^t$ of corank 1.
    Moreover, the kernel of $A^t$ is spanned by a non-negative vector $\delta^t$. 
  \end{lemma}

  Define $\zeta^t := -B^t \delta^t$.
  \begin{lemma}
    The vector $\zeta^t$ is an imaginary $\bfg$-vector.
  \end{lemma}

  \begin{lemma}
    With respect to the seed $t$, the $\bfd$-vector of the imaginary theta basis element $\vartheta_{\zeta^t}$ is $\delta^t$. 
  \end{lemma}

  Define the (opposite) defect vector of signs $\boldsymbol{\varepsilon}^t=(\varepsilon^t_1,\ldots,\varepsilon^t_n)$ where $\varepsilon^t_i=\sgn\,\langle \bfg^{\vee,t}_i,\delta\rangle$ is (the negative of) the defect of the representation of $Q$ with $\bfg$-vector $\bfg^t_i$.
  Let $\kappa^t$ denote the vector $(G^{\vee,t})^T\delta_0$ which spans the kernel of $A^t$.
  \begin{align*}
    E_{k,\varepsilon}^t B^t \boldsymbol{\varepsilon}^t \kappa^t
    &=
    E_{k,\varepsilon}^t B^t F^t_{k,\varepsilon} F^t_{k,\varepsilon} \boldsymbol{\varepsilon}^t \kappa^t\\
    &=
    B^{t'} F^t_{k,\varepsilon} \boldsymbol{\varepsilon}^t \kappa^t\\
    &\stackrel{?}{=} B^{t'} \boldsymbol{\varepsilon}^{t'} \kappa^{t'}
  \end{align*}
  \begin{align*}
    A^t\kappa^t=0=(F^{\vee,t}_{k,\varepsilon_{trop}})^T A^t F^t_{k,\varepsilon_{trop}} F^t_{k,\varepsilon_{trop}} \kappa^t=A^{t'} F^t_{k,\varepsilon_{trop}} \kappa^t
  \end{align*}
  so $\kappa^{t'}=F^t_{k,\varepsilon_{trop}} \kappa^t$

  \begin{lemma}
    $F^t_{k,\varepsilon} \boldsymbol{\varepsilon}^t \kappa^t = \boldsymbol{\varepsilon}^{t'} F^t_{k,\varepsilon_{trop}} \kappa^t$
  \end{lemma}
  \[F^t_{k,\varepsilon} \boldsymbol{\varepsilon}^t \kappa^t - F^t_{k,-\varepsilon} \boldsymbol{\varepsilon}^t \kappa^t=-\varepsilon B^t_{k\bullet} \boldsymbol{\varepsilon}^t \kappa^t \le 0\]
  since $\varepsilon$ is the sign of the $k$-th entry of $B^t \boldsymbol{\varepsilon}^t \kappa^t$


\bibliographystyle{amsalpha}
\bibliography{bibliography}  

\end{document}
