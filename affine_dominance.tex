\documentclass{amsart}
\usepackage{amsmath,amssymb,latexsym,color,ulem,mathabx,pdflscape}
\usepackage{geometry}
\usepackage{tikz}
\usepackage[bookmarks=true, bookmarksopen=true, bookmarksdepth=3,bookmarksopenlevel=2, colorlinks=true, linkcolor=blue, citecolor=blue, filecolor=blue, menucolor=blue, urlcolor=blue]{hyperref}
\usepackage{cancel}

\usepackage[draft]{say}
\newcommand{\sayNR}[1]{\say[NR]{\color{purple}\;#1}}
\newcommand{\sayDR}[1]{\say[DR]{\color{red}\;#1}}
\newcommand{\saySS}[1]{\say[SS]{\color{blue}\;#1}}

\newtheorem{theorem}{Theorem}
\newtheorem{corollary}[theorem]{Corollary}
\newtheorem{definition}[theorem]{Definition}
\newtheorem{example}[theorem]{Example}
\newtheorem{lemma}[theorem]{Lemma}
\newtheorem{proposition}[theorem]{Proposition}
\newtheorem{remark}[theorem]{Remark}
\newtheorem{question}{Question}

\numberwithin{theorem}{section}

\newcommand{\bfb}{\boldsymbol{b}}
\newcommand{\bfc}{\boldsymbol{c}}
\newcommand{\bfd}{\boldsymbol{d}}
\newcommand{\bfe}{\boldsymbol{e}}
\newcommand{\bfg}{\boldsymbol{g}}
\newcommand{\bfk}{\boldsymbol{k}}
\newcommand{\bfn}{\boldsymbol{n}}
\newcommand{\bfr}{\boldsymbol{r}}
\newcommand{\bfu}{\boldsymbol{u}}
\newcommand{\bfx}{\boldsymbol{x}}
\newcommand{\bfy}{\boldsymbol{y}}

\newcommand{\cA}{\mathcal{A}}
\newcommand{\cC}{\mathcal{C}}
\newcommand{\cD}{\mathcal{D}}
\newcommand{\cE}{\mathcal{E}}
\newcommand{\cI}{\mathcal{I}}
\newcommand{\cP}{\mathcal{P}}
\newcommand{\cQ}{\mathcal{Q}}

\newcommand{\fp}{\mathfrak{p}}

\newcommand{\CC}{\mathbb{C}}
\newcommand{\QQ}{\mathbb{Q}}
\newcommand{\RR}{\mathbb{R}}
\newcommand{\TT}{\mathbb{T}}
\newcommand{\ZZ}{\mathbb{Z}}

\newcommand{\ol}[1]{{\overline{#1}}}
\newcommand{\vv}[1]{{{}^\vee \! #1}}

\newcommand{\Aut}{\operatorname{Aut}}
\newcommand{\beps}{{\boldsymbol{\varepsilon}}}
\newcommand{\eps}{{\varepsilon}}
\newcommand{\Col}{\operatorname{Col}}
\newcommand{\diag}{\operatorname{diag}}
\newcommand{\dpt}{\operatorname{dp}}
\newcommand{\hgt}{\operatorname{ht}}
\newcommand{\Id}{\operatorname{Id}}
\newcommand{\into}{\hookrightarrow}
\newcommand{\obeta}{{\overline{\beta}}}
\newcommand{\oi}{{\overline{\imath}}}
\newcommand{\ot}{{\overline{t}}}
\newcommand{\rep}{\mathrm{rep}}
\newcommand{\sgn}{{\operatorname{sgn}}}
\newcommand{\WA}{{W\!\!A}}
\newcommand{\wt}{{\operatorname{wt}}}

\title{Dominance Regions for Affine Cluster Algebras}

\author{Nathan Reading}
\author{Dylan Rupel}
\author{Salvatore Stella}

\begin{document}
  \begin{abstract}
    We study the dominance order for $\bfg$-vectors in affine cluster algebras.
  \end{abstract}
  \maketitle

  %%%%%%%%%%%%%%%%%%%%%%
  \section{Introduction}


  %%%%%%%%%%%%%%%%%%%%%%%%%%%%%%%%%%%%%
  \section{Mutation Maps and Dominance}
  Let $\tilde B^0=(b_{ij})$ be an $m\times n$ exchange matrix with principal $n\times n$ submatrix $B$.
  Then $B^0$ is skew-symmetrizable with $DB^0$ skew-symmetric for some diagonal integer matrix $D=\diag(d_1,\ldots,d_n)$.
  The matrix $B$ is \emph{acyclic} if there is no sequence $i_1,\ldots,i_m,i_{m+1}=i_1$ so that $b_{i_\ell i_{\ell+1}}>0$ for $1\le\ell\le m$.
  In the case when $B$ is acyclic, there exists a labeling $\{k_1,\ldots,k_n\}=\{1,\ldots,n\}$ so that $r<r'$ implies $b_{k_{r'} k_r}\ge 0$.
  The sequence $(k_1,\ldots,k_n)$ (resp. $(k_n,\ldots,k_1)$) is then said to be \emph{source-adapted} (resp. \emph{sink-adapted}).

  For $b\in\RR$, write $[b]_+=\max(b,0)$.
  Similarly, given a vector $\bfb\in\RR^n$, let $[\bfb]_+$ denote the vector in $\RR_{\ge0}^n$ obtained by applying $[-]_+$ to each entry.
  Given a sign $\varepsilon\in\{\pm1\}$ and $1\le k\le n$, define an $m\times m$ matrix $E_{k,\varepsilon}=(e_{ij})$ with
  \begin{equation}
    \label{eq:left mutation matrix}
    e_{ij}=\begin{cases} 1 & \text{if $i=j\ne k$;}\\ -1 & \text{if $i=j=k$;}\\ [\varepsilon b_{ik}]_+ & \text{if $i\ne j=k$;}\\ 0 & \text{otherwise;} \end{cases}
  \end{equation}
  and an $n\times n$ matrix $F_{k,\varepsilon}=(f_{ij})$ with
  \begin{equation}
    \label{eq:right mutation matrix}
    f_{ij}=\begin{cases} 1 & \text{if $k\ne i=j$;}\\ -1 & \text{if $k=i=j$;}\\ [-\varepsilon b_{kj}]_+ & \text{if $k=i\ne j$;}\\ 0 & \text{otherwise.} \end{cases}
  \end{equation}
  Observe that $E^2_{k,\varepsilon}=\Id$ and $F^2_{k,\varepsilon}=\Id$ for any choice of $\varepsilon$.

  The index $k$ also determines a new matrix $\mu_k \tilde{B}=(b'_{ij})$ given by
  \begin{equation}
    \label{eq:matrix mutation}
    b'_{ij}=\begin{cases} -b_{ij} & \text{if $i=k$ or $j=k$;}\\ b_{ij}+[b_{ik}]_+b_{kj}+b_{ik}[-b_{kj}]_+ & \text{otherwise.} \end{cases}
  \end{equation}
  One easily observes that the principal part $\mu_k B$ of $\mu_k \tilde{B}$ is again skew-symmetrizable using the same matrix $D$.
  \begin{remark}
    Note that $\mu_k \tilde B=E_{k,\varepsilon} \tilde B F_{k,\varepsilon}$ for $\varepsilon=\pm 1$, the case $\varepsilon=1$ being obvious from the definitions and the case $\varepsilon=-1$ following from the identity $b_{ij}=[b_{ij}]_+-[-b_{ij}]_+$.
  \end{remark}

  To record sequences of these matrix mutations, we introduce the $n$-regular rooted tree $\TT_n$ with root vertex $t_0$ and edges labeled by $\{1,\ldots,n\}$.
  Associate $2n\times n$ matrices $\tilde{B}^t$ to the vertices $t\in\TT_n$ so that:
  \begin{itemize}
    \item $\tilde{B}^{t_0}=\tilde{B}$;
    \item if $t,t'\in\TT_n$ are joined by an edge labeled $k$, then $\tilde{B}^{t'}=\mu_k \tilde{B}^t$.
  \end{itemize}
  Write $B^t$ for the principal part of $\tilde B^t$.

  Associated to a skew-symmetrizable matrix $B=(b_{ij})$ there is a Cartan matrix $A=(a_{ij})$ with $a_{ii}=2$ and $a_{ij}=-|b_{ij}|$.
  We say that $B^0$ is of \emph{affine type} if there exists $B^t$ mutation equivalent to $B^0$ whose associated Cartan matrix gives rise to an affine Dynkin diagram.

  For $t\in\TT_n$ and $k\in\{1,\ldots,n\}$, define $\phi^t_k:\RR^m\to\RR^m$ as the piecewise-linear map
  \[
    \phi^t_k(\lambda)=\begin{cases} E^t_{k,+}\lambda & \text{if $\lambda_k\ge0$;}\\ E^t_{k,-}\lambda & \text{if $\lambda_k<0$;} \end{cases}
  \]
  where the entries of $E^t_{k,\varepsilon}$ are given by \eqref{eq:left mutation matrix} with $b^t_{ij}$ in place of $b_{ij}$.
  We leave it as an exercise for the reader to check that $(\phi^t_k)^{-1}=\phi^{t'}_k$ whenever $t,t'\in\TT_n$ are joined by an edge labeled $k$.

  %\begin{lemma}
  %  Suppose $t,t'\in\TT_n$ are joined by an edge labeled $k$.
  %  If $\mu\in\lambda+\Col(\tilde{B}^t)$, then $\phi^t_k(\mu)\in\phi^t_k(\lambda)+\Col(\tilde{B}^{t'})$.
  %\end{lemma}
  %\begin{proof}
  %  Suppose $\mu=\lambda+\tilde{B}^t\alpha$ for some $\alpha\in\RR^n$.
  %  Let $\varepsilon$ (resp. $\varepsilon'$) be such that $\phi^t_k(\mu)=E^t_{k,\varepsilon}\mu$ (resp. $\phi^t_k(\lambda)=E^t_{k,\varepsilon'}\lambda$).
  %  Then 
  %  \[
  %    \phi^t_k(\mu)=E^t_{k,\varepsilon}\mu=E^t_{k,\varepsilon}(\lambda+\tilde{B}^t\alpha)=E^t_{k,\varepsilon}\lambda + E^t_{k,\varepsilon} \tilde{B}^t F^t_{k,\varepsilon} F^t_{k,\varepsilon}\alpha=E^t_{k,\varepsilon}\lambda + \tilde{B}^{t'} F^t_{k,\varepsilon}\alpha.
  %  \]
  %  If $\varepsilon'=\varepsilon$, this is equal to $\phi^t_k(\lambda) + \tilde{B}^{t'} F^t_{k,\varepsilon}\alpha$ and so $\phi^t_k(\mu)\in\phi^t_k(\lambda)+\Col(\tilde{B}^{t'})$.
  %  Otherwise, 
  %  \[
  %    E^t_{k,\varepsilon}\lambda=E^t_{k,\varepsilon'}\lambda+\varepsilon(\tilde{B}^t)^{\bullet k}\lambda=\phi^t_k(\lambda)-\varepsilon(\tilde{B}^{t'})^{\bullet k}\lambda=\phi^t_k(\lambda)-\varepsilon \tilde{B}^{t'} \lambda_k \mathbf{e}_k
  %  \]
  %  and again we see that $\phi^t_k(\mu)\in\phi^t_k(\lambda)+\Col(\tilde{B}^{t'})$.
  %\end{proof}

  For $t\in\TT_n$, define the piecewise-linear automorphisms $\phi_t:\RR^m\to\RR^m$ by
  \begin{itemize}
    \item $\phi_{t_0}=\Id$;
    \item if $t,t'\in\TT_n$ are joined by an edge labeled $k$, then $\phi_{t'}=\phi^t_k \phi_t$.
  \end{itemize}
  
  For $\lambda\in\RR^m$ and $t\in\TT_n$, define the \emph{$\tilde B^t$-cone} $\cC_t(\lambda):=\lambda+\tilde B^t\cdot\RR_{\ge0}^n$ pointed at $\lambda$.

  \begin{definition}
    For $\lambda\in\RR^m$, define the \emph{dominance region} 
    \[
      \cP(\lambda) = \bigcap_{t\in\TT_n} \phi_t^{-1} \cC_t(\phi_t \lambda).
    \]
    When $\mu\in\cP(\lambda)$, we say \emph{$\lambda$ dominates $\mu$}.
  \end{definition}


  %%%%%%%%%%%%%%%%%%%%%%%%%%%%%%%%%%%%%%%%%%%
  \section{$\bfg$-vectors and $\bfc$-vectors}

  Definitions

  $G^t$ is $m\times m$ and $C^t$ is $n\times n$
  \begin{lemma}
    $G^t$ is a product of matrices $E^t_{k,\varepsilon}$ and $C^t$ is a product of matrices $F^t_{k,\varepsilon}$
  \end{lemma}

  \begin{lemma}
    For any $t\in\TT_n$, we have
    \[G^t\tilde B^t=\tilde B^0 C^t.\]
  \end{lemma}

  \begin{lemma}
    Limit of $\bfc$-vectors under Coxeter transformation
  \end{lemma}


  %%%%%%%%%%%%%%%%%%%%%%%%%%%%%%%
  \section{Green and Red Regions}

  \begin{lemma}
    Consider $\lambda$ inside the $\bfg$-vector fan, say $\lambda$ lies inside the cone spanned by $G_t$.
    Then the intersection of $\phi_t^{-1} \cC_t(\phi_t \lambda)$ with this $\bfg$-cone is spanned by the columns of $\tilde B_0 C_t$.
  \end{lemma}

  \begin{lemma}
    Similar statement for the red region
  \end{lemma}

  \begin{corollary}
    Let $\lambda\in\ZZ^m$ correspond to a cluster monomial
    \sayDR{Possibly multiplied by a Laurent monomial in coefficients}. 
    Then $\cP(\lambda)=\{\lambda\}$.
  \end{corollary}

  \begin{corollary}
    Assume $\tilde B$ is affine.
    For imaginary $\lambda$, $\cP(\lambda)$ is contained in the line through $\lambda$ in direction $\nu_c(\delta)$. 
  \end{corollary}


  %%%%%%%%%%%%%%%%%%%%%%%%%%%
  \section{Quasi-Cartan Companions}



  %%%%%%%%%%%%%%%%%%%%%
  \section{Affine Type}
  
  \begin{proposition}
    The dominance region for an imaginary $\bfg$-vector $\lambda$ does not intersect the interior of the $\bfg$-vector fan.
  \end{proposition}
  \begin{proof}
    Anything in the interior of the $\bfg$-vector fan is a part of a single cluster.
    Moreover, it contains a preprojective or postinjective summand.
    It follows that a sufficiently high power of $\tau$ or $\tau^{-1}$ will move the $\bfg$-vector for that summand far away from the origin and close to the boundary of the $\bfg$-vector fan.
    Thus it cannot lie in the $B$-cone pointed at $\lambda$.
  \end{proof}

  \begin{proposition}
    For acyclic $\tilde B$, the first $n$ entries of $\tilde B\cdot\delta$ is twice the $\bfg$-vector of $\delta$.
  \end{proposition}
  \begin{proof}
    $B+C$ is twice the Euler matrix, multiplying by $\delta$ gives the result.
  \end{proof}


  Let $B$ be an acyclic exchange matrix of affine type with $DB$ skew-symmetric.
  Write $A$ for the Cartan companion of $B$ and note that $A$ has corank 1.
  Consider $B^t$ mutation equivalent to $B$ and $C^t$ the associated $\bfc$-matrix.

  Let $A_0$ denote the Cartan companion of $B$ and write $\delta_0$ for the positive vector spanning the kernel of $A_0$.
  For $t\in\TT_n$, let $A^t:=(C^{\vee,t})^T A_0 C^t$ denote the Reading-Speyer quasi-Cartan companion of $B^t$ \cite[Cor. 3.29]{Combinatorial Frameworks for Cluster Algebras}.
  \begin{theorem}
    Let $\delta^t$ be the absolute value of the kernel of $A^t$.
    Then the primitive purely imaginary $\bfg$-vector direction at the seed $t$ is $-B^t\delta^t/2$.
  \end{theorem}

  \begin{lemma}
    The matrix $\cE^t A^t \cE^t$ is an admissible quasi-Cartan companion of $B^t$ of corank 1.
    Moreover, the kernel of $A^t$ is spanned by a non-negative vector $\delta^t$. 
  \end{lemma}

  Define $\zeta^t := -B^t \delta^t$.
  \begin{lemma}
    The vector $\zeta^t$ is an imaginary $\bfg$-vector.
  \end{lemma}

  \begin{lemma}
    With respect to the seed $t$, the $\bfd$-vector of the imaginary theta basis element $\vartheta_{\zeta^t}$ is $\delta^t$. 
  \end{lemma}

  Define the (opposite) defect vector of signs $\boldsymbol{\varepsilon}^t=(\varepsilon^t_1,\ldots,\varepsilon^t_n)$ where $\varepsilon^t_i=\sgn\,\langle \bfg^{\vee,t}_i,\delta\rangle$ is (the negative of) the defect of the representation of $Q$ with $\bfg$-vector $\bfg^t_i$.
  Let $\kappa^t$ denote the vector $(G^{\vee,t})^T\delta_0$ which spans the kernel of $A^t$.
  \begin{align*}
    E_{k,\varepsilon}^t B^t \boldsymbol{\varepsilon}^t \kappa^t
    &=
    E_{k,\varepsilon}^t B^t F^t_{k,\varepsilon} F^t_{k,\varepsilon} \boldsymbol{\varepsilon}^t \kappa^t\\
    &=
    B^{t'} F^t_{k,\varepsilon} \boldsymbol{\varepsilon}^t \kappa^t\\
    &\stackrel{?}{=} B^{t'} \boldsymbol{\varepsilon}^{t'} \kappa^{t'}
  \end{align*}
  \begin{align*}
    A^t\kappa^t=0=(F^{\vee,t}_{k,\varepsilon_{trop}})^T A^t F^t_{k,\varepsilon_{trop}} F^t_{k,\varepsilon_{trop}} \kappa^t=A^{t'} F^t_{k,\varepsilon_{trop}} \kappa^t
  \end{align*}
  so $\kappa^{t'}=F^t_{k,\varepsilon_{trop}} \kappa^t$

  \begin{lemma}
    $F^t_{k,\varepsilon} \boldsymbol{\varepsilon}^t \kappa^t = \boldsymbol{\varepsilon}^{t'} F^t_{k,\varepsilon_{trop}} \kappa^t$
  \end{lemma}
  \[F^t_{k,\varepsilon} \boldsymbol{\varepsilon}^t \kappa^t - F^t_{k,-\varepsilon} \boldsymbol{\varepsilon}^t \kappa^t=-\varepsilon B^t_{k\bullet} \boldsymbol{\varepsilon}^t \kappa^t \le 0\]
  since $\varepsilon$ is the sign of the $k$-th entry of $B^t \boldsymbol{\varepsilon}^t \kappa^t$


\bibliographystyle{amsalpha}
\bibliography{bibliography}  

\end{document}
