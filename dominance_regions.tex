\documentclass{amsart}
\usepackage{amsmath,amssymb,latexsym,color}
\usepackage[margin=1in]{geometry}

\newtheorem{theorem}{Theorem}
\newtheorem{corollary}[theorem]{Corollary}
\newtheorem{definition}[theorem]{Definition}
\newtheorem{lemma}[theorem]{Lemma}
\newtheorem{remark}[theorem]{Remark}

\numberwithin{theorem}{section}

\newcommand{\bfc}{\mathbf{c}}
\newcommand{\bfg}{\mathbf{g}}

\newcommand{\cC}{\mathcal{C}}
\newcommand{\cI}{\mathcal{I}}
\newcommand{\cP}{\mathcal{P}}
\newcommand{\cQ}{\mathcal{Q}}

\newcommand{\fp}{\mathfrak{p}}

\newcommand{\CC}{\mathbb{C}}
\newcommand{\RR}{\mathbb{R}}
\newcommand{\TT}{\mathbb{T}}
\newcommand{\ZZ}{\mathbb{Z}}

\newcommand{\ol}[1]{{\overline{#1}}}

\newcommand{\Aut}{\operatorname{Aut}}
\newcommand{\diag}{\operatorname{diag}}
\newcommand{\Id}{\operatorname{Id}}
\newcommand{\into}{\hookrightarrow}
\newcommand{\obeta}{{\overline{\beta}}}
\newcommand{\oi}{{\overline{\imath}}}
\newcommand{\ot}{{\overline{t}}}

\title{Dominance Regions for Imaginary $\bfg$-vectors}

\author{Dylan Rupel}
\author{Salvatore Stella}

\begin{document}
  \begin{abstract}
    We investigate the shapes of the polytopes defining the dominance order for imaginary $\bfg$-vectors in rank two.
    We conjecture how our observations might generalize to higher dimensions.
  \end{abstract}
  \maketitle

  \section{Introduction}
  The investigations of this paper are inspired by the dominance order for the tropical labellings for bases of cluster algebras.
  This dominance order is used by Qin to understand how these bases may be deformed in the case when there is a green-to-red sequence of mutations.

  \section{Lattice Mutations}
  Let $B=(b_{ij})$ be an $n\times n$ skew-symmetrizable matrix, i.e. there exists a diagonal integer matrix $D=\diag(d_1,\ldots,d_n)$ so that $DB$ is skew-symmetric.
  For $b\in\ZZ$, write $[b]_+=max(b,0)$.
  Given a sign $\varepsilon\in\{\pm\}$ and $k\in\{1,\ldots,n\}$, define a matrix $E_{k,\varepsilon}=(e_{ij})$ with
  \[ e_{ij}=\begin{cases} 1 & \text{if $i=j\ne k$;}\\ -1 & \text{if $i=j=k$;}\\ [\varepsilon b_{ik}]_+ & \text{if $i\ne j=k$;}\\ 0 & \text{otherwise.} \end{cases} \]
  The index $k\in\{1,\ldots,n\}$ also determines a new skew-symmetrizable matrix $\mu_kB=(b'_{ij})$ given by
  \[ b'_{ij}=\begin{cases} -b_{ij} & \text{if $i=k$ or $j=k$;}\\ b_{ij}+[b_{ik}]_+b_{kj}+b_{ik}[-b_{kj}]_+ & \text{otherwise.} \end{cases} \]
  To record sequences of these matrix mutations, we introduce the labelled $n$-regular rooted tree $\TT_n$ with root vertex $t_0$ and associate skew-symmetrizable matrices $B_t$ for $t\in\TT_n$ satisfying:
  \begin{itemize}
    \item $B_{t_0}=B$;
    \item if $t,t'\in\TT_n$ are joined by an edge labelled $k$, then $B_{t'}=\mu_k B_t$.
  \end{itemize}

  Define $\phi_k:\ZZ^n\to\ZZ^n$ as the piecewise-linear map $\phi_k(\lambda)=\begin{cases} E_{k,+}\lambda & \text{if $\lambda_k\ge0$;}\\ E_{k,-}\lambda & \text{if $\lambda_k<0$.} \end{cases}$.
  Define the piecewise-linear automorphisms $\phi_t:\ZZ^n\to\ZZ^n$ by
  \begin{itemize}
    \item $\phi_{t_0}=\Id$;
    \item if $t,t'\in\TT_n$ are joined by an edge labelled $k$, then $\phi_{t'}=\phi_k \phi_t$.
  \end{itemize}
  For $\lambda\in\ZZ^n$, define the $B$-cone $\cC(\lambda)=\lambda+B\RR_{\ge0}^n$ pointed at $\lambda$.
  \begin{definition}
    For $\lambda\in\ZZ^n$, define the \emph{dominance polytope} 
    \[ \cP(\lambda) = \bigcap_{t\in\TT_n} \phi_t^{-1} \cC(\phi_t \lambda). \]
  \end{definition}
  Our goal is to understand the dominance polytopes for all $\lambda$.

  \section{Rank Two}
  Fix integers $b,c>0$ and assume $bc\ge4$.
  Let $B=\left[ \begin{array}{cc} 0 & -b \\ c & 0 \end{array} \right]$.
  Here we may take $\TT_2$ to have vertices labelled $t_\ell$ for $\ell\in\ZZ$, with $t_{2r}$ joined to $t_{2r+1}$ by an edge labelled $1$ and $t_{2r-1}$ joined to $t_{2r}$ by an edge labelled $2$.

  Observe that $\phi_{t_2}=\phi_{t_{-2}}^{-1}$ is the piecewise-linear map given by
  \begin{equation}
    \label{eq:forward two step mutation}
    \phi_{t_2}(\lambda)=\begin{cases} ((bc-1)\lambda_1+b\lambda_2, -c\lambda_1-\lambda_2) & \text{if $\lambda_1\ge 0$ and $c\lambda_1+\lambda_2\ge 0$;}\\ (-\lambda_1, -c\lambda_1-\lambda_2) & \text{if $\lambda_1\ge 0$ and $c\lambda_1+\lambda_2<0$;}\\ (-\lambda_1+b\lambda_2, -\lambda_2) & \text{if $\lambda_1<0$ and $\lambda_2\ge 0$;}\\ (-\lambda_1,-\lambda_2) & \text{if $\lambda_1<0$ and $\lambda_2<0$;}\end{cases}
  \end{equation}
  and $\phi_{t_2}^{-1}=\phi_{t_{-2}}$ is the piecewise-linear map given by
  \begin{equation}
    \label{eq:backward two step mutation}
    \phi_{t_2}^{-1}(\lambda)=\begin{cases} (-\lambda_1-b\lambda_2, c\lambda_1+(bc-1)\lambda_2) & \text{if $\lambda_2\le 0$ and $\lambda_1+b\lambda_2\le 0$;}\\ (-\lambda_1-b\lambda_2, -\lambda_2) & \text{if $\lambda_2\le 0$ and $\lambda_1+b\lambda_2>0$;}\\ (-\lambda_1, c\lambda_1-\lambda_2) & \text{if $\lambda_2>0$ and $\lambda_1\le 0$;}\\ (-\lambda_1,-\lambda_2) & \text{if $\lambda_2>0$ and $\lambda_1>0$;}\end{cases}
  \end{equation}

  By an eigenvector of a piecewise-linear map $\phi$, we mean a vector $\lambda$ such that for each $k\ge0$, there exists a scalar $\alpha_k$ so that $\phi^k(\lambda)=\alpha_k\lambda$.
  \begin{lemma}
    Any nonzero eigenvector of $\phi_{t_2}$ is a positive multiple of one of the vectors $\big(2b,-bc\pm\sqrt{bc(bc-4)}\big)$.
  \end{lemma}
  \begin{proof}
    First observe that the equation $\phi_{t_2}^k(\lambda)=\alpha_k\lambda$ cannot be satisfied for all $k\ge0$ unless $\lambda_1\ge 0$ and $c\lambda_1+\lambda_2\ge 0$.
    In this region, $\phi_{t_2}$ is linear with eigenvalue $\alpha$ satisfying $\alpha^2-(bc-2)\alpha+1=0$, i.e. $\alpha=\frac{bc-2\pm\sqrt{bc(bc-4)}}{2}$.
    We thus require that 
    \[\frac{bc-2\pm\sqrt{bc(bc-4)}}{2}\lambda_1=(bc-1)\lambda_1+b\lambda_2 \qquad\text{and}\qquad \frac{bc-2\pm\sqrt{bc(bc-4)}}{2}\lambda_2= -c\lambda_1-\lambda_2,\]
    or
    \[\frac{-bc\pm\sqrt{bc(bc-4)}}{2b}\lambda_1=\lambda_2 \qquad\text{and}\qquad \frac{-bc\mp\sqrt{bc(bc-4)}}{2c}\lambda_2=\lambda_1.\]
    As these represent the same relation, the result follows.
  \end{proof}

  Write $\cI \subset \RR^2$ for the \emph{imaginary cone} spanned by the vectors $(2b,-bc\pm\sqrt{bc(bc-4)})$.
  \begin{lemma}
    The vectors $(2b, -bc+\sqrt{bc(bc-4)})$ (resp. $(2b, -bc-\sqrt{bc(bc-4)})$) and $(2c, bc+\sqrt{bc(bc-4)})$ (resp. $(2c, bc-\sqrt{bc(bc-4)})$) are perpendicular.
  \end{lemma}

  \begin{lemma}
    If $\lambda\in\ZZ^2\setminus\cI$, then $\cP(\lambda)=\{\lambda\}$.
  \end{lemma}
  \begin{proof}
    It is enough to consider $\phi_{t_{-2}}^{-1}\cC(\phi_{t_{-2}}\lambda) \cap \cC(\lambda) \cap \phi_{t_2}^{-1}\cC(\phi_{t_2}\lambda)$, we leave the details to the reader.
  \end{proof}

  \begin{lemma}
    \label{le:one direction}
    For $\lambda\in\cI\cap\ZZ^2$, 
    \begin{enumerate}
      \item $\cP_+(\lambda):=\bigcap_{i \ge 0}\phi_{t_i}^{-1}\cC(\phi_{t_i}\lambda)$ is the quadrilateral with corner vertices $\lambda$, $(\frac{2-bc-\sqrt{bc(bc-4)}}{2}\lambda_1-\frac{bc+\sqrt{bc(bc-4)}}{2c}\lambda_2,0)$, $(0,\frac{bc+\sqrt{bc(bc-4)}}{2b}\lambda_1+\lambda_2)$, and $(\frac{2-bc-\sqrt{bc(bc-4)}}{2}\lambda_1-b\lambda_2,\lambda_2)$.
      \item $\cP_-(\lambda):=\bigcap_{i \le 0}\phi_{t_i}^{-1}\cC(\phi_{t_i}\lambda)$ is the quadrilateral with corner vertices $\lambda$, $(\lambda_1+\frac{bc+\sqrt{bc(bc-4)}}{2c}\lambda_2,0)$, $(0,-\frac{bc+\sqrt{bc(bc-4)}}{2b}\lambda_1+\frac{2-bc-\sqrt{bc(bc-4)}}{2}\lambda_2)$, and $(\lambda_1,-c\lambda_1+\frac{2-bc-\sqrt{bc(bc-4)}}{2}\lambda_2)$.
    \end{enumerate}
  \end{lemma}
  \begin{proof}
    We prove (1) as (2) is obtained by interchanging $b$ with $c$ and swapping all ordered pairs.  
    We first show that $\cC(\lambda) \cap \phi_{t_i}^{-1}\cC(\phi_{t_i}\lambda)\supsetneq \cC(\lambda) \cap \phi_{t_j}^{-1}\cC(\phi_{t_j}\lambda)$ for $0\le i<j$.
  \end{proof}

  \begin{theorem}
    For $\lambda\in\cI\cap\ZZ^2$, there are five classes of dominance polytopes.
    \begin{enumerate}
      \item if $\lambda$ lies in the interior of the cone spanned by the vectors $(2b,-bc-\sqrt{bc(bc-4)})$ and $(2,-c)$, then $\cC(\lambda)$ is the trapezoid with corner vertices $\lambda$, $(\frac{2-bc-\sqrt{bc(bc-4)}}{2}\lambda_1-\frac{bc+\sqrt{bc(bc-4)}}{2c}\lambda_2,0)$, $(0,\frac{bc+\sqrt{bc(bc-4)}}{2b}\lambda_1+\lambda_2)$, and ???.
      \item if $\lambda$ lies on the ray spanned by $(2,-c)$, say $\lambda=(2\ell,-c\ell)$, then $\cC(\lambda)$ is the triangle with corner vertices $\lambda$, $(2\ell-\frac{bc+\sqrt{bc(bc-4)}}{2}\ell,0)$, and $(0,\frac{\sqrt{bc(bc-4)}}{b}\ell)$.
      \item if $\lambda$ lies in the interior of the cone spanned by the vectors $(2,-c)$ and $(b,-2)$, then $\cC(\lambda)$ is the kite with corner vertices $\lambda$, $(\lambda_1+\frac{bc+\sqrt{bc(bc-4)}}{2c}\lambda_2,0)$, $\frac{bc+\sqrt{bc(bc-4)}}{2\sqrt{bc(bc-4)}}(2\lambda_1+b\lambda_2,c\lambda_1+2\lambda_2)$, and $(0,\frac{bc+\sqrt{bc(bc-4)}}{2b}\lambda_1+\lambda_2)$.
      \item if $\lambda$ lies on the ray spanned by $(b,-2)$, say $\lambda=(b\ell,-2\ell)$, then $\cC(\lambda)$ is the triangle with corner vertices $\lambda$, $(-\frac{\sqrt{bc(bc-4)}}{c}\ell,0)$, and $(0,\frac{bc+\sqrt{bc(bc-4)}}{2}\ell-2\ell)$.
      \item if $\lambda$ lies in the interior of the cone spanned by the vectors $(b,-2)$ and $(2b,-bc+\sqrt{bc(bc-4)})$, then $\cC(\lambda)$ is the triangle with corner vertices $\lambda$, $(\lambda_1+\frac{bc+\sqrt{bc(bc-4)}}{2c}\lambda_2,0)$, $(0,-\frac{bc+\sqrt{bc(bc-4)}}{2b}\lambda_1+\frac{2-bc-\sqrt{bc(bc-4)}}{2}\lambda_2)$, and ???.
    \end{enumerate}
  \end{theorem}
  \begin{proof}
    Following Lemma~\ref{le:one direction}, we see that the intersection $\cP_+(\lambda)\cap\cP_-(\lambda)$ degenerates to a triangle precisely when
    \[\frac{2-bc-\sqrt{bc(bc-4)}}{2}\lambda_1-\frac{bc+\sqrt{bc(bc-4)}}{2c}\lambda_2=\lambda_1+\frac{bc+\sqrt{bc(bc-4)}}{2c}\lambda_2\]
    or 
    \[\frac{bc+\sqrt{bc(bc-4)}}{2b}\lambda_1+\lambda_2=-\frac{bc+\sqrt{bc(bc-4)}}{2b}\lambda_1+\frac{2-bc-\sqrt{bc(bc-4)}}{2}\lambda_2.\]
    The first equations reduces to $c\lambda_1+2\lambda_2=0$, while the second reduces to $2\lambda_1+b\lambda_2=0$.
    In particular, the rays spanned by $(2,-c)$ and $(b,-2)$ determine this change of state in the dominance regions $\cP(\lambda)$.
  \end{proof}

  \begin{remark}
    Note that the rays which separate the regions inside $\cI$ correspond exactly to the columns of the associated Cartan matrix $\left[ \begin{array}{cc} 2 & -b \\ -c & 2 \end{array} \right]$.
    This unexpected coincidence is our reason for writing these results.
  \end{remark}

\end{document}
