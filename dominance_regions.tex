\documentclass{amsart}
\usepackage{amsmath}
\usepackage{amssymb}
\usepackage{latexsym}
\usepackage{color}
%\usepackage{ulem}
\usepackage[margin=1in]{geometry}
\usepackage{tikz}
\usetikzlibrary{math}
\usepackage[bookmarks=true, bookmarksopen=true, bookmarksdepth=3,bookmarksopenlevel=2, colorlinks=true, linkcolor=blue, citecolor=blue, filecolor=blue, menucolor=blue, urlcolor=blue]{hyperref}

\usepackage[draft]{say}
\newcommand{\sayDR}[1]{\say[DR]{\color{red}{\bf DR:}\;#1}}
\newcommand{\saySS}[1]{\say[SS]{\color{blue}{\bf SS:}\;#1}}

\newtheorem{theorem}{Theorem}
\newtheorem{corollary}[theorem]{Corollary}
\newtheorem{definition}[theorem]{Definition}
\newtheorem{lemma}[theorem]{Lemma}
\newtheorem{proposition}[theorem]{Proposition}
\newtheorem{remark}[theorem]{Remark}
\newtheorem{conjecture}[theorem]{Conjecture}
\newtheorem{question}{Question}

\numberwithin{theorem}{section}

\newcommand{\bfa}{\boldsymbol{a}}
\newcommand{\bfc}{\boldsymbol{c}}
\newcommand{\bfg}{\boldsymbol{g}}
\newcommand{\bfr}{\boldsymbol{r}}
\newcommand{\bfx}{\boldsymbol{x}}
\newcommand{\bfy}{\boldsymbol{y}}

\newcommand{\cA}{\mathcal{A}}
\newcommand{\cC}{\mathcal{C}}
\newcommand{\cD}{\mathcal{D}}
\newcommand{\cG}{\mathcal{G}}
\newcommand{\cI}{\mathcal{I}}
\newcommand{\cP}{\mathcal{P}}
\newcommand{\cQ}{\mathcal{Q}}
\newcommand{\cR}{\mathcal{R}}
\newcommand{\cS}{\mathcal{S}}

\newcommand{\fp}{\mathfrak{p}}

\newcommand{\CC}{\mathbb{C}}
\newcommand{\QQ}{\mathbb{Q}}
\newcommand{\RR}{\mathbb{R}}
\newcommand{\TT}{\mathbb{T}}
\newcommand{\ZZ}{\mathbb{Z}}

\newcommand{\ol}[1]{{\overline{#1}}}
\newcommand{\vv}[1]{{{}^\vee \! #1}}

\newcommand{\Aut}{\operatorname{Aut}}
\newcommand{\Col}{\operatorname{Col}}
\newcommand{\diag}{\operatorname{diag}}
\newcommand{\dpt}{\operatorname{dp}}
\newcommand{\hgt}{\operatorname{ht}}
\newcommand{\Id}{\operatorname{Id}}
\newcommand{\into}{\hookrightarrow}
\newcommand{\obeta}{{\overline{\beta}}}
\newcommand{\oi}{{\overline{\imath}}}
\newcommand{\ot}{{\overline{t}}}
\newcommand{\rsh}{{\operatorname{rsh}}}
\newcommand{\sh}{{\operatorname{sh}}}
\newcommand{\WA}{{W\!\!A}}
\newcommand{\wt}{{\operatorname{wt}}}

\newcommand{\kk}{\Bbbk}

\title{Dominance Regions for Rank Two Cluster Algebras}

\author{Dylan Rupel}
\author{Salvatore Stella}

\begin{document}


  \begin{abstract}
    We investigate the shapes of the polytopes defining the dominance order for $\bfg$-vectors in cluster algebras of rank two.
    We arrive at a complete description: 
    \begin{itemize}
      \item cluster monomials are never deformable;
      \item the imaginary dominance polytopes are kites, triangles, or trapezoids which extend far outside the imaginary cone; the triangles appearing along rays corresponding to columns of the associated Cartan matrix.
    \end{itemize}
  \end{abstract}
  \maketitle

\begin{figure}[h!]
  \centering
    \newcommand{\setconstants}{
      \tikzmath{\b = 3; \c = 6 / \b; \s = sqrt(\b * \c * (\b * \c -4)); \bcmo = \b * \c - 1; \bcms = \b * \c - \s; \bcps = \b * \c + \s;}
      \tikzmath{ \xmin = -6; \xmax = 5; \ymin = -4; \ymax = 8;}
      \clip ({\xmin-0.2},{\ymin-0.2}) rectangle ({\xmax+0.2},{\ymax+0.2});
    }
    \newcommand{\boundingrays}{
      \draw [color=lightgray, thick] (2*\xmin,0) -- (2*\xmax,0);
      \draw [color=lightgray, thick] (0,2*\ymin) -- (0, 2*\ymax);
      \draw [color=lightgray, thick, dashed] (0,0) -- (2*\xmax, -\xmax * \bcms / \b); 
      \draw [color=lightgray, thick, dashed] (0,0) -- ( - 4 * \b * \ymin / \bcps, 2*\ymin); 
    }
  \begin{tikzpicture}[scale=.5]
      \setconstants;
      \foreach \x in {\xmin,...,\xmax} {
        \foreach \y in {\ymin,...,\ymax} {
          \fill[color=gray] (\x,\y) circle (0.05);
        }
      };
      \tikzmath{ \x=4; \y=-3;};
      \boundingrays;
      \fill [fill=gray!30, fill opacity=0.4] (\x,\y) -- ( \x+\b*\y, \y) -- ( \x+\b*\y , -\c*\x - \b*\c*\y + \y) -- (\bcps*\x/\s + \bcps*\b*\y/2/\s, -\bcps*\c*\x/2/\s - \bcps*\b*\c*\y/2/\s + \bcps*\y/\s) -- cycle;
      \fill [fill=red!30, fill opacity=0.5]  ( \x+\b*\y , -\c*\x - \b*\c*\y + \y)  -- (\bcps*\x/\s + \bcps*\b*\y/2/\s, -\bcps*\c*\x/2/\s - \bcps*\b*\c*\y/2/\s + \bcps*\y/\s) -- ({\bcms/(2*\c)*\y+\x},0) --  (0,{\bcps/(2*\b)*(\x+\b*\y)+(-\c*\x - \b*\c*\y + \y)}) -- cycle;
      \fill [fill=darkgreen!30, fill opacity=0.5] (\x,\y) -- ({\x+\bcps/(2*\c)*\y},0) -- ({\bcps/(2*\s)*(2*\x+\b*\y)},{\bcps/(2*\s)*(\c*\x+2*\y)}) -- (0,{\bcps/(2*\b)*\x+\y}) -- cycle;
      \draw [red, thick, join=round]  ( \x+\b*\y , -\c*\x - \b*\c*\y + \y)  -- (\bcps*\x/\s + \bcps*\b*\y/2/\s, -\bcps*\c*\x/2/\s - \bcps*\b*\c*\y/2/\s + \bcps*\y/\s) -- ({\bcms/(2*\c)*\y+\x},0) -- (0,{\bcps/(2*\b)*(\x+\b*\y)+(-\c*\x - \b*\c*\y + \y)}) -- cycle;
      \draw [darkgreen, thick, join=round] (\x,\y) -- ({\x+\bcps/(2*\c)*\y},0) -- ({\bcps/(2*\s)*(2*\x+\b*\y)},{\bcps/(2*\s)*(\c*\x+2*\y)}) -- (0,{\bcps/(2*\b)*\x+\y}) -- cycle;
      \draw [thick, join=round] (\x , \y) -- ( \x+\b*\y, \y) -- ( \x+\b*\y , -\c*\x - \b*\c*\y + \y);;
      \draw [thick, dotted, join=round] ( \x+\b*\y , -\c*\x - \b*\c*\y + \y) --  (\bcps*\x/\s + \bcps*\b*\y/2/\s, -\bcps*\c*\x/2/\s - \bcps*\b*\c*\y/2/\s + \bcps*\y/\s) --  (\x , \y);
      \draw [thick, dashed, join=round] ( \x+\b*\y , -\c*\x - \b*\c*\y + \y) -- (0,0) --  (\x , \y);
      \draw [thick, fill=white]  ( \x+\b*\y , -\c*\x - \b*\c*\y + \y) circle (5pt) node[above left]{$\lambda'$};
      \fill (\x , \y)  circle (5pt) node[below right]{$\lambda$};
      \foreach \x/\y in { 1/-3, -2/-3, -5/-3, -2/-1, -5/-1, -5/1, -2/3, -5/3, -2/5, -5/5 } {
        \draw[color=blue, fill=white] (\x,\y) circle (4pt);
      };
      \foreach \x/\y in { 1/-1, 1/1, -2/1 } {
        \fill[color=blue] (\x,\y) circle (4pt);
      };
      \draw (-3.5,-3.5) node{$\cS_\lambda$};
      \draw [darkgreen](2.5,1) node{$\cP_\lambda$};
      \draw [red](-2,6.5) node{$\cP'_\lambda$};
  \end{tikzpicture}
\end{figure}
  \section{Introduction}
  Cluster algebras were introduced by Fomin and Zelevinsky as a tool in the study of Lusztig's dual canonical bases.
  Since their inception they have found application in a variety of different areas in mathematics, nevertheless a fundamental problem in the theory remains constructing bases with ``good'' properties. 
  
  Over time several bases for cluster algebras have been constructed in a variety of generalities \cite{CI12,DT13,SZ04,LLZ14, Dup11,Dup12,Pla13, MSW13,T14,BZ14,GHKK18}.
  All of them share a property of being ``pointed''.
  Being pointed turned out to be a desirable feature for a basis to have and a natural question is to find all bases enjoying this property.
  Recently Qin studied the deformability of pointed bases whenever the cluster algebra has full rank \cite{qin}. 
  As no explicit calculation is carried out in his work, we aim here at describing explicitly the space of deformability for pointed bases in rank two, i.e. when clusters contain a pair of mutable cluster variables.
  In this setting, frozen variables do not carry any additional information and we can work in the coefficient-free case. 

  Fix integers $b,c>0$.  The cluster algebra $\cA(b,c)$ is the $\ZZ$-subalgebra of $\QQ(x_0,x_1)$ generated by \emph{cluster variables} $x_m$, $m\in\ZZ$, defined recursively by
  \begin{equation}
    x_{m-1}x_{m+1}=\begin{cases} x_m^b+1 & \text{if $m$ is even;}\\ x_m^c+1 & \text{if $m$ is odd.} \end{cases}
  \end{equation}
  By the Laurent Phenomenon \cite{fomin-zelevinsky}, each $x_m$ is actually an element of $\ZZ[x_k^{\pm1},x_{k+1}^{\pm1}]$ for any $k\in\ZZ$.
  Moreover, these Laurent polynomials are known to have positive coefficients \cite{LLZ,lee-schiffler,GHKK}.

  Our goal in this note is to explicitly compute the possible bases which are pointed with respect to the embedding $\cA(b,c)\into\ZZ[x_k^{\pm1},x_{k+1}^{\pm1}]$ for every $k\in\ZZ$.
  That is, a basis element with \emph{$\bfg$-vector} $\lambda=(\lambda_0,\lambda_1)\in\ZZ^2$ when expanded in terms of $\{x_0,x_1\}$ can be written in the form
  \begin{equation}
    \label{eq:pointed}
    x_0^{\lambda_0}x_1^{\lambda_1}\sum\limits_{\alpha_0,\alpha_1 \ge 0} \rho_{\alpha_0,\alpha_1} x_0^{-b\alpha_0} x_1^{c\alpha_1}
  \end{equation}
  with $\rho_{0,0}=1$.
  Being \emph{pointed} means an analogous structure reproduces after expanding in terms of any pair $\{x_k,x_{k+1}\}$.
  %The collection of all $\bfg$-vectors $\lambda\in\ZZ^2$ parametrize the elements of any pointed basis.

  Important examples of pointed elements are the \emph{cluster monomials} of $\cA(b,c)$, i.e. the elements that can be written in the form $x_k^{\alpha_k}x_{k+1}^{\alpha_{k+1}}$ for some $k\in\ZZ$ and $\alpha_k,\alpha_{k+1}$ non-negative integers.
  Indeed, by \cite[Lemma 3.4.12]{qin} cluster monomials are part of any pointed basis in any cluster algebra.
  We achieve the same conclusion in our setting by elementary calculations (cf. Lemma~\ref{le:cluster monomials}).

  Elements of any pointed basis are parametrized by $\ZZ^2$ thought of as the collection of possible $\bfg$-vectors.
  Qin introduced a partial order $\preceq$ on $\bfg$-vectors called the \emph{dominance order} and showed that it provides a characterization of pointed bases.
  We restate his results in the generality needed for this paper and using our notation.
  \begin{theorem}
    \label{th:dominance}
    \cite[Theorem 1.2.1]{qin}
    Let $x=\{x_\lambda\}$ and $y=\{y_\lambda\}$ be pointed bases of $\cA(b,c)$.
    Then for each $\lambda\in\ZZ^2$, there exist scalars $q_{\lambda,\mu}$ for $\mu\prec\lambda$ such that
    \[y_\lambda=x_\lambda+\sum_{\mu\prec\lambda} q_{\lambda,\mu} x_\mu.\]
    Moreover, any choice of scalars $q_{\lambda,\mu}$, for $\mu\prec\lambda$, $\lambda\in\ZZ^2$, as above provides a basis of $\cA(b,c)$.
  \end{theorem}

  When $bc\le3$, the cluster algebra $\cA(b,c)$ will be of finite-type and cluster monomials form its only pointed basis.
  We will therefore assume that $bc\ge4$.
  In this case we give an explicit description of the dominance relation among $\bfg$-vectors.
  Specifically, we show that the $\bfg$-vector $\lambda$ dominates the collection of $\bfg$-vectors of the form $\lambda+(b \alpha_0 ,c \alpha_1)$, $\alpha_0,\alpha_1\in\ZZ$, inside its \emph{dominance region} $\cP_\lambda$.
  %In this case, we compute the possible transitions between any two pointed bases.
  %The collection of $\bfg$-vectors which can appear in the deformation of a pointed basis is the \emph{dominance region}.
  \begin{theorem}
    \label{th:dominance inequalities}
    The dominance region $\cP_\lambda$ pointed at $\lambda=(\lambda_0,\lambda_1)$ is the polygon consisting of those $\mu=(\mu_0,\mu_1)\in\RR^2$ satisfying $\mu_0 \leq \lambda_0, \mu_1 \geq\lambda_1$, and the following inequalities:
    {
      \everymath={\displaystyle}
      \def\arraystretch{2.8}
      \[
        \begin{array}{rcccl}
          0 & \leq & \frac{b c-\sqrt{b c (b c-4)}}{2 b}(\mu_0-\lambda_0)+(\mu_1-\lambda_1) & \leq & \max\left(-c\lambda_0-\frac{b c+\sqrt{b c (b c-4)}}{2b}b\lambda_1,0\right)
          \\
          \min\left(-c\lambda_0,0\right) & \leq & \frac{b c-\sqrt{b c (b c-4)}}{2 b}(\mu_0-\lambda_0)-(\mu_1-\lambda_1) & \leq & 0
          \\
          0 & \leq &  -(\mu_0-\lambda_0)-\frac{b c-\sqrt{b c (b c-4)}}{2 c}(\mu_1-\lambda_1) & \leq & \max\left(\frac{b c+\sqrt{b c (b c-4)}}{2c}c\lambda_0+b\lambda_1,0\right)
          \\
          \min\left(b \lambda_1,0\right) & \leq & (\mu_0-\lambda_0) - \frac{b c-\sqrt{b c (b c-4)}}{2 c} (\mu_1-\lambda_1) & \leq & 0
        \end{array}
      \]
    }
    %\begin{align}
    %  0 & \leq \frac{b c-\sqrt{b c (b c-4)}}{2 b}(\mu_0-\lambda_0)+(\mu_1-\lambda_1) && \leq \max\left(-c\lambda_0-\frac{b c+\sqrt{b c (b c-4)}}{2b}b\lambda_1,0\right)
    %  \\
    %  \min\left(-c\lambda_0,0\right) & \leq \frac{b c-\sqrt{b c (b c-4)}}{2 b}(\mu_0-\lambda_0)-(\mu_1-\lambda_1) && \leq 0
    %  \\
    %  0 & \leq -(\mu_0-\lambda_0)-\frac{b c-\sqrt{b c (b c-4)}}{2 c}(\mu_1-\lambda_1) && \leq \max\left(\frac{b c+\sqrt{b c (b c-4)}}{2c}c\lambda_0+b\lambda_1,0\right)
    %  \\
    %  \min\left(b \lambda_1,0\right) & \leq (\mu_0-\lambda_0) - \frac{b c-\sqrt{b c (b c-4)}}{2 c} (\mu_1-\lambda_1) && \leq 0
    %\end{align}
    
  \end{theorem}

  Write $\cI \subset \RR^2$ for the \emph{imaginary cone} (positively) spanned by the vectors $\big(2b,-bc\pm\sqrt{bc(bc-4)}\big)$.
  \begin{corollary}
    \label{cor:dominance vertices}
    There are six classes of dominance polygons.
    \begin{enumerate}
      \item If $\lambda$ lies outside of $\cI$, i.e. if it is the $\bfg$-vector of a cluster monomial, then $\cP_\lambda$ is the point $\lambda$. 
      \item If $\lambda$ lies in the interior of the cone spanned by the vectors $(2b,-bc-\sqrt{bc(bc-4)})$ and $(2,-c)$, then $\cP_\lambda$ is the trapezoid with  vertices $\lambda$, $-\frac{bc+\sqrt{bc(bc-4)}}{2c}\big(\frac{bc+\sqrt{bc(bc-4)}}{2b}\lambda_0+\lambda_1,0\big)$, $\big(0,\frac{bc+\sqrt{bc(bc-4)}}{2b}\lambda_0+\lambda_1\big)$, and $\frac{bc+\sqrt{bc(bc-4)}}{2\sqrt{bc(bc-4)}}\big(-(bc-2)\lambda_0-b\lambda_1,c\lambda_0+2\lambda_1\big)$.
      \item If $\lambda$ lies on the ray spanned by $(2,-c)$, say $\lambda=(2\ell,-c\ell)$, then $\cP_\lambda$ is the triangle with  vertices $\lambda$, $\big(2\ell-\frac{bc+\sqrt{bc(bc-4)}}{2}\ell,0\big)$, and $\big(0,\frac{\sqrt{bc(bc-4)}}{b}\ell\big)$.
      \item If $\lambda$ lies in the interior of the cone spanned by the vectors $(2,-c)$ and $(b,-2)$, then $\cP_\lambda$ is the kite with  vertices $\lambda$, $\big(\lambda_0+\frac{bc+\sqrt{bc(bc-4)}}{2c}\lambda_1,0\big)$, $\frac{bc+\sqrt{bc(bc-4)}}{2\sqrt{bc(bc-4)}}(2\lambda_0+b\lambda_1,c\lambda_0+2\lambda_1)$, and $\big(0,\frac{bc+\sqrt{bc(bc-4)}}{2b}\lambda_0+\lambda_1\big)$.
      \item If $\lambda$ lies on the ray spanned by $(b,-2)$, say $\lambda=(b\ell,-2\ell)$, then $\cP_\lambda$ is the triangle with  vertices $\lambda$, $\big(-\frac{\sqrt{bc(bc-4)}}{c}\ell,0\big)$, and $\big(0,\frac{bc+\sqrt{bc(bc-4)}}{2}\ell-2\ell\big)$.
      \item If $\lambda$ lies in the interior of the cone spanned by the vectors $(b,-2)$ and $(2b,-bc+\sqrt{bc(bc-4)})$, then $\cP_\lambda$ is the trapezoid with  vertices $\lambda$, $\big(\lambda_0+\frac{bc+\sqrt{bc(bc-4)}}{2c}\lambda_1,0\big)$, $-\frac{bc+\sqrt{bc(bc-4)}}{2b}\big(0,\lambda_0+\frac{bc+\sqrt{bc(bc-4)}}{2c}\lambda_1\big)$, and $\frac{bc+\sqrt{bc(bc-4)}}{2\sqrt{bc(bc-4)}}\big(2\lambda_0+b\lambda_1,-c\lambda_0-(bc-2)\lambda_1\big)$.
    \end{enumerate}
  \end{corollary}

  \begin{remark}
    Note that the rays which separate the regions inside $\cI$ correspond exactly to the columns of the associated Cartan matrix $\left[ \begin{array}{cc} 2 & -b \\ -c & 2 \end{array} \right]$.
    This unexpected coincidence is one of our reasons for deciding to write down these results.
  \end{remark}

  A number of bases for rank two cluster algebras with various defining properties have been constructed.
  A natural starting basis is the so-called \emph{greedy basis} in which all $\rho_{\alpha_0,\alpha_1}$ are non-negative and these coefficients are chosen to be minimal.
  It is known that this choice coincides with the theta basis \cite{GHKK, CGMMRSW}.
  The \emph{standard monomial bases} of $\cA(b,c)$ come from natural identifications of $\cA(b,c)$ with the subalgebras of $\ZZ[x_0^{\pm1},x_1^{\pm1}]$ generated by $x_{k-1}$, $x_k$, $x_{k+1}$, and $x_{k+2}$ for $k\in\ZZ$.
  However, these bases are not mutation invariant.
  The \emph{triangular basis} is defined by a triangular transition from the standard monomial basis.
  The \emph{generic basis} is defined from the categorification of $\cA(b,c)$ using generic representations of an associated quiver.
  It would be useful to explicitly (or combinatorially) compute the transitions between these known bases.
  Interestingly, explicit calculations reveal that the $\bfg$-vectors actually appearing in these transitions are far smaller than the entire dominance polygon.

  Given a Laurent polynomial in $\ZZ[x_0^{\pm1},x_1^{\pm1}]$, its \emph{support} is the set of its exponent vectors inside $\ZZ^2$.
  Given a $\bfg$-vector $\lambda$, in the next result we identify a polygon $\cS_\lambda$ whose lattice points are the maximum possible support of a pointed basis element $x_\lambda$. 
  \begin{theorem}
    \label{th:maximum support}
    Let $\lambda=(\lambda_0,\lambda_1)\in\ZZ^2$ be the $\bfg$-vector for a pointed basis element $x_\lambda$.
    \begin{enumerate}
      \item If $\lambda$ lies outside of $\cI$, then the support of $x_\lambda$ is precisely the points of the form $\lambda+(b \alpha_0 ,c \alpha_1)$, $\alpha_0,\alpha_1\in\ZZ$, inside $\cS_\lambda$ given as follows:
        \begin{enumerate}
          \item If $0\le\lambda_0,\lambda_1$, then $\cS_\lambda$ is just the point $\lambda$.
          \item If $\lambda_1 < 0$ and $0\le\lambda_0+b\lambda_1$, then $\cS_\lambda$ is the segment joining $\lambda$ and $(\lambda_0+b\lambda_1,\lambda_1)$.
          \item If $\lambda_0 < 0$ and $0\le\lambda_1$, then $\cS_\lambda$ is the segment joining $\lambda$ and $(\lambda_0,-c\lambda_0+\lambda_1)$.
          \item If $\lambda_0,\lambda_1 < 0$, then $\cS_\lambda$ has  vertices $\lambda$, $(\lambda_0+b\lambda_1,\lambda_1)$, $\big(\lambda_0+b\lambda_1,-c\lambda_0-(bc-1)\lambda_1\big)$, and ${(\lambda_0,-c\lambda_0+\lambda_1)}$.
          \item If $\lambda_0 > 0$, $\lambda_0+b\lambda_1 < 0$, and $-c\lambda_0-(bc-1)\lambda_1\le 0$, then $\cS_\lambda$ has  vertices $\lambda$, $(\lambda_0+b\lambda_1,\lambda_1)$, $\big(\lambda_0+b\lambda_1,-c\lambda_0-(bc-1)\lambda_1\big)$, and $\big((bc+1)\lambda_0-b^2c\lambda_1,-c\lambda_0-(bc-1)\lambda_1\big)$.
          \item If $\lambda_0 > 0$, $\lambda_0+b\lambda_1 < 0$, and $0 < -c\lambda_0-(bc-1)\lambda_1$, then $\cS_\lambda$ has  vertices $\lambda$, $(\lambda_0+b\lambda_1,\lambda_1)$, $\big(\lambda_0+b\lambda_1,-c\lambda_0-(bc-1)\lambda_1\big)$, and $(0,0)$.  Here the point $(0,0)$ and its adjacent open segments are excluded from $\cS_\lambda$.
        \end{enumerate}
      \item If $\lambda$ lies inside of $\cI$, then the support of $x_\lambda$ is contained in $\cS_\lambda$ with  vertices $\lambda$, $(\lambda_0+b\lambda_1,\lambda_1)$, $\big(\lambda_0+b\lambda_1,-c\lambda_0-(bc-1)\lambda_1\big)$, and $\frac{bc+\sqrt{bc(bc-4)}}{2\sqrt{bc(bc-4)}}\big(2\lambda_0+b\lambda_1,-c\lambda_0-(bc-2)\lambda_1\big)$.
        Moreover, there is an element pointed at $\lambda$ whose support is precisely the points of the form $\lambda+(b \alpha_0 ,c \alpha_1)$, $\alpha_0,\alpha_1\in\ZZ$, inside $\cS_\lambda$.
    \end{enumerate}
  \end{theorem}

\begin{figure}[h!]
  \centering
    \newcommand{\setconstants}{
      \tikzmath{\b = 3.5; \c = 4.4 / \b; \s = sqrt(\b * \c * (\b * \c -4)); \bcmo = \b * \c - 1; \bcms = \b * \c - \s; \bcps = \b * \c + \s;}
      \tikzmath{ \xmin = -6; \xmax = 6; \ymin = -6; \ymax = 6;}
      \clip (\xmin,\ymin) rectangle (\xmax,\ymax);
    }
    \newcommand{\boundingrays}{
      \draw [color=lightgray, thick] (2*\xmin,0) -- (2*\xmax,0);
      \draw [color=lightgray, thick] (0,2*\ymin) -- (0, 2*\ymax);
      \draw [color=lightgray, thick] (0,0) -- (2*\xmax, -2*\xmax / \b);
      \draw [color=lightgray, thick] (0,0) -- (2*\xmax, -2*\xmax * \c / \bcmo );
      \draw [color=lightgray, thick, dashed] (0,0) -- (2*\xmax, -\xmax * \bcms / \b); 
      \draw [color=lightgray, thick, dashed] (0,0) -- ( - 4 * \b * \ymin / \bcps, 2*\ymin); 
    }
  \begin{tikzpicture}[scale=.3]
    \begin{scope}[shift={(0,0)}]
      \setconstants;
      \draw[draw=none, fill=gray!20] (0,0) -- (2*\xmax,0) -- (2*\xmax,2*\ymax) --  (0,2*\ymax) -- cycle; 
      \boundingrays;
      \tikzmath{ \t=3; \r=1; \q=1; \xx = \r; \yy=\q; \x={\t*\xx/sqrt((\xx)^2+(\yy)^2)}; \y={\t*\yy/sqrt((\xx)^2+(\yy)^2)}; }
      \fill (\x,\y) circle (5pt);
    \end{scope}
    \begin{scope}[shift={(13,0)}]
      \setconstants;
      \draw[draw=none, fill=gray!20] (0,0) -- (2*\xmax,0) -- (4*\xmax,-2*\xmax / \b) --  (2*\xmax,-2*\xmax / \b) -- cycle; 
      \boundingrays;
      \tikzmath{ \t=5; \r=2; \q=1; \xx = \r + \b*\q; \yy=-\q; \x={\t*\xx/sqrt((\xx)^2+(\yy)^2)}; \y={\t*\yy/sqrt((\xx)^2+(\yy)^2)}; }
      \fill (\x, \y) circle (5pt);
      \draw [thick, join=round] (\x, \y) -- ({\x+\b * \y},\y);
      \draw [thick, fill=white] ({\x+\b * \y},\y) circle (5pt);
    \end{scope}
    \begin{scope}[shift={(26,0)}]
      \setconstants;
      \draw[draw=none, fill=gray!20] (0,0) -- (2*\xmin,0) -- (2*\xmin,2*\ymax) --  (0,2*\ymax) -- cycle; 
      \boundingrays;
      \tikzmath{ \t=3; \r=1; \q=1; \xx=-\r; \yy=\q; \x={\t*\xx/sqrt((\xx)^2+(\yy)^2)}; \y={\t*\yy/sqrt((\xx)^2+(\yy)^2)}; }
      \fill (\x, \y) circle (5pt);
      \draw [thick, join=round] (\x, \y) -- (\x,{\y - \c * \x});
      \draw [thick, fill=white] (\x,{\y - \c * \x}) circle (5pt);
    \end{scope}
    \begin{scope}[shift={(39, 0)}]
      \setconstants;
      \draw[draw=none, fill=gray!20] (0,0) -- (2*\xmin,0) -- (2*\xmin,2*\ymin) --  (0,2*\ymin) -- cycle; 
      \boundingrays;
      \tikzmath{ \t=1; \r=1; \q=1; \xx=-\r; \yy=-\q; \x={\t*\xx/sqrt((\xx)^2+(\yy)^2)}; \y={\t*\yy/sqrt((\xx)^2+(\yy)^2)}; }
      \fill (\x, \y) circle (5pt);
      \draw [thick, join=round] (\x, \y) -- ({\x + \b * \y}, \y) -- ({\x + \b * \y}, {\y -\c*(\x + \b * \y)}) -- (\x, {\y - \c*\x}) -- cycle;
      \draw [thick, fill=white] ({\x + \b * \y}, {\y -\c*(\x + \b * \y)})  circle (5pt);
    \end{scope}
    \begin{scope}[shift={(0,-13)}]
      \setconstants;
      \draw[draw=none, fill=gray!20] (0,0) -- (2*\xmax, -2*\xmax / \b) -- (4*\xmax, -2*\xmax / \b-2*\xmax * \c / \bcmo ) --  (2*\xmax, -2*\xmax * \c / \bcmo ) -- cycle; 
      \boundingrays;
      \tikzmath{ \t=5; \r=1; \q=2; \xx={\r*\b+\q*(\b*\c-1) }; \yy = {-\r -\c*\q};  \x={\t*\xx/sqrt((\xx)^2+(\yy)^2)}; \y={\t*\yy/sqrt((\xx)^2+(\yy)^2)}; }
      \fill (\x,\y) circle (5pt);
      \draw [thick, join=round] (\x,\y) -- ( \x+\b*\y , \y) -- ( \x+\b*\y , -\c*\x - \b*\c*\y + \y) -- ( \b*\c*\x + \x + \b*\b*\c*\y,  -\c*\x - \b*\c*\y + \y) --  cycle;
      %\draw [dashed, thick]  ( \x+\b*\y , -\c*\x - \b*\c*\y + \y) -- (0,0) -- (\x,\y);
      \draw [thick, fill=white] ( \x+\b*\y , -\c*\x - \b*\c*\y + \y) circle (5pt);
    \end{scope}
    \begin{scope}[shift={(13,-13)}]
      \setconstants;
      \draw[draw=none, fill=gray!20] (0,0) --  (2*\xmax, -2*\xmax * \c / \bcmo ) -- (4*\xmax, -2*\xmax * \c / \bcmo -\xmax * \bcms / \b) -- (2*\xmax, -\xmax * \bcms / \b) -- cycle; 
      \boundingrays;
      \tikzmath{ \t=5; \r=1; \q=2; \xx={2*\r*\b+\q*(\b*\c-1) }; \yy = {-\r*\bcms -\c*\q};  \x={\t*\xx/sqrt((\xx)^2+(\yy)^2)}; \y={\t*\yy/sqrt((\xx)^2+(\yy)^2)}; }
      \fill (\x,\y) circle (5pt);
      %\fill ( \x+\b*\y , -\c*\x - \b*\c*\y + \y) circle (2pt);
      \draw [thick, join=round] (\x , \y) -- ( \x+\b*\y , \y) -- ( \x+\b*\y , -\c*\x - \b*\c*\y + \y);
      \draw [thick, dash dot, join=round] ( \x+\b*\y , -\c*\x - \b*\c*\y + \y) -- (0,0) --  (\x , \y);
      \draw [thick, fill=white] ( \x+\b*\y , -\c*\x - \b*\c*\y + \y) circle (5pt);
    \end{scope}
    \begin{scope}[shift={(26,-13)}]
      \setconstants;
      \draw[draw=none, fill=gray!20] (0,0) -- (0, 2*\ymin) -- ( - 4 * \b * \ymin / \bcps, 2*\ymin +2*\ymin) --  ( - 4 * \b * \ymin / \bcps, 2*\ymin) -- cycle; 
      \boundingrays;
      \tikzmath{ \t=2; \r=0.3; \q=1; \xx={2*\r*\b }; \yy = {-\r*\bcps -\q};  \x={\t*\xx/sqrt((\xx)^2+(\yy)^2)}; \y={\t*\yy/sqrt((\xx)^2+(\yy)^2)}; }
      \fill (\x,\y) circle (5pt);
      %\fill ( \x+\b*\y , -\c*\x - \b*\c*\y + \y) circle (2pt);
      \draw [thick, join=round] (\x , \y) -- ( \x+\b*\y , \y) -- ( \x+\b*\y , -\c*\x - \b*\c*\y + \y);
      \draw [thick, dash dot, join=round] ( \x+\b*\y , -\c*\x - \b*\c*\y + \y) -- (0,0) --  (\x , \y);
      \draw [thick, fill=white] ( \x+\b*\y , -\c*\x - \b*\c*\y + \y) circle (5pt);
    \end{scope}
    \begin{scope}[shift={(39,-13)}]
      \setconstants;
      \draw[draw=none, fill=gray!20] (0,0) -- (2*\xmax, -\xmax * \bcms / \b) -- ( - 4 * \b * \ymin / \bcps + 2*\xmax, 2*\ymin  -\xmax * \bcms / \b) -- ( - 4 * \b * \ymin / \bcps, 2*\ymin) -- cycle; 
      \boundingrays;
      \tikzmath{ \t=5; \r=1; \q=2; \xx={2*(\r+\q)*\b }; \yy = {-\r*\bcps -\q*\bcms}; \x={\t*\xx/sqrt((\xx)^2+(\yy)^2)}; \y={\t*\yy/sqrt((\xx)^2+(\yy)^2)};  }
      \fill (\x , \y)  circle (5pt);
      %\fill ( \x+\b*\y , -\c*\x - \b*\c*\y + \y) circle (2pt);
      \draw [thick, join=round] (\x , \y) -- ( \x+\b*\y, \y) -- ( \x+\b*\y , -\c*\x - \b*\c*\y + \y);;
      \draw [thick, dotted, join=round] ( \x+\b*\y , -\c*\x - \b*\c*\y + \y) --  (\bcps*\x/\s + \bcps*\b*\y/2/\s, -\bcps*\c*\x/2/\s - \bcps*\b*\c*\y/2/\s + \bcps*\y/\s) --  (\x , \y);
      \draw [thick, dashed, join=round] ( \x+\b*\y , -\c*\x - \b*\c*\y + \y) -- (0,0) --  (\x , \y);
      \draw [thick, fill=white]  ( \x+\b*\y , -\c*\x - \b*\c*\y + \y) circle (5pt);
    \end{scope}
  \end{tikzpicture}
\end{figure}

\begin{remark}
  This result reveals the maximal support possible for a pointed basis which remains pointed under all possible mutations.
  In particular, any candidate basis could be ruled out using these results based upon their support alone.
\end{remark}

The paper is organized as follows.
In Section~\ref{sec:chebyshev}, we collect useful results related to two-parameter Chebyshev polynomials which support our main calculations.
Section~\ref{sec:tropical} contains calculations related to the transformation of $\bfg$-vectors under mutations.
Section~\ref{sec:dominance inequalities} proves Theorem~\ref{th:dominance inequalities}.
Section~\ref{sec:dominance vertices} proves Corollary~\ref{cor:dominance vertices}.
Section~\ref{sec:maximum support} proves Theorem~\ref{th:maximum support}.
The paper ends with Section~\ref{sec:affine} interpreting the dominance polygon in terms of generalized minors in the cases where $b=c=2$.


\section{Chebyshev Polynomials}
  \label{sec:chebyshev}
  Define \emph{two-parameter Chebyshev polynomials} $u_i^\varepsilon$ for $i\in\ZZ$ and $\varepsilon\in\{\pm\}$ recursively by
  \[u_0^\varepsilon=0,\quad u_1^\varepsilon=1,\quad u_{i+1}^\varepsilon=\begin{cases} bu_i^- -u_{i-1}^+ & \text{if $\varepsilon=+$;}\\ cu_i^+-u_{i-1}^- & \text{if $\varepsilon=-$.} \end{cases}\]
    Observe that, by an easy induction, $u_{-i}^\varepsilon=-u_i^\varepsilon$.

  The recursion can be written more succinctly as $u_{i+1}^\varepsilon=u_2^\varepsilon u_i^{-\varepsilon}-u_{i-1}^\varepsilon$ with the additional initial condition $u_2^\varepsilon=\begin{cases} b & \text{if $\varepsilon=+$;}\\ c & \text{if $\varepsilon=-$.} \end{cases}$
  This leads to the following more general recursive structure.
  \begin{lemma}
    For $i,\ell\in\ZZ$ and $\varepsilon\in\{\pm\}$, we have
    \begin{equation}
      \label{eq:long Chebyshev recursion}
      u_{i+\ell}^\varepsilon=\begin{cases} u_{\ell+1}^\varepsilon u_i^{-\varepsilon}-u_\ell^{-\varepsilon} u_{i-1}^\varepsilon & \text{\upshape if $\ell$ is odd;}\\ u_{\ell+1}^{-\varepsilon} u_i^\varepsilon-u_\ell^\varepsilon u_{i-1}^{-\varepsilon} & \text{\upshape if $\ell$ is even.} \end{cases}
    \end{equation}
  \end{lemma}
  \begin{proof}
    We work by induction on $\ell$, the cases $\ell=0,1$ being tautological and reproducing the defining recursion, respectively.
    Using the claim for $\ell>0$ and then the defining recursion twice, we get
    \begin{align*}
      u_{i+\ell+1}^\varepsilon
      &=
      u_{\ell+1}^\varepsilon u_{i+1}^{-\varepsilon}-u_\ell^{-\varepsilon} u_i^\varepsilon
      =(u_2^{-\varepsilon} u_{\ell+1}^\varepsilon-u_\ell^{-\varepsilon}) u_i^\varepsilon-u_{\ell+1}^\varepsilon u_{i-1}^{-\varepsilon}
      =u_{\ell+2}^{-\varepsilon} u_i^\varepsilon - u_{\ell+1}^\varepsilon u_{i-1}^{-\varepsilon}
      \qquad \text{if $\ell$ is odd;}\\ 
      u_{i+\ell+1}^\varepsilon
      &=
      u_{\ell+1}^{-\varepsilon} u_{i+1}^\varepsilon-u_\ell^\varepsilon u_i^{-\varepsilon}
      =(u_2^\varepsilon u_{\ell+1}^{-\varepsilon}-u_\ell^\varepsilon) u_i^{-\varepsilon}-u_{\ell+1}^{-\varepsilon} u_{i-1}^\varepsilon
      =u_{\ell+2}^\varepsilon u_i^{-\varepsilon}-u_{\ell+1}^{-\varepsilon} u_{i-1}^\varepsilon
      \qquad \text{if $\ell$ is even;} 
    \end{align*}
    which gives the claimed recursion for $\ell+1$.
    These calculations can be reversed to show the result for $\ell<0$.
  \end{proof}

  The standard Chebyshev polynomials (normalized, of the second kind) are defined by the recursion $u_0=0$, $u_1=1$, $u_{i+1}=ru_i-u_{i-1}$, which can be computed explicitly as
  \[u_i(r)=\frac{1}{2^i\sqrt{r^2-4}}\left(\big(r+\sqrt{r^2-4}\big)^i-\big(r-\sqrt{r^2-4}\big)^i\right).\]
  By the equalities 
  \[u_i^\varepsilon=\begin{cases} \frac{\sqrt{b}}{\sqrt{c}}u_i(\sqrt{bc}) & \text{if $i$ is even and $\varepsilon=+$;}\\ \frac{\sqrt{c}}{\sqrt{b}}u_i(\sqrt{bc}) & \text{if $i$ is even and $\varepsilon=-$;}\\ u_i(\sqrt{bc}) & \text{if $i$ is odd;} \end{cases}\]
  it follows that $u_i^\varepsilon$ can be computed explicitly as
  \[u_i^\varepsilon=\begin{cases} \frac{\sqrt{b}}{2^i\sqrt{c(bc-4)}}\left(\big(\sqrt{bc}+\sqrt{bc-4}\big)^i-\big(\sqrt{bc}-\sqrt{bc-4}\big)^i\right) & \text{if $i$ is even and $\varepsilon=+$;}\\ \frac{\sqrt{c}}{2^i\sqrt{b(bc-4)}}\left(\big(\sqrt{bc}+\sqrt{bc-4}\big)^i-\big(\sqrt{bc}-\sqrt{bc-4}\big)^i\right) & \text{if $i$ is even and $\varepsilon=-$;}\\ \frac{1}{2^i\sqrt{bc-4}}\left(\big(\sqrt{bc}+\sqrt{bc-4}\big)^i-\big(\sqrt{bc}-\sqrt{bc-4}\big)^i\right) & \text{if $i$ is odd.} \end{cases}\]
  \begin{remark}
    Note that $u_{2j+1}^+=u_{2j+1}^-$ for all $j\in\ZZ$.
    Moreover, since $u_{-i}^\varepsilon=-u_i^\varepsilon$, these formulas can also easily be seen to hold for $i<0$.
  \end{remark}

  \begin{lemma}
    \label{le:limits}
    We have
    \[\lim_{i\to\infty} \frac{u_i^-}{u_{i-1}^+}=\frac{bc+\sqrt{bc(bc-4)}}{2b}=\lim_{i\to-\infty} \frac{u_{i-1}^-}{u_i^+};\]
    \[\lim_{i\to\infty} \frac{u_{i-1}^-}{u_i^+}=\frac{bc-\sqrt{bc(bc-4)}}{2b}=\lim_{i\to-\infty} \frac{u_i^-}{u_{i-1}^+}.\]
    Moreover, each of these sequences is monotonic.
  \end{lemma}
  \begin{remark}
    The analogous limits with $+$ and $-$ reversed are obtained from these by interchanging the roles of $b$ and $c$.
  \end{remark}
  \begin{proof}
    For any $i\ne1$, we have
    \[\frac{\big(\sqrt{bc}+\sqrt{bc-4}\big)^i-\big(\sqrt{bc}-\sqrt{bc-4}\big)^i}{\big(\sqrt{bc}+\sqrt{bc-4}\big)^{i-1}-\big(\sqrt{bc}-\sqrt{bc-4}\big)^{i-1}}=\frac{\big(\sqrt{bc}+\sqrt{bc-4}\big)\cdot\left(1-\left(\frac{\sqrt{bc}-\sqrt{bc-4}}{\sqrt{bc}+\sqrt{bc-4}}\right)^i\right)}{1-\left(\frac{\sqrt{bc}-\sqrt{bc-4}}{\sqrt{bc}+\sqrt{bc-4}}\right)^{i-1}}.\]
    It follows that
    \[\lim_{i\to\infty} \frac{u_i^-}{u_{i-1}^+} = \lim_{i\to\infty} \left( \frac{\sqrt{c}}{2\sqrt{b}}\cdot\frac{\big(\sqrt{bc}+\sqrt{bc-4}\big)\cdot\left(1-\left(\frac{\sqrt{bc}-\sqrt{bc-4}}{\sqrt{bc}+\sqrt{bc-4}}\right)^i\right)}{1-\left(\frac{\sqrt{bc}-\sqrt{bc-4}}{\sqrt{bc}+\sqrt{bc-4}}\right)^{i-1}} \right) = \frac{\sqrt{c}}{2\sqrt{b}}\cdot\big(\sqrt{bc}+\sqrt{bc-4}\big),\]
    which is equivalent to the desired expression.
    Similarly, for $i\ne0$ we have
    \[\frac{\big(\sqrt{bc}+\sqrt{bc-4}\big)^{i-1}-\big(\sqrt{bc}-\sqrt{bc-4}\big)^{i-1}}{\big(\sqrt{bc}+\sqrt{bc-4}\big)^i-\big(\sqrt{bc}-\sqrt{bc-4}\big)^i}=\frac{\big(\sqrt{bc}-\sqrt{bc-4}\big)\cdot\left(1-\left(\frac{\sqrt{bc}-\sqrt{bc-4}}{\sqrt{bc}+\sqrt{bc-4}}\right)^{i-1}\right)}{4\left(1-\left(\frac{\sqrt{bc}-\sqrt{bc-4}}{\sqrt{bc}+\sqrt{bc-4}}\right)^i\right)},\]
    so that
    \[\lim_{i\to\infty} \frac{u_{i-1}^-}{u_i^+} = \lim_{i\to\infty} \left( \frac{2\sqrt{c}}{\sqrt{b}}\cdot\frac{\big(\sqrt{bc}-\sqrt{bc-4}\big)\cdot\left(1-\left(\frac{\sqrt{bc}-\sqrt{bc-4}}{\sqrt{bc}+\sqrt{bc-4}}\right)^{i-1}\right)}{4\left(1-\left(\frac{\sqrt{bc}-\sqrt{bc-4}}{\sqrt{bc}+\sqrt{bc-4}}\right)^i\right)} \right) = \frac{\sqrt{c}}{2\sqrt{b}}\cdot\big(\sqrt{bc}-\sqrt{bc-4}\big),\]
    which is again equivalent to the desired expression.
    This proves the claim for $i\to\infty$, the cases $i\to-\infty$ follow from these using $u_{-i}^\varepsilon=-u_i^\varepsilon$.

    For the final claim, observe that $\frac{u_{i+1}^{-\varepsilon}}{u_i^\varepsilon}<\frac{u_i^{-\varepsilon}}{u_{i-1}^\varepsilon}$ and $\frac{u_{i-1}^{-\varepsilon}}{u_i^\varepsilon}<\frac{u_i^{-\varepsilon}}{u_{i+1}^\varepsilon}$ are both equivalent to $u_{i+1}^{-\varepsilon}u_{i-1}^\varepsilon<u_i^{-\varepsilon}u_i^\varepsilon$.
    This is immediate from the following induction:
    \begin{align*}
      u_{i+1}^{-\varepsilon}u_{i-1}^\varepsilon
      =(cu_i^\varepsilon-u_{i-1}^{-\varepsilon})u_{i-1}^\varepsilon
      =cu_i^\varepsilon u_{i-1}^\varepsilon-u_{i-1}^{-\varepsilon} u_{i-1}^\varepsilon
      <cu_i^\varepsilon u_{i-1}^\varepsilon-u_i^\varepsilon u_{i-2}^{-\varepsilon}
      =u_i^\varepsilon(cu_{i-1}^\varepsilon-u_{i-2}^{-\varepsilon})
      =u_i^\varepsilon u_i^{-\varepsilon}.
    \end{align*}
  \end{proof}


\section{$\bfg$-vector Mutations}
\label{sec:tropical}

  We begin by studying transformations of $\RR^2$ which determine the change of $\bfg$-vectors when expanding a pointed expression of the form~\eqref{eq:pointed} in terms of a cluster $\{x_k,x_{k+1}\}$.
  Write $\phi_0:\RR^2\to\RR^2$ for the identity map and define piecewise-linear maps $\phi_{\pm 1}:\RR^2\to\RR^2$ as follows:
  \begin{equation}
    \label{eq:forward mutation 1}
    \phi_1(\lambda)
    :=
    \begin{cases} 
      (-\lambda_0,c\lambda_0+\lambda_1) & \text{if $\lambda_0 \ge 0$;}\\
      (-\lambda_0,\lambda_1) & \text{if $\lambda_0 < 0$;}
    \end{cases}
    \qquad
    \phi_{-1}(\lambda)
    :=
    \begin{cases} 
      (\lambda_0,-\lambda_1) & \text{if $\lambda_1 \ge 0$;}\\
      (\lambda_0+b\lambda_1,-\lambda_1) & \text{if $\lambda_1 < 0$.}
    \end{cases}
  \end{equation}
  For $k\in\ZZ$ with $|k|>1$, define piecewise-linear maps
  \[\phi_k
    :=
    \begin{cases}
      (\phi_{-1}^{-1}\phi_1)^j & \text{if $k=2j$;}\\
      \phi_1(\phi_{-1}^{-1}\phi_1)^j & \text{if $k=2j+1$.}\\
    \end{cases}
  \]
  These determine the dominance region as follows.
  \begin{definition}
    \label{def:dominance}
    For $\lambda\in\ZZ^2$ and $k\in\ZZ$, define cones 
    \[\cC_k(\lambda)
      :=
      \begin{cases}
        \{(\lambda_0-r,\lambda_1+s):r,s\in\RR_{\ge0}\} & \text{if $k=2j$;}\\
        \{(\lambda_0+r,\lambda_1-s):r,s\in\RR_{\ge0}\} & \text{if $k=2j+1$.}
      \end{cases}
    \]
    The \emph{dominance region} $\cP_\lambda$ is the intersection  $\bigcap_{k\in\ZZ}\phi_k^{-1}\cC_k(\phi_k\lambda)$.
    When $\mu\in\cP_\lambda$, we say \emph{$\lambda$ dominates $\mu$}.
  \end{definition}

  We record here a few useful calculations relating to the tropical transformations $\phi_k$.
  First observe the following explicit expression for $\phi_2$:
  \begin{equation}
    \label{eq:forward two step mutation}
    \phi_2(\lambda)
    =
    \begin{cases}
      \big((bc-1)\lambda_0+b\lambda_1, -c\lambda_0-\lambda_1\big) & \text{if $\lambda_0\ge 0$ and $c\lambda_0+\lambda_1\ge 0$;}\\
      (-\lambda_0, -c\lambda_0-\lambda_1) & \text{if $\lambda_0\ge 0$ and $c\lambda_0+\lambda_1<0$;}\\
      (-\lambda_0+b\lambda_1, -\lambda_1) & \text{if $\lambda_0<0$ and $\lambda_1\ge 0$;}\\
      (-\lambda_0,-\lambda_1) & \text{if $\lambda_0<0$ and $\lambda_1<0$;}
    \end{cases}
  \end{equation}

  By an eigenvector of a piecewise-linear map $\phi$, we will mean a vector $\lambda$ such that for each $n>0$, there exists a scalar $\nu_n$ so that $\phi^n(\lambda)=\nu_n\lambda$.
  \begin{lemma}
    Any nonzero eigenvector of $\phi_2$ is a positive multiple of one of the vectors $\big(2b,-bc\pm\sqrt{bc(bc-4)}\big)$.
  \end{lemma}
  \begin{proof}
    First observe that the equation $\phi_2^n(\lambda)=\nu_n\lambda$ cannot be satisfied for all $n>0$ unless $\lambda_0\ge 0$ and $c\lambda_0+\lambda_1\ge 0$.
    In this region, $\phi_2$ is linear with eigenvalue $\nu$ satisfying $\nu^2-(bc-2)\nu+1=0$, i.e. $\nu=\frac{bc-2\pm\sqrt{bc(bc-4)}}{2}$.
    We thus require 
    \[\frac{bc-2\pm\sqrt{bc(bc-4)}}{2}\lambda_0=(bc-1)\lambda_0+b\lambda_1 \qquad\text{and}\qquad \frac{bc-2\pm\sqrt{bc(bc-4)}}{2}\lambda_1= -c\lambda_0-\lambda_1,\]
    or equivalently
    \[\frac{-bc\pm\sqrt{bc(bc-4)}}{2b}\lambda_0=\lambda_1 \qquad\text{and}\qquad \frac{-bc\mp\sqrt{bc(bc-4)}}{2c}\lambda_1=\lambda_0.\]
    As these represent the same relationship, the result immediately follows by inspection.
  \end{proof}

  %\begin{lemma}
  %  The vectors $\big(2b, -bc+\sqrt{bc(bc-4)}\big)$ (resp. $\big(2b, -bc-\sqrt{bc(bc-4)}\big)$) and $\big(2c, bc+\sqrt{bc(bc-4)}\big)$ (resp. $\big(2c, bc-\sqrt{bc(bc-4)}\big)$) are perpendicular.
  %\end{lemma}
  %\begin{proof}
  %  This is immediate from the equality $\big(-bc+\sqrt{bc(bc-4)}\big)\big(bc+\sqrt{bc(bc-4)}\big)=-4bc$.
  %\end{proof}
  \begin{lemma}
    \label{le:imaginary stability}
    For $j\in\ZZ$ and $\lambda\in\cI$, we have $\phi_{2j}(\lambda)\in\cI$.
  \end{lemma}
  \begin{proof}
    Since $\phi_{2j}=\phi_2^j$, the result follows from the cases $j=1$.
    But this is immediate for $\phi_2$ since the vectors $\big(2b,-bc\pm\sqrt{bc(bc-4)}\big)$ are eigenvectors and $\phi_2$ is linear in $\cI$.
  \end{proof}

  It will be useful to have explicit expressions for $\phi_k(\lambda)$ for $\lambda\in\cI$.
  \begin{lemma}
    \label{le:imaginary transformations}
    For $j\in\ZZ$ and $\lambda\in\cI$, we have
    \begin{align*}
      \phi_{2j}(\lambda)&=(u_{2j+1}^+\lambda_0+u_{2j}^+\lambda_1,-u_{2j}^-\lambda_0-u_{2j-1}^-\lambda_1);\\
      \phi_{2j+1}(\lambda)&=(-u_{2j+1}^+\lambda_0-u_{2j}^+\lambda_1,u_{2j+2}^-\lambda_0+u_{2j+1}^-\lambda_1).
    \end{align*}
  \end{lemma}
  \begin{proof}
    We work by induction on $j$, the case $j=0$ being clear from the definitions.
    For $\lambda\in\cI$, the action of $\phi_2$ from \eqref{eq:forward two step mutation} can be rewritten as
    \[\phi_2(\lambda)=(u_3^-\lambda_0+u_2^+\lambda_1,-u_2^-\lambda_0-u_1^+\lambda_1).\]
    Therefore
    \begin{align*}
      \phi_{2j+2}(\lambda)&=\phi_2\phi_{2j}(\lambda)\\
      &=\big( (u_3^- u_{2j+1}^+-u_2^+u_{2j}^-)\lambda_0+(u_3^-u_{2j}^+-u_2^+u_{2j-1}^-)\lambda_1,-(u_2^-u_{2j+1}^+-u_1^+u_{2j}^-)\lambda_0-(u_2^-u_{2j}^+-u_1^+u_{2j-1}^-)\lambda_1 \big)\\
      &=(u_{2j+3}^+\lambda_0+u_{2j+2}^+\lambda_1, -u_{2j+2}^-\lambda_0-u_{2j+1}^-\lambda_1),
    \end{align*}
    as desired, where the last equality uses \eqref{eq:long Chebyshev recursion} with $i=2j+1$ and $\ell=2$.

    Similarly, using $\phi_1(\lambda)=(-\lambda_0,u_2^-\lambda_0+u_1^+\lambda_1)$ for $\lambda\in\cI$ together with the basic Chebyshev recursion and the equality $\phi_{2j+1}=\phi_1\phi_2^j$ gives the claimed formula for $\phi_{2j+1}$ from that of $\phi_{2j}$.
  \end{proof}


\section{Proof of Theorem~\ref{th:dominance inequalities}}
\label{sec:dominance inequalities}

  Here we explicitly compute the dominance regions $\cP_\lambda$.
  When $\lambda$ is not imaginary, the dominance region is particularly simple.
  \begin{lemma}
    \label{le:cluster monomials}
    If $\lambda\in\ZZ^2\setminus\cI$, then $\cP_\lambda=\{\lambda\}$.
  \end{lemma}
  \begin{proof}
    Any such $\lambda$ is the $\bfg$-vector of a cluster monomial, say $x_k^{\alpha_k}x_{k+1}^{\alpha_{k+1}}$.
    In this case, the intersection
    \[\phi_{k-1}^{-1}\cC_{k-1}(\phi_{k-1}\lambda) \cap \phi_{k+1}^{-1}\cC_{k+1}(\phi_{k+1}\lambda)\]
    is precisely $\{\lambda\}$. 
    Indeed, for $k$ odd, the cone $\cC_{k-1}(\phi_{k-1}\lambda)$ lies entirely within a domain of linearity for $\phi_{k-1}^{-1}$.
    In particular, $\phi_{k-1}^{-1}\cC_{k-1}(\phi_{k-1}\lambda)$ is a cone containing $\lambda$ directed away from the origin (with walls parallel to the boundary of the domain of linearity containing $\lambda$).
    Next, again for $k$ odd, the cone $\cC_{k+1}(\phi_{k+1}\lambda)$ intersects the domain of linearity for $\phi_{k+1}^{-1}$ containing $\phi_{k+1}\lambda$ in a (possibly degenerate) convex quadrilateral with corners at $(0,0)$ and $\phi_{k+1}\lambda$.
    In particular, the intersection of $\phi_{k+1}^{-1}\cC_{k+1}(\phi_{k+1}\lambda)$ with the cone containing $\lambda$ is a (possibly degenerate) convex quadrilateral with corners at $(0,0)$ and $\lambda$.
    Combining these observations proves the result for $k$ odd, the case of even $k$ is similar.
  \end{proof}
  To compare with Theorem~\ref{th:dominance inequalities}, we consider three cases in which $\lambda\in\ZZ^2\setminus\cI$:
  \begin{itemize}
    \item if $\lambda_0 \le 0$ and $\lambda_1 \ge 0$, the second and fourth pairs of inequalities become equations for lines which intersect at $\lambda$;
    \item if $\lambda_0 > 0$ and $\lambda_1 \ge \frac{-bc+\sqrt{bc(bc-4)}}{2b}\lambda_0$, the first and second pairs of inequalities become equations for lines which intersect at $\lambda$;
    \item if $\lambda_1 < 0$ and $\lambda_0 \le \frac{bc-\sqrt{bc(bc-4)}}{2c}\lambda_1$, the third and fourth pairs of inequalities become equations for lines which intersect at $\lambda$.
  \end{itemize}
  This proves Theorem~\ref{th:dominance inequalities} when $\lambda\notin\cI$.

  Next we aim to understand how inequalities transform under the action of a linear map, this will be the key tool for describing the dominance regions $\cP_\lambda$ for $\lambda\in\cI$.

  \begin{lemma}
    \label{le:transformed inequalities}
    Let $A$ be an invertible $2\times 2$ matrix.
    Under the left action of $A$, say $A\mu=\mu'$ for $\mu,\mu'\in\RR^2$, the region inside $\RR^2$ defined by the inequality $\bfr\bullet\mu \le t$ with $\bfr\in\RR^2$ and $t\in\RR$ is transformed into the region defined by the inequality \[(A^{-T}\bfr)\bullet\mu'\le t.\]
  \end{lemma}
  \begin{proof}
    This is immediate from the equalities
    \[\bfr\bullet\mu=\bfr^T\mu=\bfr^TA^{-1}A\mu=(A^{-T}\bfr)^TA\mu=(A^{-T}\bfr)\bullet\mu'.\]
  \end{proof}
  
  \begin{lemma}
    \label{le:dominance inequalities}
    For $k\in\ZZ$, $k\ne0$, and $\lambda\in\cI$, the region $\phi_k^{-1}\cC_k(\phi_k\lambda)\subset\RR^2$ is determined by the following inequalities when $k>0$:
    \begin{align}
      \label{ineq:1} u_k^-\mu_0+u_{k-1}^+\mu_1 &\le u_k^-\lambda_0+u_{k-1}^+\lambda_1;\\
      \label{ineq:2} u_{k+1}^-\mu_0+u_k^+\mu_1 &\le u_{k+1}^-\lambda_0+u_k^+\lambda_1;\\
      \label{ineq:3} -u_{k-1}^-\mu_0+u_k^+\mu_1 &\le u_{k+1}^-\lambda_0+u_k^+\lambda_1;\\
      \label{ineq:4} -u_{k-1}^-\mu_0-u_{k-2}^+\mu_1 &\le u_{k+1}^-\lambda_0+u_k^+\lambda_1;
    \end{align}
    and by the following inequalities when $k<0$: \saySS{the third equation here is wrong!} \sayDR{I am not seeing what is wrong with it, doesn't it match the points you plotted?}
    \saySS{It looks to me the current equation is the red one in the picture (wrong) and the correct one is the blue one. They differ in signs.}
    \begin{align*}
      u_{k+1}^-\mu_0+u_k^+\mu_1 &\le u_{k+1}^-\lambda_0+u_k^+\lambda_1;\\
      u_k^-\mu_0+u_{k-1}^+\mu_1 &\le u_k^-\lambda_0+u_{k-1}^+\lambda_1;\\
      u_k^-\mu_0-u_{k+1}^+\mu_1 &\le u_k^-\lambda_0+u_{k-1}^+\lambda_1;\\
      -u_{k+2}^-\mu_0-u_{k+1}^+\mu_1 &\le u_k^-\lambda_0+u_{k-1}^+\lambda_1.
    \end{align*}
    
    %given as in the figure below:
\begin{figure}[h!]
  \centering
    \newcommand{\setconstants}{
      \tikzmath{\b = 2.5; \c =4.5 / \b; \s = sqrt(\b * \c * (\b * \c -4)); \bcmo = \b * \c - 1; \bcms = \b * \c - \s; \bcps = \b * \c + \s;}
      \tikzmath{ \xmin = -6; \xmax = 6; \ymin = -6; \ymax = 6;}
      \clip (\xmin,\ymin) rectangle (\xmax,\ymax);
    }
    \newcommand{\boundingrays}{
      \draw [color=lightgray, thick] (2*\xmin,0) -- (2*\xmax,0);
      \draw [color=lightgray, thick] (0,2*\ymin) -- (0, 2*\ymax);
      %\draw [color=lightgray, thick] (0,0) -- (2*\xmax, -2*\xmax / \b);
      %\draw [color=lightgray, thick] (0,0) -- (2*\xmax, -2*\xmax * \c / \bcmo );
      \draw [color=lightgray, thick, dashed] (0,0) -- (2*\xmax, -\xmax * \bcms / \b); 
      \draw [color=lightgray, thick, dashed] (0,0) -- ( - 4 * \b * \ymin / \bcps, 2*\ymin); 
    }
  \begin{tikzpicture}[scale=.3]
    %negatives k=-3
    \begin{scope}[shift={(0,0)}]
      \setconstants;
      \tikzmath{ \t=3; \r=1; \q=1.6; \xx={2*(\r+\q)*\b }; \yy = {-\r*\bcps -\q*\bcms}; \x={\t*\xx/sqrt((\xx)^2+(\yy)^2)}; \y={\t*\yy/sqrt((\xx)^2+(\yy)^2)};  }
      \fill[fill=gray!20] (\xmax, {\c*(\x-\xmax)/(\b*\c-1)+\y}) --  (\x,\y) -- ({\x+\b*(\b*\c-2)*\y/(\b*\c-1)},0) -- (0,{-(\b*\c-1)*\x/\b-(\b*\c-2)*\y}) -- (\xmax, {-(\b*\c-1)*\x/\b-(\b*\c-2)*\y-\xmax/\b});
      \boundingrays;
      \fill (\x , \y)  circle (5pt) node[below]{$\lambda$};
      \draw[thick] (\xmax, {\c*(\x-\xmax)/(\b*\c-1)+\y}) --  (\x,\y) -- ({\x+\b*(\b*\c-2)*\y/(\b*\c-1)},0) -- (0,{-(\b*\c-1)*\x/\b-(\b*\c-2)*\y}) -- (\xmax, {-(\b*\c-1)*\x/\b-(\b*\c-2)*\y-\xmax/\b});
      \draw[domain=\xmin:\xmax, smooth, variable=\p, red] plot ({\p}, {(\b*\c-1)*(\x-\p)/\b+(\b*\c-2)*\y});
      \draw[domain=\xmin:\xmax, smooth, variable=\p, blue] plot ({\p}, {(\b*\c-1)*(-\x+\p)/\b-(\b*\c-2)*\y});
			%\draw [align=center, below] (0,\ymin) node{$\phi_k^{-1}\cC_k(\phi_k\lambda)$};
    \end{scope}
    \draw (0,-7) node{$\phi_{-3}^{-1}\cC_{-3}(\phi_{-3}\lambda)$};
    %positives k=3
    \begin{scope}[shift={(15,0)}]
      \setconstants;
      \tikzmath{ \t=3; \r=1; \q=1.6; \xx={2*(\r+\q)*\b }; \yy = {-\r*\bcps -\q*\bcms}; \x={\t*\xx/sqrt((\xx)^2+(\yy)^2)}; \y={\t*\yy/sqrt((\xx)^2+(\yy)^2)};  }
      \fill[fill=gray!20] ({\x+\b*(\y-\ymin)/(\b*\c-1)}, \ymin) -- (\x,\y) -- (0, {\c*(\b*\c-2)*\x/(\b*\c-1)+\y}) -- ({-(\b*\c-2)*\x-(\b*\c-1)*\y/\c},0) -- ({-\ymin/\c-(\b*\c-2)*\x-(\b*\c-1)*\y/\c},\ymin);
      \boundingrays;
      \draw[domain=\xmin:\xmax, smooth, variable=\p, black!60] plot ({\p}, {(\b*\c-1)*(\x-\p)/\b+\y});
      \draw[domain=\xmin:\xmax, smooth, variable=\p, black!60] plot ({\p}, {\c*(\b*\c-2)/(\b*\c-1)*(\x-\p)+\y});
      \draw[domain=\xmin:\xmax, smooth, variable=\p, black!60] plot ({\p}, {\c*(\b*\c-2)/(\b*\c-1)*\x+\c/(\b*\c-1)*\p +\y});
      \draw[domain=\xmin:\xmax, smooth, variable=\p, black!60] plot ({\p}, {-\c*(\b*\c-2)*\x-\c*\p -(\b*\c-1)*\y});
      \fill (\x , \y)  circle (5pt) node[right]{$\lambda$};
      \draw[thick] ({\x+\b*(\y-\ymin)/(\b*\c-1)}, \ymin) -- (\x,\y) -- (0, {\c*(\b*\c-2)*\x/(\b*\c-1)+\y}) -- ({-(\b*\c-2)*\x-(\b*\c-1)*\y/\c},0) -- ({-\ymin/\c-(\b*\c-2)*\x-(\b*\c-1)*\y/\c},\ymin);
    \end{scope}
    \draw (15,-7) node{$\phi_{3}^{-1}\cC_{3}(\phi_{3}\lambda)$};
  \end{tikzpicture}
\end{figure}

%    \[
%      \begin{tikzpicture}
%        \filldraw[color=lightgray!30] (47/10,-5) --(3.5,-3) -- (0,26/10) -- (-26/4,0) -- (-16/4,-5);
%        \draw[-] (-8,0) -- (5,0);
%        \draw[-] (0,-5) -- (0,5);
%        %\filldraw[color=lightgray!30] (0,0) -- (5,-2.113) -- (5,-5) -- (3.17,-5) -- (0,0);
%        \draw[dashed] (0,0) -- (5,-2.113);
%        \draw[dashed] (0,0) -- (3.17,-5);
%        \node at (1,-0.9) {$\cI$};
%        \draw[fill=black] (3.5,-3) circle (2pt);
%        \node[right] at (3.5,-3) {$\lambda$};
%        \draw[->,thick] (3.5,-3) -- (47/10,-5);
%        \draw[-,thick] (3.5,-3) -- (0,26/10);
%        \draw[-,thick] (0,26/10) -- (-26/4,0);
%        \draw[->,thick] (-26/4,0) -- (-16/4,-5);
%        \node at (4.5,-4) {\eqref{ineq:1}};
%        \node at (2.34,-0.5) {\eqref{ineq:2}};
%        \node at (-3,1.75) {\eqref{ineq:3}};
%        \node at (-5.36,-3) {\eqref{ineq:4}};
%      \end{tikzpicture}
%    \]
  \end{lemma}
  \begin{proof}
    We prove the claim for $k>0$, the proof for $k<0$ is similar or can be deduced from the other case by a symmetry argument.
    Following Lemma~\ref{le:imaginary transformations}, we consider even and odd sequences of mutations separately.

    For $k=2j$, $j>0$, and $\lambda\in\cI$, we observe that $\cC_k(\phi_k\lambda)\subset\RR^2$ is given by the inequalities 
    \[ (\dagger)\ \mu_0 \le u_{2j+1}^+\lambda_0+u_{2j}^+\lambda_1 = u_{2j+1}^-\lambda_0+u_{2j}^+\lambda_1 \qquad\text{and}\qquad (\ddagger)\ -\mu_1\le u_{2j}^-\lambda_0+u_{2j-1}^-\lambda_1 = u_{2j}^-\lambda_0+u_{2j-1}^+\lambda_1. \]
    We inductively compute the region $\phi_{2j}^{-1}\cC_{2j}(\phi_{2j}\lambda)$, using Lemma~\ref{le:transformed inequalities} and the equality $\phi_{2j}^{-1}=\phi_2^{-j}$.
    First observe that $\phi_2^{-1}=\phi_{-2}$ is given as follows:
    \begin{equation}
      \label{eq:backward two step mutation}
      \phi_{-2}(\lambda)
      =
      \begin{cases}
        \big(-\lambda_0-b\lambda_1, c\lambda_0+(bc-1)\lambda_1\big) & \text{if $\lambda_1\le 0$ and $\lambda_0+b\lambda_1\le 0$;}\\
        (-\lambda_0-b\lambda_1, -\lambda_1) & \text{if $\lambda_1\le 0$ and $\lambda_0+b\lambda_1>0$;}\\
        (-\lambda_0, c\lambda_0-\lambda_1) & \text{if $\lambda_1>0$ and $\lambda_0\le 0$;}\\
        (-\lambda_0,-\lambda_1) & \text{if $\lambda_1>0$ and $\lambda_0>0$.}
      \end{cases}
    \end{equation}

    \subsubsection*{Claim:} For $i>0$, $\phi_{2i}^{-1}\cC_{2j}(\phi_{2j}\lambda)$ is the region determined by the inequalities 
    \begin{align*}
      \tag{a} u_{2i}^-\mu_0+u_{2i-1}^+\mu_1 &\le u_{2j}^-\lambda_0+u_{2j-1}^+\lambda_1;\\
      \tag{b} u_{2i+1}^-\mu_0+u_{2i}^+\mu_1 &\le u_{2j+1}^-\lambda_0+u_{2j}^+\lambda_1;\\
      \tag{c} -u_{2i-1}^-\mu_0+u_{2i}^+\mu_1 &\le u_{2j+1}^-\lambda_0+u_{2j}^+\lambda_1;\\
      \tag{d} -u_{2i-1}^-\mu_0-u_{2i-2}^+\mu_1 &\le u_{2j+1}^-\lambda_0+u_{2j}^+\lambda_1.
    \end{align*}

    We see from \eqref{eq:backward two step mutation}, that $\lambda\in\cI$ implies the boundary ray for $\cC_{2j}(\phi_{2j}\lambda)$ corresponding to ($\ddagger$) lies entirely in the region in which $\phi_2^{-1}$ acts according to the matrix $\left[ \begin{array}{cc} -1 & -b\\ c & bc-1 \end{array}\right]$ of determinant 1.
    By Lemma~\ref{le:transformed inequalities}, the inequality ($\ddagger$) transforms into the inequality $c\mu_0+\mu_1\le u_{2j}^-\lambda_0+u_{2j-1}^+\lambda_1$ which corresponds to (a) with $i=1$.
    Similarly, the boundary ray for $\cC_{2j}(\phi_{2j}\lambda)$ corresponding to ($\dagger$) intersects the three domains of linearity in which $\phi_2^{-1}$ acts according to the matrices $\left[ \begin{array}{cc} -1 & -b\\ c & bc-1 \end{array}\right]$, $\left[ \begin{array}{cc} -1 & -b\\ 0 & -1 \end{array}\right]$, $\left[ \begin{array}{cc} -1 & 0\\ 0 & -1 \end{array}\right]$, each of determinant 1.
    By Lemma~\ref{le:transformed inequalities}, the inequality ($\dagger$) can be seen to transform by these into each of inequalities (b), (c), (d) with $i=1$.
    This establishes the base of an induction on $i$.

    Assuming the inequalities (a)-(d) hold for $i$, we apply Lemma~\ref{le:transformed inequalities} with $\phi_2^{-1}$.
    Both of the boundary rays corresponding to the inequalities (a) and (d) lie entirely in the region where $\phi_2^{-1}$ acts according to the matrix $\left[ \begin{array}{cc} -1 & -b\\ c & bc-1 \end{array}\right]$, also the boundary segment corresponding to (b) intersects this region.
    Thus applying Lemma~\ref{le:transformed inequalities} to (a) gives the inequality 
    \[u_{2i+2}^-\mu_0+u_{2i+1}^+\mu_1=(u_3^+u_{2i}^--u_2^-u_{2i-1}^+)\mu_0+(u_2^+u_{2i}^--u_1^-u_{2i-1}^+)\mu_1\le u_{2j}^-\lambda_0+u_{2j-1}^+\lambda_1,\]
    which is the inequality (a) for $i+1$ by Lemma~\ref{eq:long Chebyshev recursion}; while applying this to (d) gives the inequality 
    \[-u_{2i+1}^-\mu_0-u_{2i}^+\mu_1=(-u_3^+u_{2i-1}^-+u_2^-u_{2i-2}^+)\mu_0+(-u_2^+u_{2i-1}^-+u_1^-u_{2i-2}^+)\mu_1\le u_{2j+1}^-\lambda_0+u_{2j}^+\lambda_1,\]
    which is the inequality (d) for $i+1$ again by Lemma~\ref{eq:long Chebyshev recursion}; finally applying this to (b) gives the inequality 
    \[u_{2i+3}^-\mu_0+u_{2i+2}^+\mu_1=(u_3^+u_{2i+1}^--u_2^-u_{2i}^+)\mu_0+(u_2^+u_{2i+1}^--u_1^-u_{2i}^+)\mu_1\le u_{2j+1}^-\lambda_0+u_{2j}^+\lambda_1,\]
    which is the inequality (b) for $i+1$.
    Similarly, the boundary segment corresponding to (c) lies entirely in the region where $\phi_2^{-1}$ acts according to the matrix $\left[ \begin{array}{cc} -1 & 0\\ c & -1 \end{array}\right]$.
    Thus applying Lemma~\ref{le:transformed inequalities} to (c) gives the inequality 
    \[-u_{2i+1}^-\mu_0-u_{2i}^+\mu_1=(u_1^+u_{2i-1}^--u_2^-u_{2i}^+)\mu_0+(-u_0^+u_{2i-1}^--u_1^-u_{2i}^+)\mu_1\le u_{2j+1}^-\lambda_0+u_{2j}^+\lambda_1,\]
    which is the inequality (d) for $i+1$ by Lemma~\ref{eq:long Chebyshev recursion}, in particular we see that the segment determined by (c) and the ray determined by (d) align in the image.
    Lastly, the boundary segment corresponding to (b) also intersects the regions where $\phi_2^{-1}$ acts according to the matrices $\left[ \begin{array}{cc} -1 & -b\\ 0 & -1 \end{array}\right]$ and $\left[ \begin{array}{cc} -1 & 0\\ 0 & -1 \end{array}\right]$ respectively.
    Applying Lemma~\ref{le:transformed inequalities} to (b) with the first matrix gives the inequality 
    \[-u_{2i+1}^-\mu_0+u_{2i+2}^+\mu_1=(-u_1^+u_{2i+1}^-+u_0^-u_{2i}^+)\mu_0+(u_2^+u_{2i+1}^--u_1^-u_{2i,+})\mu_1\le u_{2j+1}^-\lambda_0+u_{2j}^+\lambda_1,\]
    which is the inequality (c) for $i+1$ by Lemma~\ref{eq:long Chebyshev recursion}, while applying Lemma~\ref{le:transformed inequalities} to (b) with the second matrix gives the inequality 
    \[-u_{2i+1}^-\mu_0-u_{2i}^+\mu_1\le u_{2j+1}^-\lambda_0+u_{2j}^+\lambda_1,\]
    which again reproduces the inequality (d) and aligns with the previous segment and ray in the image.
    This completes the induction on $i$, proving the Claim and the result for $k$ even.

    For $k=2j+1$, $j\ge0$, and $\lambda\in\cI$, we get $\cC_{2j+1}(\phi_{2j+1}\lambda)\subset\RR^2$ is given by the inequalities 
    \[ (\dagger')\ -\mu_0\le u_{2j+1}^+\lambda_0+u_{2j}^+\lambda_1 \qquad\text{and}\qquad (\ddagger')\ \mu_1\le u_{2j+2}^-\lambda_0+u_{2j+1}^-\lambda_1.\]
    Using that $\phi_{2j+1}^{-1}=\phi_2^{-j}\phi_1^{-1}$, we compute the image inductively as above.
    From \eqref{eq:forward mutation 1} and Lemma~\ref{le:imaginary stability}, we see that the boundary ray for $\cC_{2j+1}(\phi_{2j+1}\lambda)$ corresponding to ($\dagger'$) lies entirely in the region in which $\phi_1^{-1}$ acts according to the matrix $\left[ \begin{array}{cc} -1 & 0\\ c & 1 \end{array}\right]$ of determinant $-1$.
    Thus applying Lemma~\ref{le:transformed inequalities}, the inequality ($\dagger'$) is transformed by $\phi_1^{-1}$ into the inequality $\mu_0\le u_{2j+1}^+\lambda_0+u_{2j}^+\lambda_1$.
    The boundary ray corresponding to ($\ddagger'$) intersects both domains of linearity for $\phi_1^{-1}$ and thus produces the inequalities
    \[ c\mu_0+\mu_1\le u_{2j+2}^+\lambda_0+u_{2j+1}^+\lambda_1 \qquad\text{and}\qquad \mu_1\le u_{2j+2}^+\lambda_0+u_{2j+1}^+\lambda_1.\]

    \subsubsection*{Claim:} For $i\ge 0$, $\phi_{2i+1}^{-1}\cC_{2j+1}(\phi_{2j+1}\lambda)$ is the region determined by the inequalities 
    \begin{align*}
      \tag{a$'$} u_{2i+1}^-\mu_0+u_{2i}^+\mu_1 &\le u_{2j+1}^-\lambda_0+u_{2j}^+\lambda_1;\\
      \tag{b$'$} u_{2i+2}^-\mu_0+u_{2i+1}^+\mu_1 &\le u_{2j+2}^-\lambda_0+u_{2j+1}^+\lambda_1;\\
      \tag{c$'$} -u_{2i}^-\mu_0+u_{2i+1}^+\mu_1 &\le u_{2j+2}^-\lambda_0+u_{2j+1}^+\lambda_1;\\
      \tag{d$'$} -u_{2i}^-\mu_0-u_{2i-1}^+\mu_1 &\le u_{2j+2}^-\lambda_0+u_{2j+1}^+\lambda_1.
    \end{align*}
    By essentially the same calculations as above, these inequalities reproduce under the action of $\phi_2^{-1}$ and this completes the proof.
  \end{proof}

  We are now ready to prove Theorem~\ref{th:dominance inequalities}.
  The dominance region $\cP_\lambda=\bigcap_{k\in\ZZ}\phi_k^{-1}\cC_k(\phi_k\lambda)$ for $\lambda\in\cI$ is obtained by imposing all of the inequalities from Lemma~\ref{le:dominance inequalities} together with $\mu_0 -\lambda_0 \le 0$ and $0 \le \mu_1 - \lambda_1$ coming from $k=0$.

  To begin, observe that the boundary lines for \eqref{ineq:2} and \eqref{ineq:3} (and hence \eqref{ieq1} and \eqref{ieq2} below) intersect where $\mu_0=0$ and the boundary lines for \eqref{ineq:3} and \eqref{ineq:4} (and hence \eqref{ieq2} and \eqref{ieq3} with $k-1$) intersect where $\mu_1=0$.
  We then rewrite the inequalities from Lemma~\ref{le:dominance inequalities} using the Chebyshev recursion.
  For $k>0$, we get
  \begin{align*}
    u_k^-(\mu_0-\lambda_0)+u_{k-1}^+(\mu_1-\lambda_1) &\le 0;\\
    u_{k+1}^-(\mu_0-\lambda_0)+u_k^+(\mu_1-\lambda_1) &\le 0;\\
    -u_{k-1}^-(\mu_0-\lambda_0)+u_k^+(\mu_1-\lambda_1) &\le cu_k^+\lambda_0;\\
    -u_{k-1}^-(\mu_0-\lambda_0)-u_{k-2}^+(\mu_1-\lambda_1) &\le cu_k^+\lambda_0+bu_{k-1}^-\lambda_1;
  \end{align*}
  and, for $k<0$, we get
  \begin{align*}
    u_{k+1}^-(\mu_0-\lambda_0)+u_k^+(\mu_1-\lambda_1) &\le 0;\\
    u_k^-(\mu_0-\lambda_0)+u_{k-1}^+(\mu_1-\lambda_1) &\le 0;\\
    u_k^-(\mu_0-\lambda_0)-u_{k+1}^+(\mu_1-\lambda_1) &\le bu_k^-\lambda_1;\\
    -u_{k+2}^-(\mu_0-\lambda_0)-u_{k+1}^+(\mu_1-\lambda_1) &\le cu_{k+1}^+\lambda_0+bu_k^-\lambda_1.
  \end{align*}
  The first inequality in each list is redundant so we drop them.
  Moreover, for $k=1$ the third and fourth inequalities in the first list are the same so we can increment $k$ in the last equality without losing any information and similarly for $k=-1$ in the second list. 
  This gives
  \begin{align*}
    u_{k+1}^-(\mu_0-\lambda_0)+u_k^+(\mu_1-\lambda_1) &\le 0;\\
    -u_{k-1}^-(\mu_0-\lambda_0)+u_k^+(\mu_1-\lambda_1) &\le cu_k^+\lambda_0;\\
    -u_k^-(\mu_0-\lambda_0)-u_{k-1}^+(\mu_1-\lambda_1) &\le cu_{k+1}^+\lambda_0+bu_k^-\lambda_1;
  \end{align*}
  for $k>0$ and
  \begin{align*}
    u_k^-(\mu_0-\lambda_0)+u_{k-1}^+(\mu_1-\lambda_1) &\le 0;\\
    u_k^-(\mu_0-\lambda_0)-u_{k+1}^+(\mu_1-\lambda_1) &\le bu_k^-\lambda_1;\\
    -u_{k+1}^-(\mu_0-\lambda_0)-u_k^+(\mu_1-\lambda_1) &\le cu_k^+\lambda_0+bu_{k-1}^-\lambda_1;
  \end{align*}
  for $k<0$.
  Then, using that $-u_k^\varepsilon<0$ for $k>0$ and $u_k^\varepsilon<0$ for $k<0$, we divide by these to rewrite the inequalities again as
  \begin{align}
    \label{ieq1} 0 & \le \frac{-u_{k+1}^-}{u_k^+}(\mu_0-\lambda_0)-(\mu_1-\lambda_1);\\
    \label{ieq2} -c\lambda_0 & \le \frac{u_{k-1}^-}{u_k^+}(\mu_0-\lambda_0)-(\mu_1-\lambda_1);\\
    \label{ieq3} -(\mu_0-\lambda_0)-\frac{u_{k-1}^+}{u_k^-}(\mu_1-\lambda_1) &\le c\frac{u_{k+1}^+}{u_k^-}\lambda_0+b\lambda_1;
  \end{align}
  for $k>0$ and
  \begin{align}
    \label{ieq4} 0 &\le (\mu_0-\lambda_0)+\frac{u_{k-1}^+}{u_k^-}(\mu_1-\lambda_1);\\
    \label{ieq5} b\lambda_1 &\le (\mu_0-\lambda_0)-\frac{u_{k+1}^+}{u_k^-}(\mu_1-\lambda_1);\\
    \label{ieq6} c\lambda_0+b\frac{u_{k-1}^-}{u_k^+}\lambda_1 &\le \frac{-u_{k+1}^-}{u_k^+}(\mu_0-\lambda_0)-(\mu_1-\lambda_1);
  \end{align}
  for $k<0$.
  We now study each sequence of inequalities in turn.

  For $\mu_1-\lambda_1\ge 0$, the inequalities \eqref{ieq1} become more restrictive as $k$ gets larger since the sequence $\frac{-u_{k+1}^-}{u_k^+}$ of negative slopes is monotonically increasing (cf. Lemma~\ref{le:limits}).
  Thus, following Lemma~\ref{le:limits}, in the limit as $k\to\infty$, we obtain the inequality 
  \[ 0 \le \frac{-bc-\sqrt{bc(bc-4)}}{2b}(\mu_0-\lambda_0)-(\mu_1-\lambda_1) \]
  or, equivalently,
  \[ 0 \le -(\mu_0-\lambda_0)-\frac{bc-\sqrt{bc(bc-4)}}{2c}(\mu_1-\lambda_1) \]
  determining a boundary of $\cP_\lambda$.

  Similarly, the inequalities \eqref{ieq2} become more restrictive for $\mu_0\le0$ and $\mu_1\ge0$ as $k$ gets larger since the sequence $\frac{u_{k-1}^-}{u_k^+}$ of positive slopes is monotonically increasing and the intersection with \eqref{ieq1} moves lower on the $\mu_1$-axis as $k$ increases.
  Thus, following Lemma~\ref{le:limits}, in the limit as $k\to\infty$, we obtain the inequality
  \[ -c\lambda_0 \le \frac{bc-\sqrt{bc(bc-4)}}{2b}(\mu_0-\lambda_0)-(\mu_1-\lambda_1) \]
  determining a boundary of $\cP_\lambda$.
  Observe further that the inequalities $\mu_0-\lambda_0 \le 0$ and $0 \le \mu_1-\lambda_1$ allow to strengthen this as
  \[ -c\lambda_0 \le \frac{bc-\sqrt{bc(bc-4)}}{2b}(\mu_0-\lambda_0)-(\mu_1-\lambda_1) \le 0. \]

  Finally, the inequalities \eqref{ieq3} become more restrictive for $\mu_1\le0$ as $k$ gets larger since the sequence $\frac{-u_k^-}{u_{k-1}^+}$ of negative slopes is monotonically increasing (cf. Lemma~\ref{le:limits}) and the intersection with \eqref{ieq2} (for $k+1$) moves to the right on the $\mu_0$-axis as $k$ increases.
  Thus, following Lemma~\ref{le:limits}, in the limit as $k\to\infty$, we obtain the inequality
  \[ -(\mu_0-\lambda_0)-\frac{bc-\sqrt{bc(bc-4)}}{2c}(\mu_1-\lambda_1) \le \frac{bc+\sqrt{bc(bc-4)}}{2c}c\lambda_0+b\lambda_1 \]
  determining a boundary of $\cP_\lambda$.

  %Similar arguments with the inequalities \eqref{ieq4}-\eqref{ieq6} lead to the remaining inequalities determining the boundary of $\cP_\lambda$:
  Similar arguments using \eqref{ieq4}-\eqref{ieq6} lead to the remaining inequalities determining the boundary of $\cP_\lambda$:
  \[ 0 \le (\mu_0-\lambda_0)+\frac{bc+\sqrt{bc(bc-4)}}{2c}(\mu_1-\lambda_1);\]
  \[ b\lambda_1 \le (\mu_0-\lambda_0)-\frac{bc-\sqrt{bc(bc-4)}}{2c}(\mu_1-\lambda_1) \le 0;\]
  \[ c\lambda_0+\frac{bc+\sqrt{bc(bc-4)}}{2b}b\lambda_1 \le -\frac{bc-\sqrt{bc(bc-4)}}{2b}(\mu_0-\lambda_0)-(\mu_1-\lambda_1).\]
  These can easily be seen to be equivalent to the remaining inequalities from Theorem~\ref{th:dominance inequalities} and this complete the proof.


\section{Proof of Corollary~\ref{cor:dominance vertices}}
\label{sec:dominance vertices}

  This follows from basic manipulations finding the intersection points of the boundary segments determined by the inequalities from Theorem~\ref{th:dominance inequalities}.
  Note that in each of the cases (2), (4), and (6) there are only four inequalities to consider while in cases (3) and (5) there are only three inequalities to consider.
  We leave the details as an exercise for the reader.


\section{Proof of Theorem~\ref{th:maximum support}}
\label{sec:maximum support} 


  Every pointed basis element for $\cA(b,c)$ admits an opposite $\bfg$-vector arising by interchanging the roles of $b$, $c$ and $x_0$, $x_1$ in \eqref{eq:pointed}. 
  One can easily compute the following correspondence.
  \begin{lemma}
    Given a $\bfg$-vector $\lambda$, the opposite $\bfg$-vector $\lambda'$ is obtained as follows:
    \begin{itemize}
      \item if $\lambda_0,\lambda_1\ge0$, then $\lambda'=\lambda$;
      \item if $\lambda_1\ge0$ and $\lambda_0<0$, then $\lambda'=(\lambda_0,-c\lambda_0+\lambda_1)$;
      \item if $\lambda_0>0$ and $\lambda_0+b\lambda_1>0$, then $\lambda'=(\lambda_0+b\lambda_1,\lambda_1)$;
      \item otherwise, $\lambda'=(\lambda_0+b\lambda_1,-c\lambda_0-(bc-1)\lambda_1)$.
    \end{itemize}
  \end{lemma}

  In particular, we obtain an opposite dominance region $\cP'_\lambda$ for each $\bfg$-vector $\lambda$.
  \begin{lemma}
    For $\lambda\in\cI$, the opposite dominance polygon $\cP'_\lambda$ pointed at its opposite $\bfg$-vector $\lambda'=(\lambda'_0,\lambda'_1)$ is the region consisting of those $\mu\in\RR^2$ satisfying $\mu_0 \geq \lambda'_0, \mu_1 \leq\lambda'_1$, and the following inequalities:
    {
      \everymath={\displaystyle}
      \def\arraystretch{2.8}
      \[
        \begin{array}{rcccl}
          0 & \leq & \frac{b c-\sqrt{b c (b c-4)}}{2 c}(\mu_1-\lambda'_1)+(\mu_0-\lambda'_0) & \leq & -b\lambda'_1-\frac{b c+\sqrt{b c (b c-4)}}{2c}c\lambda'_0
          \\
          -b\lambda'_1 & \leq & \frac{b c-\sqrt{b c (b c-4)}}{2 c}(\mu_1-\lambda'_1)-(\mu_0-\lambda'_0)
          \\
          0 & \leq &  -(\mu_1-\lambda'_1)-\frac{b c-\sqrt{b c (b c-4)}}{2 b}(\mu_0-\lambda'_0) & \leq & \frac{b c+\sqrt{b c (b c-4)}}{2b}b\lambda'_1+c\lambda'_0
          \\
          c \lambda'_0 & \leq & (\mu_1-\lambda'_1) - \frac{b c-\sqrt{b c (b c-4)}}{2 b} (\mu_0-\lambda'_0)
        \end{array}
      \]
    }
  \end{lemma}
  \begin{proof}
    As stated above, we obtain the opposite $\bfg$-vector by interchanging $b$, $c$ and swapping the roles of $\lambda_0,\lambda_1$.
    Translating from Theorem~\ref{th:dominance inequalities}, it immediately follows that the opposite dominance region is given by the claimed inequalities.
    %{
    %  \everymath={\displaystyle}
    %  \def\arraystretch{2.8}
    %  \[
    %    \begin{array}{rcccl}
    %      0 & \leq & -\frac{b c+\sqrt{b c (b c-4)}}{2 b}(\mu_0-\lambda_0)-(\mu_1-\lambda_1) & \leq & c\lambda_0 + \frac{b c-\sqrt{b c (b c-4)}}{2 b}b\lambda_1
    %      \\
    %       & & (\mu_0-\lambda_0) -\frac{b c-\sqrt{b c (b c-4)}}{2 c}(\mu_1-\lambda_1) & \leq & -\frac{b c+\sqrt{b c (b c-4)}}{2}(\lambda_0+b\lambda_1)+2b\lambda_1
    %      \\
    %      \frac{b c - 2 +\sqrt{b c (b c-4)}}{2}(c\lambda_0+(bc-2)\lambda_1) -c\lambda_0 -2\lambda_1 & \leq & \frac{b c-\sqrt{b c (b c-4)}}{2 b}(\mu_0-\lambda_0) + (\mu_1-\lambda_1) & \leq & -c\lambda_0 - \frac{b c+\sqrt{b c (b c-4)}}{2 b}b\lambda_1 
    %      \\
    %      -c\lambda_0+\frac{b c-2+\sqrt{b c (b c-4)}}{2}(c\lambda_0 + \sqrt{b c (b c-4)}\lambda_1) & \leq & \frac{b c-\sqrt{b c (b c-4)}}{2 b}(\mu_0-\lambda_0) + (\mu_1-\lambda_1) & \leq & -c\lambda_0 - \frac{b c+\sqrt{b c (b c-4)}}{2 b}b\lambda_1 
    %      \\
    %      b\lambda_1 & \leq & - (\mu_0-\lambda_0) + \frac{b c+\sqrt{b c (b c-4)}}{2 c} (\mu_1-\lambda_1) 
    %    \end{array}
    %  \]
    %}
  \end{proof}

  For $\lambda\in\cI$, the maximal support $\cS_\lambda$ is determined by the following inequalities:
  \begin{align*}
    \lambda_1 &\leq \mu_1\\
    \lambda'_0 &\leq \mu_0\\
    0 &\leq -\frac{b c+\sqrt{b c (b c-4)}}{2 b}(\mu_0-\lambda_0)-(\mu_1-\lambda_1)\\
    0 &\leq -\frac{b c-\sqrt{b c (b c-4)}}{2 b}(\mu_0-\lambda'_0) -(\mu_1-\lambda'_1)
  \end{align*}
  We claim that $\cP_\lambda,\cP'_\lambda\subset\cS_\lambda$.

  Observe that the last equality can be rewritten as
  \[ \frac{b c-\sqrt{b c (b c-4)}}{2 b}(\mu_0-\lambda_0) + (\mu_1-\lambda_1) \leq -c\lambda_0 - \frac{b c + \sqrt{b c (b c-4)}}{2 } \lambda_1,\]
  which together with the second to last inequality already gives two of the boundaries for the dominance region $\cP_\lambda$.
  Note then that, under the assumption $\mu_0\leq\lambda_0$, the inequality 
  \[0 \leq \frac{b c-\sqrt{b c (b c-4)}}{2 b}(\mu_0-\lambda_0)+(\mu_1-\lambda_1)\]
  defining another boundary of $\cP_\lambda$ is more restrictive than the inequality $\lambda_1\leq\mu_1$ bounding $\cS_\lambda$.

  It follows from Corollary~\ref{cor:dominance vertices} that the minimum value for $\mu_0$ inside $\cP_\lambda$ occurs when $\mu_1=0$, i.e. either at the point $\big( \lambda_0+\frac{bc+\sqrt{bc(bc-4)}}{2c}\lambda_1 , 0 \big)$ when $0 \leq c\lambda_0+2\lambda_1$ or at the point $-\frac{bc+\sqrt{bc(bc-4)}}{2c} \big( \frac{bc+\sqrt{bc(bc-4)}}{2b}\lambda_0+\lambda_1 , 0 \big)$ when $c\lambda_0+2\lambda_1<0$.
  The second point can be rewritten as
  \[\big(\lambda_0 -\frac{bc+\sqrt{bc(bc-4)}}{2c} c\lambda_0-\frac{bc+\sqrt{bc(bc-4)}}{2c} \lambda_1 , 0 \big)\]
  but, since $c\lambda_0 < -2\lambda_1$, the first coordinate is greater than $\lambda_0+\frac{bc+\sqrt{bc(bc-4)}}{2c}\lambda_1$.
  Then, using $\frac{bc+\sqrt{bc(bc-4)}}{2c}<b$ and $\lambda_1<0$, we see that the minimum value of $\mu_0$ inside $\cP_\lambda$ satisfies the inequality $\lambda'_0=\lambda_0+b\lambda_1 \leq \mu_0$.
  In particular, combining with the observation above, we see that $\cP_\lambda\subset\cS_\lambda$.

  Similarly, the second to last inequality can be rewritten as
  \[ \frac{b c-\sqrt{b c (b c-4)}}{2 c}(\mu_1-\lambda'_1)+(\mu_0-\lambda'_0) \leq -b\lambda'_1-\frac{b c+\sqrt{b c (b c-4)}}{2c}c\lambda'_0,\]
  which together with the last inequality already gives two of the boundaries for the opposite dominance region $\cP'_\lambda$.
  Note then that, under the assumption $\mu_1\leq\lambda'_1$, the inequality
  \[ 0 \leq \frac{b c-\sqrt{b c (b c-4)}}{2 c}(\mu_1-\lambda'_1)+(\mu_0-\lambda'_0) \]
  defining another boundary of $\cP_\lambda$ is more restrictive than the inequality $\lambda'_0\leq\mu_0$ bounding $\cS_\lambda$.
  As above, the minimum value of $\mu_1$ inside $\cP'_\lambda$ occurs when $\mu_0=0$, i.e. either at the point $\big(0, \frac{bc+\sqrt{bc(bc-4)}}{2b}\lambda'_0+\lambda'_1 \big)$
  %\[\big(0, \frac{bc+\sqrt{bc(bc-4)}}{2b}\lambda'_0+\lambda'_1 \big) = \big(0, -\frac{bc-\sqrt{bc(bc-4)}}{2b}(\lambda_0+b\lambda_1)+\lambda_1 \big)\]
  when $0 \leq 2\lambda'_0+b\lambda'_1$ or at the point $-\frac{bc+\sqrt{bc(bc-4)}}{2b} \big(0, \lambda'_0+\frac{bc+\sqrt{bc(bc-4)}}{2c}\lambda'_1 \big)$
  %\[-\frac{bc+\sqrt{bc(bc-4)}}{2b} \big(0, \lambda'_0+\frac{bc+\sqrt{bc(bc-4)}}{2c}\lambda'_1 \big) = -\frac{bc+\sqrt{bc(bc-4)}}{2b} \big(0, -\frac{bc-2+\sqrt{bc(bc-4)}}{2}(\lambda_0+b\lambda_1)+\frac{bc+\sqrt{bc(bc-4)}}{2c}\lambda_1 \big)\]
  when $2\lambda'_0+b\lambda'_1<0$.
  The second point can be rewritten as
  \[\big(0, -\frac{bc+\sqrt{bc(bc-4)}}{2b} \lambda'_0-\frac{bc+\sqrt{bc(bc-4)}}{2b} b\lambda'_1+\lambda'_1 \big)\]
  but, since $b\lambda'_1\le -2\lambda'_0$, the first coordinate is greater than $\frac{bc+\sqrt{bc(bc-4)}}{2b}\lambda'_0+\lambda'_1$.
  Then, using $\frac{bc+\sqrt{bc(bc-4}}{2b}<c$ and $\lambda'_0<0$, we see that the minimum value of $\mu_1$ inside $\cP'_\lambda$ satisfies the inequality $\lambda_1=c\lambda'_0+\lambda'_1\le\mu_1$ and so $\cP'_\lambda\subset\cS_\lambda$.

  To continue, we compare the support for an arbitrary basis element pointed at $\lambda\in\cI$ with the corresponding greedy basis elements with $\bfg$-vectors inside the dominance polygon $\cP_\lambda$.

  The support $\cG_\lambda$ of the greedy basis element of $\cA(b,c)$ with $\bfg$-vector $\lambda$ is well-known \cite{LLZ,cgmmrsw}.
  This is indicated by the solid and dashed lines in Theorem~\ref{th:maximum support} where dashed lines indicate points that are excluded from the support (note that the white dot does not indicate that this point is excluded).
  For our purposes, it is enough to observe the following closure property for the greedy support.
  \begin{definition}
    A subset $\cS\subset\RR^2$ is called \emph{min-convex} if $\lambda,\mu\in\cS$ implies the segments joining $\lambda$ and $\mu$ to $\big(\min(\lambda_0,\mu_0),\min(\lambda_1,\mu_1)\big)$ are contained in $\cS$.
  \end{definition}
  In particular, (an upper bound for) the greedy support can be found using the $\bfg$-vector $\lambda$ indicated by a black dot and the opposite $\bfg$-vector $\lambda'$ indicated by a white dot in Theorem~\ref{th:maximum support}.
  Write $\overline{\cG}_\lambda$ for the downward scaling of the min-convex closure of $\{\lambda,\lambda'\}$, that is $\overline{\cG}_\lambda$ contains the segments joining $\lambda$ and $\lambda'$ with $\big(\min(\lambda_0,\lambda'_0),\min(\lambda_1,\lambda'_1)\big)$ and for any $\mu\in\overline{\cG}_\lambda$ we have $t\mu\in\overline{\cG}_\lambda$ for $0\le t\le 1$.
  Clearly, $\cG_\lambda\subset\overline{\cG}_\lambda$.

  As we saw above, for any $\bfg$-vector $\mu\in\cP_\lambda$ its opposite $\bfg$-vector $\mu'$ is also contained in $\cS_\lambda$.
  But $\cS_\lambda$ is min-convex and closed under downward scaling, i.e. for any $\mu\in\cS_\lambda$ we have $t\mu\in\cS_\lambda$ for $0\le t\le 1$.
  It follows that $\overline{\cG}_\mu\subset\cS_\lambda$ for any $\mu\in\cP_\lambda$ and hence, following Theorem~\ref{th:dominance}, the support of any basis element pointed at $\lambda$ is contained in $\cS_\lambda$.

  In particular, $\cG_\lambda\subset\cS_\lambda$.
  Note also that the dominance region $\cP_\lambda$ contains the intersection of $\cS_\lambda$ with the region $\cR_1$ defined by the inequalities $\mu_0\ge0$ and $\lambda_1\mu_0-\lambda_0\mu_1\ge0$.
  Similarly, the opposite dominance region $\cP'_\lambda$ contains the intersection of $\cS_\lambda$ with the region $\cR_2$ defined by the inequalities $\mu_1\ge0$ and $-\lambda'_1\mu_0+\lambda'_0\mu_1\ge0$.
  But $\cS_\lambda=\cG_\lambda\cup(\cS_\lambda\cap\cR_1)\cup(\cS_\lambda\cap\cR_2)$ so $\cS_\lambda$ is the maximum possible support for an element pointed at~$\lambda$.


\section{The affine case}
\label{sec:affine}

  In this section we compare our main result with the construction of \cite{RSW19} in the case $b=c=2$.
  To match conventions, in this section, we work over an algebraically closed field $\kk$ of characteristic 0.
  Because the exchange matrix is full rank there is no loss of generality in continuing to work in the coefficient-free case.
  We will identify the family of bases in Theorem \ref{th:dominance} with the continuous family of bases of $\cA(2,2)$ constructed in \cite{RSW19} from generalized minors, which we recast here with the current notation. 

  \begin{theorem}[{\cite[Theorem 4.6]{RSW19}}]
    \label{thm:rsw}
    Choose a point $\bfa^{(n)}=(a_1,\dots,a_n)\in(\kk^\times)^n$ for each $n\geq1$.
    Then, together with all cluster monomials, the elements 
    \begin{equation}
      \label{eq:generalized minor}
      x_{(n,-n)}^{\bfa^{(n)}}:= x_0^{-n} x_1^{-n} \sum_{0 \leq k \leq \ell \leq n} \sum\limits_{r=0}^\ell {\ell-r\choose k} {n-2r\choose \ell-r} S_{\bfa^{(n)},r}  \, x_0^{2(\ell-k)} x_1^{2(n-\ell)}
    \end{equation}
    with
    \[
      S_{\bfa^{(n)},r}=\sum\limits_{\substack{I,J\subseteq[1,n]\\|I|=r=|J|\\ I\cap J=\varnothing}}\frac{\prod_{i\in I} a_i}{\prod_{j\in J} a_j}
    \]
    form a linear basis of $\cA(2,2)$.
  \end{theorem}

  We begin by noting that when $b=c=2$ the imaginary cone $\cI$ degenerates to the ray spanned by $(1,-1)$.
  For any $\lambda\in\cI$, the dominance region $\cP_\lambda$ is the segment connecting $\lambda$ to the origin.
  It follows that $\bfg$-vectors dominated by $\lambda=(n,-n)$ are of the form $(n-2r,-n+2r)$ for $0\leq r \leq n/2$.

  In order to use Theorem \ref{th:dominance} we need to fix a reference pointed basis of $\cA(2,2)$; to simplify our computations, we choose to work with the \emph{generic basis}.
  This basis consists of the cluster monomials of $\cA(2,2)$ together with the elements
  \[
    x_{(n,-n)}^{ge}
    :=
		\Big(x_0 x_1^{-1} + x_0^{-1}x_1^{-1} + x_0^{-1}x_1\Big)^n
		=
    x_0^{-n} x_1^{-n}
    \sum_{0 \leq k \leq \ell \leq n}
    {\ell\choose k} {n\choose \ell}
    \, x_0^{2(\ell-k)} x_1^{2(n-\ell)}
  \]

  \begin{proposition}
    \label{prop:rewrite}
    For $n\ge0$, we have
    \[
      x_{(n,-n)}^{\bfa^{(n)}}
      =
      \sum_{r=0}^{\lfloor n/2 \rfloor}
      S_{\bfa^{(n)},r}
      \,
      x_{(n-2r,-n+2r)}^{ge}.
    \]
  \end{proposition}
  \begin{proof}
    To begin, we observe that \eqref{eq:generalized minor} may be rewritten as
    \[
      x_{(n,-n)}^{\bfa^{(n)}} = x_0^{-n} x_1^{-n} \sum_{k=0}^n \sum_{\ell=k}^n \sum_{r=0}^\ell {\ell-r\choose k} {n-2r\choose \ell-r} S_{\bfa^{(n)},r}  \, x_0^{2(\ell-k)} x_1^{2(n-\ell)}.
    \]
    The first binomial coefficient above is zero if $0 \le \ell-r < k$ while the second is zero for $\ell < r \le \lfloor n/2\rfloor$ or if $n-r < \ell$, therefore $x_{(n,-n)}^{\bfa^{(n)}}$ can be expressed as
    \begin{align*}
      x_{(n,-n)}^{\bfa^{(n)}}
      &=
      x_0^{-n} x_1^{-n} \sum_{k=0}^n \sum_{r=0}^{\lfloor n/2\rfloor} \sum_{\ell=k+r}^{n-r} {\ell-r\choose k} {n-2r\choose \ell-r} S_{\bfa^{(n)},r}  \, x_0^{2(\ell-k)} x_1^{2(n-\ell)}.
    \end{align*}
    But then rearranging terms and replacing $\ell$ by $\ell+r$, this becomes
    \begin{align*}
      x_{(n,-n)}^{\bfa^{(n)}}
      &=
      \sum_{r=0}^{\lfloor n/2\rfloor} S_{\bfa^{(n)},r}\, x_0^{-n+2r} x_1^{-n+2r} \sum_{k=0}^n \sum_{\ell=k}^{n-2r} {\ell\choose k} {n-2r\choose \ell} \, x_0^{2(\ell-k)} x_1^{2(n-2r-\ell)}
      =
      \sum_{r=0}^{\lfloor n/2\rfloor} S_{\bfa^{(n)},r}\, x_{n-2r,-n+2r}^{ge}
    \end{align*}
    as desired.
  \end{proof}

  \begin{remark}
    \label{rk:symmetric}
    Observe that the expansion coefficients $S_{\bfa^{(n)},r}$ can be expressed as the ratio of monomial symmetric functions $\frac{m_{2^{(r)},1^{(n-2r)}}}{m_{1^{(n)}}}$ evaluated at $\bfa^{(n)}$.
    The analogous expansion coefficients when $x_{(n,-n)}^{\bfa^{(n)}}$ is expressed in terms of the triangular basis (resp. greedy basis) are the ratios of Schur functions $\frac{s_{2^{(r)},1^{(n-2r)}}}{s_{1^{(n)}}}$ (resp. ratios of elementary symmetric functions $\frac{e_{r,n-r}}{e_n}$) evaluated at $\bfa^{(n)}$.
    We leave the details to the reader.
  \end{remark}

  \begin{theorem}
    As the points $\bfa^{(n)}$ vary in $(\kk^\times)^n$, the bases in Theorem \ref{thm:rsw} recover precisely all the pointed bases of $\cA(2,2)$.
  \end{theorem}
  \begin{proof}
    By Theorem \ref{th:dominance}, in view of Proposition \ref{prop:rewrite} and the discussion immediately before it, it suffices to show that as $\bfa^{(n)}$ vary in $(\kk^\times)^n$ the tuple of coefficients $\big(S_{\bfa^{(n)},r}\big)_{1\leq r \leq \lfloor n/2\rfloor}$ assume all the values in $\kk^{\lfloor n/2\rfloor}$.
    (The fact that $S_{\bfa^{(n)},0}=1$ is immediate from the definition.)

    By Remark \ref{rk:symmetric}, this is equivalent to the fact that the symmetric monomial functions in $n$ variables $m_{2^{(r)},1^{(n-2r)}}$ are algebraically independent.
    But this follows immediately from the observation that 
    \[ 
      m_{2^{(r)},1^{(n-2r)}} = \sum_{i=0}^r c_i\, e_{n-i,i}
    \]
    for some coefficients $c_i$ with $c_r=1$, and the fundamental theorem of symmetric polynomials. 
  \end{proof}

  \begin{thebibliography}{XXX}
    \bibitem{cgmmrsw}
      our other paper

    \bibitem{ghkk}
      Gross et al

    \bibitem{lee-li-zelevinsky}
      K.~Lee, L.~Li, A.~Zelevinsky: Greedy elements in rank 2 cluster algebras. Selecta Math. \textbf{20} (2014), pp.~57--82.

    \bibitem{qin}
      Fan's paper

    %\bibitem{rupel}
      %D.~Rupel: Rank two non-commutative Laurent phenomenon and pseudo-positivity.  Algebraic Combinatorics, \textbf{2} (2019) no. 6, pp.~1239--1273. \href{https://doi.org/10.5802/alco.81}{DOI: 10.5802/alco.81}.

    \bibitem{RSW19}
      Our paper
  \end{thebibliography}


  
\end{document}

  \section{Bases From Compatible Pairs on Dyck Paths}
  \sayDR{This is a bit of a repetition of the above introduction to Dyck paths, but with slightly different goals it seemed that a modified exposition was more appropriate.}
  For $a_1,a_2\in\ZZ_{\ge0}$, choose a Dyck path $D$ in the lattice rectangle $[0,a_1]\times[0,a_2]$.
  When we wish to emphasize the upper right bounding corner we will write $D[a_1,a_2]$.
  Consider $D$ as the totally ordered set of its edges along the natural labeling from $(0,0)$ to $(a_1,a_2)$.
  It may be convenient to identify $D$ with the ordered set $[1,a_1+a_2]$.
  Write $D=D_1\sqcup D_2$, where $D_1=\{h_1,\ldots,h_{a_1}\}$ is the set of horizontal edges and $D_2=\{v_1,\ldots,v_{a_2}\}$ is the set of vertical edges.
  Define the \emph{height}, $\hgt(h)$ for $h\in D_1$, as one more than the number of $v\in D_2$ with $v<h$.
  Define the \emph{depth}, $\dpt(v)$ for $v\in D_2$, as the number of $h\in D_1$ with $h<v$.
  \begin{remark}
    The seemingly different conventions for computing height and depth provide uniformity later.
    Indeed, the vertical edges immediately following $h_d$ have depth $d$, while this convention allows the horizontal edges immediately preceding $v_\ell$ to similarly be said to have height $\ell$.
  \end{remark}

  Observe that the data of $D$ is just the data of a bipartition of a finite totally ordered set $D=D_1\sqcup D_2$.
  Given a Dyck path $D=D_1\sqcup D_2$ in the lattice rectangle $[0,a_1]\times[0,a_2]$, write $\bar{D}=\bar{D}_1 \sqcup \bar{D}_2$ for the Dyck path (under the identification with $[1,a_1+a_2]$) in the lattice rectangle $[0,a_2]\times[0,a_1]$ with $\bar{D}_1=a_1+a_2+1-D_2$ and $\bar{D}_2=a_1+a_2+1-D_1$.
  This operation can be visualized as transposing the Dyck path.

  Given any quantity or object $P$ whose definition depends on $b$ and $c$, write $\vv{P}$ for the same quantity or object defined with $b$ and $c$ interchanged.
  In particular, we have $\vv{u}_{m,\varepsilon}=u_{m,-\varepsilon}$ for any $m$ and any $\varepsilon=\pm1$.

  \subsection{Structure of Dyck Paths}

  Write $D^{max}=D^{max}[a_1,a_2]$ for the maximal Dyck path in the lattice rectangle $[0,a_1]\times[0,a_2]$, i.e. $D^{max}$ never crosses above the diagonal line joining $(0,0)$ to $(a_1,a_2)$ and any lattice point above $D^{max}$ also lies above the diagonal.
  Observe that $\bar{D}^{max}[a_1,a_2]=D^{max}[a_2,a_1]$.

  \begin{definition}
    Call a Dyck path $D$ 
    \begin{itemize}
      \item \emph{horizontally-adapted to $b$} if no vertical edge of $D$ is immediately preceded by more than $b$ horizontal edges;
      \item \emph{vertically-adapted to $c$} if no horizontal edge of $D$ is immediately followed by more than $c$ vertical edges.
    \end{itemize}
  \end{definition}

  \begin{definition}
    \label{def:Dyck path mutations}
    Given a Dyck path $D=D[a_1,a_2]$ which is vertically-adapted to $c$, define a Dyck path $\mu_{1,c}(D)$ inside the lattice rectangle $[0,ca_1-a_2]\times[0,a_1]$ by replacing each horizontal edge of $D$, together with the $r$ vertical edges which immediately follow it, by $c-r$ horizontal edges followed by a vertical edge.
    This provides a canonical identification of the horizontal edges of $D$ with the vertical edges of $\mu_{1,c}(D)$, this identification is again denoted by $\mu_{1,c}$.

    Given a Dyck path $D=D[a_1,a_2]$ which is horizontally-adapted to $b$, define a Dyck path $\mu_{2,b}(D)$ inside the lattice rectangle $[0,a_2]\times[0,ba_2-a_1]$ by replacing each vertical edge of $D$, together with the $r$ horizontal edges which immediately precede it, by a horizontal edge followed by $b-r$ vertical edges.
    This provides a canonical identification of the vertical edges of $D$ with the horizontal edges of $\mu_{2,b}(D)$, this identification is again denoted by $\mu_{2,b}$.
  \end{definition}
  \begin{remark}
    Observe that $\mu_{1,a}$ and $\mu_{2,a}$ are inverse for any $a\ge1$ whenever they are defined.
  \end{remark}

  The following result is proven in \cite[Lemma 2.4]{rupel} with appropriate notation changes, cf. Theorem~\ref{th:dominant Dyck path recursion} for a slight modification and explicit proof.
  \begin{lemma}\mbox{}
    \label{le:maximal Dyck path recursion}
    \begin{enumerate}
      \item If $a_2 < c a_1$, then $\mu_{1,c}\big(D^{max}[a_1,a_2]\big)=D^{max}[ca_1-a_2,a_1]$.
      \item If $a_1 < b a_2$, then $\mu_{2,b}\big(D^{max}[a_1,a_2]\big)=D^{max}[a_2,ba_2-a_1]$.
    \end{enumerate}
  \end{lemma}

  As an immediate consequence, we determine the structure of certain maximal Dyck paths.
  We introduce the following notation for convenience:
  \[d_m:=\begin{cases} b & \text{if $m$ is even;}\\ c & \text{if $m$ is odd;}\end{cases} \qquad \text{and} \qquad \vv{d}_m:=\begin{cases} c & \text{if $m$ is even;}\\ b & \text{if $m$ is odd;}\end{cases}\]
  The following was proven in \cite[Corollary 2.6]{rupel} and follows by a simple induction using Lemma~\ref{le:maximal Dyck path recursion}.
  \begin{corollary}
    Assume $b,c\ge2$ (this can be dropped with extra notation and slight modification).
    Then, for $\varepsilon=\pm1$ and $m\ge1$, the maximal Dyck path $D^{m,\varepsilon}:=D^{max}[u_{m,\varepsilon},u_{m-1,-\varepsilon}]$ can be constructed as follows:
    \begin{itemize}
      \item $D^{1,\varepsilon}$ consists of a single horizontal edge;
      \item $D^{2,\varepsilon}$ consists of $d_2$ horizontal edges followed by a vertical edge;
      \item for $m\ge3$, $D_{m,\varepsilon}$ consists of $d_m - 1$ copies of $D^{m-1,\varepsilon}$ followed by a copy of $D^{m-1,\varepsilon}$ with its first $D^{m-2,\varepsilon}$ removed;
    \end{itemize}
    and the maximal Dyck path $\vv{\bar{D}}^{m,\varepsilon}:=D^{max}[u_{m-1,\varepsilon},u_{m,-\varepsilon}]$ can be constructed as follows:
    \begin{itemize}
      \item $\vv{\bar{D}}^{1,\varepsilon}$ consists of a single vertical edge;
      \item $\vv{\bar{D}}^{2,\varepsilon}$ consists of a horizontal edge followed by $\vv{d}_2$ vertical edges;
      \item for $m\ge3$, $\vv{\bar{D}}^{m,\varepsilon}$ consists of a copy of $\vv{\bar{D}}^{m-1,\varepsilon}$ with its last $\vv{\bar{D}}^{m-2,\varepsilon}$ removed followed by $\vv{d}_m - 1$ copies of $\vv{\bar{D}}^{m-1,\varepsilon}$.
    \end{itemize}
  \end{corollary}

  Our main interest will be with the following Dyck paths.
  \begin{definition}
    \label{def:dominant Dyck path}
    For $a_1,a_2\in\ZZ_{\ge0}$ with $(-a_1+ba_2,-a_2)\in\cI$, define the \emph{dominant Dyck path} $D^{dom}=D^{dom}[a_1,a_2]$ to be the maximal Dyck path starting from $(0,0)$ and terminating at $(a_1,a_2)$ while staying below the lines $y=\frac{bc+\sqrt{bc(bc-4)}}{2b}x$ and $y-a_2=\frac{bc-\sqrt{bc(bc-4)}}{2b}(x-a_1)$.

    For $a_1,a_2\in\ZZ_{\ge0}$ with $(-a_2,-a_1+ca_2)\in\cI'$, define the \emph{dual dominant Dyck path} $\vv{D}^{dom}=\vv{D}^{dom}[a_1,a_2]$ to be the maximal Dyck path starting from $(0,0)$ and terminating at $(a_1,a_2)$ while staying below the lines $y=\frac{bc+\sqrt{bc(bc-4)}}{2c}x$ and $y-a_2=\frac{bc-\sqrt{bc(bc-4)}}{2c}(x-a_1)$.
  \end{definition}
  \begin{remark}
    \label{rem:intersection}
    The lines defining $D^{dom}[a_1,a_2]$ meet at the point 
    \[m_{a_1,a_2}:=\left( \frac{2ba_2-\big(bc-\sqrt{bc(bc-4)}\big)a_1}{2\sqrt{bc(bc-4)}} , \frac{\big(bc+\sqrt{bc(bc-4)}\big)a_2-2ca_1}{2\sqrt{bc(bc-4)}} \right).\]
  \end{remark}

  To understand the structure of the dominant Dyck paths, it will be convenient to consider their initial and final segments separately.
  \begin{definition}
    For $s>0$, define the \emph{Dyck path prefix} $\overrightarrow{D}^s$ as the maximal Dyck path with $s$ vertical edges starting from $(0,0)$ and terminating at $(d,s)$ while staying below the line $y=\frac{bc+\sqrt{bc(bc-4)}}{2b}x$, where $d=d_s=\left\lceil s \frac{bc-\sqrt{bc(bc-4)}}{2c}\right\rceil$.
    Call this $d$ the \emph{depth} of $s$.

    For $t>0$, define the \emph{Dyck path suffix} $\overleftarrow{D}^t$ as the maximal Dyck path with $t$ horizontal edges starting from $(0,0)$ and terminating at $(t,h)$ while staying below the line $y-h=\frac{bc-\sqrt{bc(bc-4)}}{2b}(x-t)$, where $h=h_t=\left\lceil t \frac{bc-\sqrt{bc(bc-4)}}{2b}\right\rceil$.
    Call this $h$ the \emph{height} of $t$.
  \end{definition}
  \begin{remark}
    There often exist multiple choices of $s$ with a given depth $d$ and multiple choices of $t$ with a given height $h$.
  \end{remark}
  \sayDR{ToDo: describe the analogue of Corollary 7.5 for the Dyck path prefixes and suffixes.}

  The following result will be useful for showing that the Dyck path prefixes and suffixes behave properly.
  \begin{lemma}
    For $s>0$, the Dyck path prefix $\overrightarrow{D}^s$ ends in a vertical edge.
    Similarly, for $t>0$, the Dyck path suffix $\overleftarrow{D}^t$ begins with a horizontal edge.
  \end{lemma}

  \begin{lemma}
    For $s,t>0$, $\overrightarrow{D}^s=D^{dom}[d_s,s]$ and $\overleftarrow{D}^t=D^{dom}[t,h_t]$.
  \end{lemma}
  \begin{proof}
    Is this obvious?
    %Following Remark~\ref{rem:intersection}, the intersection point of the lines defining $D^{dom}[d_s,s]$ is given by
    %\[m_{d_s,s}=\left( \frac{2bs-\big(bc-\sqrt{bc(bc-4)}\big)d_s}{2\sqrt{bc(bc-4)}} , \frac{\big(bc+\sqrt{bc(bc-4)}\big)s-2cd_s}{2\sqrt{bc(bc-4)}} \right).\]
    %By definition of $d_s$, we have
    %\[\frac{\big(bc+\sqrt{bc(bc-4)}\big)s-2cd_s}{2\sqrt{bc(bc-4)}} \le \frac{\big(bc+\sqrt{bc(bc-4)}\big)s-\big(bc-\sqrt{bc(bc-4)}\big)s}{2\sqrt{bc(bc-4)}} = s\]
  \end{proof}

  \begin{lemma}
    For any $s,t>0$, the Dyck path prefix $\overrightarrow{D}^s$ and Dyck path suffix $\overleftarrow{D}^t$ are both horizontally-adapted to $b$ and vertically-adapted to $c$.
  \end{lemma}
  \begin{proof}
    Observe that $\frac{c}{2}<\frac{bc+\sqrt{bc(bc-4)}}{2b}<c$.
    This shows that $\overrightarrow{D}^s$ cannot contain more than $c$ consecutive vertical edges following a horizontal edge and so $\overrightarrow{D}^s$ must be vertically-adapted to $c$.
    In addition, we see that $\overrightarrow{D}^s$ can only contain consecutive horizontal edges if $c=1$, in which case there are at most two consecutive horizontal edges preceding any vertical edge.
    But, since we assume $bc\ge4$, that would imply $b\ge4$ and so $\overrightarrow{D}^s$ must be horizontally-adapted to $b$.
  \end{proof}

  \begin{theorem}
    \label{th:dominant Dyck path recursion}
    For $a_1,a_2\in\ZZ_{\ge0}$ with $(-a_1+ba_2,-a_2)\in\cI$, we have $\mu_{1,c}\big(D^{dom}[a_1,a_2]\big)=\vv{D}^{dom}[ca_1-a_2,a_1]$ and $\mu_{2,b}\big(D^{dom}[a_1,a_2]\big)=\vv{D}^{dom}[a_2,ba_2-a_1]$.
  \end{theorem}
  \begin{proof}
    We prove the result for $\mu_{1,c}$ as the result for $\mu_{2,b}$ immediately follows from $\mu_{1,b}$ being its inverse.
    Write $D'$ for the Dyck path $\mu_{1,c}\big(D^{dom}[a_1,a_2]\big)$ and observe that by construction $D'$ has $ca_1-a_2$ horizontal edges and $a_1$ vertical edges.
    Write $v'_1,\ldots,v'_{a_1}$ for the vertical edges of $D'$ with $v'_k$ immediately preceded by $r_k$ horizontal edges.
    We have to show (i) that every lattice point on $D'$ stays below the lines 
    \[y=\frac{bc+\sqrt{bc(bc-4)}}{2c}x \qquad \text{and} \qquad y-a_1=\frac{bc-\sqrt{bc(bc-4)}}{2c}\big(x-(ca_1-a_2)\big)\]
    while (ii) any lattice point above $D'$ is also above one of these lines.

    If there exists $t$ so that $v'_t$ passes above the line $y=\frac{bc+\sqrt{bc(bc-4)}}{2c}x$, then
    \[ \frac{t}{\sum\limits_{k=1}^t r_k} > \frac{bc+\sqrt{bc(bc-4)}}{2c} \]
    or, rearranging, we have
    \[ \frac{\sum\limits_{k=1}^t r_k}{t} < \frac{2c}{bc+\sqrt{bc(bc-4)}} = \frac{bc-\sqrt{bc(bc-4)}}{2b} \]
    or
    \[ \frac{bc+\sqrt{bc(bc-4)}}{2b} < \frac{ct-\sum\limits_{k=1}^t r_k}{t} = \frac{\sum\limits_{k=1}^t (c-r_k)}{t}. \]
    But this states that the subpath of $D^{dom}[a_1,a_2]$ consisting of the first $t$ horizontal edges and all their trailing vertical edges passes above the line $y=\frac{bc+\sqrt{bc(bc-4)}}{2b}x$, contrary to the definition.

    If there exists $t$ so that $v'_t$ passes above the line $y-a_1=\frac{bc-\sqrt{bc(bc-4)}}{2c}\big(x-(ca_1-a_2)\big)$, then
    \[ \frac{bc-\sqrt{bc(bc-4)}}{2c} > \frac{a_1-t}{ca_1-a_2-\sum\limits_{k=1}^t r_k} \]
    or, rearranging, we have
    \[ \frac{ca_1-a_2-\sum\limits_{k=1}^t r_k}{a_1-t} > \frac{2c}{bc-\sqrt{bc(bc-4)}} = \frac{bc+\sqrt{bc(bc-4)}}{2b} \]
    or
    \[ \frac{bc-\sqrt{bc(bc-4)}}{2b} > \frac{ c(a_1-t) - \big(ca_1-a_2-\sum\limits_{k=1}^t r_k\big)}{a_1-t} = \frac{a_2-\sum\limits_{k=1}^t (c-r_k)}{a_1-t}. \]
    But this states that the subpath of $D^{dom}[a_1,a_2]$ consisting of the first $t$ horizontal edges and all their trailing vertical edges passes above the line $y-a_2=\frac{bc-\sqrt{bc(bc-4)}}{2b}(x-a_1)$, contrary to the definition.

    This proves (i), analogous calculations establish (ii). 
  \end{proof}

  Apart from being the prefix and suffix for the dominant Dyck path, the Dyck paths $\overrightarrow{D}^s$ and $\overleftarrow{D}^t$ can be observed inside the maximal Dyck paths defining cluster variables as follows.
  \begin{lemma}\mbox{}
    \begin{itemize}
      \item For $s>0$ and any $m \ge 2$ so that $s < u_{m,-}$, the Dyck path prefix $\overrightarrow{D}^s$ is an initial segment of the maximal Dyck path $D^{max}[u_{m-1,+},u_{m,-}]$.
      \item For $t>0$ and any $m \ge 2$ so that $t < u_{m,+}$, the Dyck path suffix $\overleftarrow{D}^t$ is a final segment of the maximal Dyck path $D^{max}[u_{m,+},u_{m-1,-}]$.
    \end{itemize}
  \end{lemma}

  \begin{lemma}\mbox{}
    \begin{enumerate}
      \item Choose a width $t>0$ and $a_1 \ge t$.
        Then there exists a unique $s$ of depth $a_1-t$ so that $\overrightarrow{D}^s \sqcup \overleftarrow{D}^t$ is the dominant Dyck path $D^{dom}[a_1,a_2]$ with $a_2=s+\left\lceil t \frac{bc-\sqrt{bc(bc-4)}}{2b}\right\rceil$.
      \item Choose a height $s>0$ and $a_2 \ge s$.
        Then there exists a unique $t$ of height $a_2-s$ so that $\overrightarrow{D}^s \sqcup \overleftarrow{D}^t$ is the dominant Dyck path $D^{dom}[a_1,a_2]$ with $a_1=\left\lceil s \frac{bc-\sqrt{bc(bc-4)}}{2c}\right\rceil+t$.
    \end{enumerate}
  \end{lemma}
  \begin{remark}
    The decomposition of $D^{dom}[a_1,a_2]$ as $\overrightarrow{D}^s \sqcup \overleftarrow{D}^t$ is not uniquely determined by $a_1$ and $a_2$.
    However, there exists at most two choices of $s$ and $t$ for which such a decomposition holds.
  \end{remark}

  \begin{lemma}
    \label{le:floors and ceilings}
    For any real number $x$ and any real number $r>1$, we have
    \[\lfloor x\rfloor \le \left\lfloor \lceil xr \rceil \frac{1}{r} \right\rfloor \le \lceil x\rceil \qquad \text{and} \qquad \lfloor x\rfloor \le \left\lceil \lfloor xr \rfloor \frac{1}{r} \right\rceil \le \lceil x\rceil.\]
  \end{lemma}
  \begin{proof}
    Observe that
    \[xr\le \lceil xr \rceil < xr+1 \qquad \text{and} \qquad xr-1 < \lfloor xr \rfloor \le xr\]
    and so
    \[x\le \lceil xr \rceil \frac{1}{r} < x+\frac{1}{r} \qquad \text{and} \qquad x-\frac{1}{r} < \lfloor xr \rfloor \frac{1}{r} \le x.\]
    Since $0<\frac{1}{r}<1$, the result follows.
  \end{proof}

  The following result describes the possible decomposition of the dominant Dyck path into its prefix and suffix subpaths.
  The intersection point from Remark~\ref{rem:intersection} is a useful point of reference.
  \begin{proposition}
    \label{prop:dominant Dyck path decomposition}
    For $a_1,a_2\in\ZZ_{\ge0}$ with $(-a_1+ba_2,-a_2)\in\cI$, the dominant Dyck path $D^{dom}[a_1,a_2]$ is the concatenation of $\overrightarrow{D}^s$ followed by $\overleftarrow{D}^t$ for any choice of $s$ and $t$ satisfying $a_2-s=\left\lceil t \frac{bc-\sqrt{bc(bc-4)}}{2b}\right\rceil$ and $a_1-t=\left\lceil s \frac{bc-\sqrt{bc(bc-4)}}{2c}\right\rceil$.
    In this case, the pair of $s$ and $t$ will be one of the following:
    \begin{align}
      \label{eq:lower bounds}
      s=\left\lfloor\frac{\big(bc+\sqrt{bc(bc-4)}\big)a_2 - 2 c a_1}{2\sqrt{bc(bc-4)}}\right\rfloor\qquad &\text{and}\qquad t=\left\lfloor\frac{\big(bc+\sqrt{bc(bc-4)}\big)a_1 - 2 b a_2}{2\sqrt{bc(bc-4)}}\right\rfloor\\
      s=\left\lfloor\frac{\big(bc+\sqrt{bc(bc-4)}\big)a_2 - 2 c a_1}{2\sqrt{bc(bc-4)}}\right\rfloor\qquad &\text{and}\qquad t=\left\lceil\frac{\big(bc+\sqrt{bc(bc-4)}\big)a_1 - 2 b a_2}{2\sqrt{bc(bc-4)}}\right\rceil\\
      \label{eq:upper bounds}
      s=\left\lceil\frac{\big(bc+\sqrt{bc(bc-4)}\big)a_2 - 2 c a_1}{2\sqrt{bc(bc-4)}}\right\rceil\qquad &\text{and}\qquad t=\left\lceil\frac{\big(bc+\sqrt{bc(bc-4)}\big)a_1 - 2 b a_2}{2\sqrt{bc(bc-4)}}\right\rceil
    \end{align}
    If two such decompositions exist, they will be of the form \eqref{eq:lower bounds} and \eqref{eq:upper bounds}.
  \end{proposition}
  \begin{remark}
    The irrationality of the slopes defining the Dyck path prefixes and suffixes makes it difficult to predict for a given choice of $a_1$ and $a_2$ precisely which of these choice of $s$ and $t$ will produce the desired equalities.
  \end{remark}
  \begin{proof}
    (In progress)

    For $s=\left\lfloor\frac{\big(bc+\sqrt{bc(bc-4)}\big)a_2 - 2 c a_1}{2\sqrt{bc(bc-4)}}\right\rfloor$, by Lemma~\ref{le:floors and ceilings} we have
    \[\left\lfloor \frac{2 b a_2 -\big(bc-\sqrt{bc(bc-4)}\big) a_1}{2\sqrt{bc(bc-4)}} \right\rfloor \le \left\lceil s \frac{bc-\sqrt{bc(bc-4)}}{2c} \right\rceil \le \left\lceil \frac{2 b a_2 -\big(bc-\sqrt{bc(bc-4)}\big) a_1}{2\sqrt{bc(bc-4)}} \right\rceil\]
    since $0<\frac{bc-\sqrt{bc(bc-4)}}{2c}<1$.
    It follows that
    \[ a_1 - \left\lceil \frac{2 b a_2 -\big(bc-\sqrt{bc(bc-4)}\big) a_1}{2\sqrt{bc(bc-4)}} \right\rceil \le a_1 - \left\lceil s \frac{bc-\sqrt{bc(bc-4)}}{2c} \right\rceil \le a_1 - \left\lfloor \frac{2 b a_2 -\big(bc-\sqrt{bc(bc-4)}\big) a_1}{2\sqrt{bc(bc-4)}} \right\rfloor\]
    and so
    \[ \left\lfloor \frac{ \big(bc+\sqrt{bc(bc-4)}\big) a_1 - 2 b a_2 }{2\sqrt{bc(bc-4)}} \right\rfloor \le a_1 - \left\lceil s \frac{bc-\sqrt{bc(bc-4)}}{2c} \right\rceil \le \left\lceil \frac{ \big(bc+\sqrt{bc(bc-4)}\big) a_1 - 2 b a_2 }{2\sqrt{bc(bc-4)}} \right\rceil\]
    with one of these inequalities being an equality.

    If the left hand inequality is an equality, we take $t=\left\lfloor\frac{\big(bc+\sqrt{bc(bc-4)}\big)a_1 - 2 b a_2}{2\sqrt{bc(bc-4)}}\right\rfloor$.

    The desired equalities can be rewritten in the form 
    \[s=a_2-\left\lceil t \frac{bc-\sqrt{bc(bc-4)}}{2b}\right\rceil=\left\lfloor a_2 - t \frac{bc-\sqrt{bc(bc-4)}}{2b}\right\rfloor\]
    and
    \[t=a_1-\left\lceil s \frac{bc-\sqrt{bc(bc-4)}}{2c}\right\rceil=\left\lfloor a_1 - s \frac{bc-\sqrt{bc(bc-4)}}{2c}\right\rfloor.\]
    Combining these, we seek $s$ and $t$ satisfying
    \[s=\left\lfloor a_2 - \left\lfloor a_1 - s \frac{bc-\sqrt{bc(bc-4)}}{2c}\right\rfloor \frac{bc-\sqrt{bc(bc-4)}}{2b}\right\rfloor\]
    and
    \[t=\left\lfloor a_1 - \left\lfloor a_2 - t \frac{bc-\sqrt{bc(bc-4)}}{2b}\right\rfloor \frac{bc-\sqrt{bc(bc-4)}}{2c}\right\rfloor.\]
  \end{proof}

  \subsection{Compatibility}
  The results of this section can mostly be found in \cite{lee-li-zelevinsky,rupel}.
  However since we work in arbitrary Dyck paths instead of just the maximal ones, some results and proofs will need to be adapted.

  For $e \le e'\in D$, write $ee'$ for the subpath of $D$ beginning with $e$ and ending with $e'$.
  When $e=e'$, we take $ee'$ to be the subpath consisting of only the edge $e$.
  \begin{definition}
    For $H\sqcup V\subset D$ with $H\subset D_1$ and $V\subset D_2$ together with a subpath $ee'\subset D$, define the \emph{horizontal shadow statistic}:
    \begin{equation}
      \label{eq:horizontal shadow statistic}
      f_H(ee')=-|ee' \cap D_2|+c|ee'\cap H|
    \end{equation}
    and the \emph{vertical shadow statistic}:
    \begin{equation}
      \label{eq:vertical shadow statistic}
      f_V(ee')=-|ee' \cap D_1|+b|ee'\cap V|.
    \end{equation}
  \end{definition}
  \begin{remark}
    Observe that the shadow statistics are additive with respect to concatenation of subpaths.
  \end{remark}

  A key notion from \cite{lee-li-zelevinsky} is that of ``compatible subsets of a maximal Dyck path'', this construction immediately generalizes to arbitrary Dyck paths as follows.
  \begin{definition}
    Consider $H\sqcup V\subset D$ with $H\subset D_1$ and $V\subset D_2$.
    For $h\in H$ and $v\in V$ with $h<v$, call the path $hv$ \emph{compatible} if there exists an edge $e\in hv$ such that one of the following conditions is satisfied:
    \begin{equation}
      \label{eq:HGC}
      \tag{HGC} e\ne v \quad \text{and} \quad f_H(he)=0;
    \end{equation}
    \begin{equation}
      \label{eq:VGC}
      \tag{VGC} e\ne h \quad \text{and} \quad f_V(ev)=0.
    \end{equation}
    Call the subset $H\sqcup V$ \emph{compatible} if every path $hv$ with $h\in H$ and $v\in V$ is compatible.
    Write $\cC_D$ for the collection of all compatible subsets of $D$.
    Given a subset $H\subset D_1$, write $\cC_D(H)$ for the collection of all $V\subset D_2$ for which $H\sqcup V\in\cC_D$.
    Define $\cC_D(V)$ for $V\subset D_2$ similarly.
  \end{definition}
  \begin{remark}
    In the construction of the greedy basis, it is necessary to consider $D$ as a closed cycle by identifying $(0,0)$ with $(a_1,a_2)$ and allowing subpaths $ee'$ with $e'<e$ to ``wrap around'' $D$ in the obvious way.
    When we wish to emphasize that this condition should be imposed, we denote the Dyck path $D$ instead by $D_\WA$.
    Following \cite[Remark 2.21]{rupel}, for $D^{m,\varepsilon}$ and $\vv{\bar{D}}^{m,\varepsilon}$ the wrap around condition does not restrict compatibility and can thus safely be omitted.
  \end{remark}

  The crucial result from \cite{lee-li-zelevinsky} can be stated as follows: if a Dyck path $D$ is vertically-adapted to $c$ (resp. horizontally adapted to $b$), then there is a bijection between certain compatible subsets of $D$ and certain compatible subsets of $\mu_{1,c}(D)$ (resp. of $\mu_{2,b}(D)$).
  The main step in the proof that needs to be adapted to the general situation is the following.
  For $e \le h\in D$ with $h=h_d\in D_1$, write $eh^{\!\uparrow}$ for the subpath $eh$ together with all vertical edges of depth $d$ immediately following $h$.
  For $v \le e\in D$ with $v=v_\ell\in D_2$, write ${}^\leftarrow\!\!\! ve$ for the subpath $ve$ together with all horizontal edges of height $\ell$ immediately preceding $v$.
  \begin{lemma}\mbox{}
    \label{le:shadow statistic recursion}
    \begin{enumerate}
      \item Suppose the Dyck path $D$ is vertically-adapted to $c$ and let $D'=\mu_{1,c}(D)$ with $D'_2=\{v'_1,\ldots,v'_{a_1}\}$.
        For $H\subset D_1$, define $V'\subset D'_2$ by $V'=D'_2\setminus \mu_{1,c}(H)$.
        Then for any $h_i \le h_j \in D_1$, we have $f_H(h_i h_j^{\!\uparrow})=-\vv{f}_{V'}({}^\leftarrow\!\!\! v'_i v'_j)$.
      \item Suppose the Dyck path $D$ is horizontally-adapted to $b$ and let $D''=\mu_{2,b}(D)$ with $D''_1=\{h''_1,\ldots,h''_{a_2}\}$.
        For $V\subset D_2$, define $H''\subset D''_1$ by $H''=D''_1\setminus \mu_{2,b}(V)$.
        Then for any $v_i \le v_j \in D_2$, we have $f_V({}^\leftarrow\!\!\! v_i v_j)=-\vv{f}_{H''}(h''_i {h''_j}^{\!\uparrow})$.
    \end{enumerate}
  \end{lemma}
  \begin{proof}
    Claim (2) is immediate from claim (1). 
    By definition, we have 
    \begin{align*}
      -\vv{f}_{V'}({}^\leftarrow\!\!\! v'_iv'_j)&=|{}^\leftarrow\!\!\! v'_i v'_j \cap D'_1|-c|{}^\leftarrow\!\!\! v'_i v'_j \cap V'|\\
      &=|{}^\leftarrow\!\!\! v'_i v'_j \cap D'_1|-c\big|{}^\leftarrow\!\!\! v'_i v'_j \cap \big(D'_2\setminus \mu_{1,c}(H)\big)\big|\\
      &=|{}^\leftarrow\!\!\! v'_i v'_j \cap D'_1|-c|{}^\leftarrow\!\!\! v'_i v'_j \cap D'_2|+c\big|{}^\leftarrow\!\!\! v'_i v'_j \cap \mu_{1,c}(H)\big|.
    \end{align*}
    But, by construction of $\mu_{1,c}(D)$ and since $\mu_{2,c}(D')=D$, we have
    \[\big|{}^\leftarrow\!\!\! v'_i v'_j \cap \mu_{1,c}(H)\big|=|h_i h_j^{\!\uparrow} \cap H| \qquad \text{and} \qquad c|{}^\leftarrow\!\!\! v'_i v'_j \cap D'_2|-|{}^\leftarrow\!\!\! v'_i v'_j \cap D'_1|=|h_i h_j^{\!\uparrow} \cap D_2|\]
    giving
    \[-\vv{f}_{V'}({}^\leftarrow\!\!\! v'_iv'_j)=-|h_i h_j^{\!\uparrow} \cap D_2|+c|h_i h_j^{\!\uparrow} \cap H|=f_H(h_i h_j^{\!\uparrow})\]
    as desired.
  \end{proof}

  The desired bijection of compatible subsets is constructed in terms of certain subpaths determined from each compatible subset.
  \begin{definition}
    \label{def:shadows}
    For $H\sqcup V\subset D$ with $H\subset D_1$ and $V\subset D_2$, define the \emph{shadow paths} as follows:
    \begin{itemize}
      \item For $h\in D_1$, write $D(h;H)$ for the shortest subpath $he \subset D$ with $f_H(he)=0$ or, if no such subpath exists, set $D(h;H)=hv_{a_2}$.
      \item For $v\in D_2$, write $D(v;V)$ for the shortest subpath $ev \subset D$ with $f_V(ev)=0$ or, if no such subpath exists, set $D(v;V)=h_1 v$.
    \end{itemize}
    Write 
    \[D(H):=\bigcup_{h\in H} D(h;H) \qquad \text{and} \qquad D(V):=\bigcup_{v\in V} D(v;V).\]
    Define the \emph{shadow} of $H$ (resp. the \emph{shadow} of $V$) as the set $\sh_D(H):=D(H)\cap D_2$ (resp.~$\sh_D(V):=D(V)\cap D_1$) and the \emph{local shadow} of $h\in H$ (resp. of $v\in V$) as the set $\sh_D(h;H)=D(h;H)\cap D_2$ (resp. $\sh_D(v;V)=D(v;V)\cap D_1$).
  \end{definition}

  The proofs of the following results are identical to the proof of the analogous results from \cite{lee-li-zelevinsky,rupel} for maximal Dyck paths.
  \begin{lemma}
    \label{le:shadow statistics}
    Consider $H\sqcup V\subset D$ with $H\subset D_1$ and $V\subset D_2$.
    \begin{enumerate}
      \item For $h\in H$, take $he$ to be the path $D(h;H)$.
        The following hold:
        \begin{enumerate}
          \item for any proper subpath $he'\subsetneq he$, we have $f_H(he')>0$;
          \item if $f_H(he)=0$, then for any proper subpath $e'e\subsetneq he$, we have $f_H(e'e)<0$;
          \item $e\in V$.
        \end{enumerate}
      \item For $v\in V$, take $ev$ to be the path $D(v;V)$.
        The following hold:
        \begin{enumerate}
          \item for any proper subpath $e'v\subsetneq ev$, we have $f_V(e'v)>0$;
          \item if $f_V(ev)=0$, then for any proper subpath $ee'\subsetneq ev$, we have $f_V(ee')<0$;
          \item $e\in H$.
        \end{enumerate}
    \end{enumerate}
  \end{lemma}
  \begin{remark}
    Observe that the subset $V$ has no influence on the claim of (1) and the subset $H$ has no influence on the claim of (2).
  \end{remark}

  \begin{definition}
    Given two Dyck paths $D,D'$ in the lattice rectangle $[0,a_1]\times[0,a_2]$, write $D'\succ D$ if no lattice point of $D'$ lies below a lattice point of $D$.
  \end{definition}
  \begin{remark}
    Equivalently, $D'\succ D$ if no lattice point of $D'$ lies to the right of a lattice point of $D$.
  \end{remark}

  \begin{proposition}
    Let $D,D'$ be two Dyck paths in the lattice rectangle $[0,a_1]\times[0,a_2]$ with $D'\succ D$.
    Identify $D_1$ with $D'_1$ and $D_2$ with $D'_2$ along the natural left-to-right ordering.
    For subsets $H\subset D_1$ and $V\subset D_2$, write $H'\subset D'_1$ and $V'\subset D'_2$ for the corresponding subsets under the identification above.
    If $(H,V) \in \cC_D$, then $(H',V') \in \cC_{D'}$.
  \end{proposition}
  \begin{proof}
    It suffices, by induction, to show that the result holds if $D'$ is an immediate successor of $D$, i.e. when a single pair of a horizontal followed by a vertical is replaced by a vertical followed by a horizontal.
    Assume $D'$ is obtained from $D$ in this way, say with horizontal edge $h_i$ and vertical edge $v_j$ satisfying $v_j=h_i+1$ in $D$ replaced by the vertical edge $v'_j$ and the horizontal edge $h'_i$ satisfying $h'_i=v'_j+1$ in $D'$.

    Consider $(H,V) \in \cC_D$.
    For $h\in H$ and $v\in V$, write $h'\in H'$ and $v'\in V'$ for the corresponding edges.
    We claim that the condition \eqref{eq:HGC} (resp. \eqref{eq:VGC}) being satisfied for the path $hv$ in $D$ implies the same condition is satisfied for the path $h'v'$ in $D'$.

    Suppose \eqref{eq:HGC} is satisfied for the path $hv$ and write $D(h;H)=he$ for $e\in V$ with $e<v$.
    If $h_i,v_j\notin D(h;H)$, then under the identification with $D'$ we have $D(h';H')=D(h;H)$ and \eqref{eq:HGC} is satisfied for the path $h'v'$.
    Alternatively, since $e\in V$, we must have both $h_i,v_j\in D(h;H)$.
    If $h_i\notin H$, then again we have $D(h';H')=D(h;H)$ and \eqref{eq:HGC} is satisfied for the path $h'v'$.
    Thus, we assume $h_i\in H$.

    We begin by assuming $h_i\ne h$.
    This gives two possibilities for $D(h';H')$ depending on $D(h'_i;H')$.
    First, observe that $D(h_i;H)$ is a proper subpath of $h_ie$, for otherwise $f_H(h_ie)=0$ and, by the additivity of $f_H$, $he$ would not be the shortest subpath with $f_H(he)=0$.
    Thus, if $D(h_i;H)$ is followed by at least two vertical edges inside $D(h;H)$, we again have $D(h';H')=D(h;H)$ and \eqref{eq:HGC} is satisfied for the path $h'v'$.
    Otherwise, $D(h';H')$ is an initial subpath of $D(h;H)$ under the identification of $D'$ with $D$ and again \eqref{eq:HGC} is satisfied for the path $h'v'$.

    Now if $h_i=h$...
  \end{proof}

  \begin{corollary}
    Consider $H\sqcup V\subset D$ with $H\subset D_1$ and $V\subset D_2$.
    \begin{enumerate}
      \item For $h\in H$ and a vertical edge $v\in D(h;H)$, the condition \eqref{eq:HGC} is not satisfied for the path $hv$ but for any $V\in\cC_D(H)$ the condition \eqref{eq:VGC} is satisfied for the path $hv$.
        In particular, $D(v;V)$ is a proper subpath of $hv$ for any $V\in\cC_D(H)$.
      \item For $v\in V$ and a horizontal edge $h\in D(v;V)$, the condition \eqref{eq:VGC} is not satisfied for the pair $hv$ but for any $H\in\cC_D(V)$ the condition \eqref{eq:HGC} is satisfied for the path $hv$.
        In particular, $D(h;H)$ is a proper subpath of $hv$ for any $H\in\cC_D(V)$.
    \end{enumerate}
  \end{corollary}

  \begin{lemma}\mbox{}
    \begin{enumerate}
      \item For $H\subset D_1$ and $h,h'\in H$, the local shadows $\sh_D(h;H)$ and $\sh_D(h';H)$ are either disjoint or one is contained in the other.
      \item For $V\subset D_2$ and $v,v'\in V$, the local shadows $\sh_D(v;V)$ and $\sh_D(v';V)$ are either disjoint or one is contained in the other.
    \end{enumerate}
  \end{lemma}

  \begin{definition}
    Consider $H\sqcup V\subset D$ with $H\subset D_1$ and $V\subset D_2$.
    \begin{enumerate}
      \item Define $\rsh_D(H)\subset\sh_D(H)$, the \emph{remote shadow} of $H$, by removing for each $d\in[1,a_1]$ the first (up to) $c$ vertical edges of depth $d$ immediately following $h_d$.
      \item Define $\rsh_D(H)\subset\sh_D(H)$, the \emph{remote shadow} of $V$, by removing for each $\ell\in[0,a_2-1]$ the first (up to) $b$ horizontal edges of height $\ell$ immediately preceding $v_{\ell+1}$.
    \end{enumerate}
  \end{definition}

  \begin{lemma}\mbox{}
    \begin{enumerate}
      \item For $H\subset D_1$, $V\in\cC_D(H)$ if and only if $V$ is disjoint from $\sh_D(H)\setminus\rsh_D(H)$ and one of the conditions \eqref{eq:HGC} or \eqref{eq:VGC} is satisfied for each path $hv$ with $h\in H$ and $v\in\rsh_D(H)$.
      \item For $V\subset D_2$, $H\in\cC_D(V)$ if and only if $H$ is disjoint from $\sh_D(V)\setminus\rsh_D(V)$ and one of the conditions \eqref{eq:HGC} or \eqref{eq:VGC} is satisfied for each path $hv$ with $h\in\rsh_D(V)$ and $v\in V$.
    \end{enumerate}
  \end{lemma}

  \begin{definition}
    For $H\subset D_1$, write $\cC_D^{rsh}(H)$ for the subset of $V\in\cC_D(H)$ with $V\subset\rsh_D(H)$.
    For $V\subset D_2$, write $\cC_D^{rsh}(V)$ for the subset of $H\in\cC_D(V)$ with $H\subset\rsh_D(V)$.
  \end{definition}

  \begin{definition}
    Consider $H\sqcup V\subset D$ with $H\subset D_1$ and $V\subset D_2$.
    \begin{enumerate}
      \item For $1 \le j < d \le a_1$ with $h_j\in H$, denote by $\rsh_D(H)_{j;d}$ the set of all $v\in\rsh_D(H)$ of depth $d$ such that $v\in\sh(h_j;H)$ and $h_j$ is the first horizontal edge before $v$ with this property.
        Define the \emph{local remote shadow} of the edge $h_j$ by $\rsh_D(h_j;H):=\bigsqcup\limits_{d\in[j+1,a_1]} \rsh_D(H)_{j;d}$.
      \item For $1 \le \ell <  k \le a_2$ with $v_k\in V$, denote by $\rsh_D(V)_{k;\ell}$ the set of all $h\in\rsh_D(V)$ of height $\ell$ such that $h\in\sh(v_k;V)$ and $v_k$ is the first vertical edge after $h$ with this property.
        Define the \emph{local remote shadow} of the edge $v_k$ by $\rsh_D(v_k;V):=\bigsqcup\limits_{\ell\in[1,k-1]} \rsh_D(V)_{k;\ell}$.
    \end{enumerate}
  \end{definition}

  As shown in \cite{lee-li-zelevinsky} for maximal Dyck paths, the remote shadow can be characterized as follows.
  \begin{lemma}
    Consider $H\sqcup V\subset D$ with $H\subset D_1$ and $V\subset D_2$.
    \begin{enumerate}
      \item For $1 \le d < j \le a_1$, we have $\rsh_D(H)_{j;d} \ne \varnothing$ if and only if $f_H(h'h_d^{\!\uparrow}) < 0 < f_H(h_jh^{\!\uparrow})$ for every pair of consecutive horizontal edges $h<h'$ in the path $h_j h_d^{\!\uparrow}$.
      \item For $1 \le \ell < k \le a_2$, we have $\rsh_D(V)_{k;\ell} \ne \varnothing$ if and only if $f_V({}^\leftarrow\!\!\! v_\ell v) < 0 < f_V({}^\leftarrow\!\!\! v' v_k)$ for every pair of consecutive vertical edges $v<v'$ in the path ${}^\leftarrow\!\!\! v_\ell v_k$.
    \end{enumerate}
  \end{lemma}

  \begin{lemma}
    Consider $H\sqcup V\subset D$ with $H\subset D_1$ and $V\subset D_2$.
    \begin{enumerate}
      \item Let $1 \le d < j \le a_1$ and suppose $\rsh_D(H)_{j;d} \ne \varnothing$, then
        \[|\rsh_D(H)_{j;d}| = \min\limits_{h_j \le h < h' \le h_d} \min\big( -f_H(h'h_d^{\!\uparrow}) , f_H(h_jh^{\!\uparrow}) \big).\]
      \item Let $1 \le \ell < k \le a_2$ and suppose $\rsh_D(V)_{k;\ell} \ne \varnothing$, then
        \[|\rsh_D(V)_{k;\ell}| = \min\limits_{v_\ell \le v < v' \le v_k} \min\big( -f_V({}^\leftarrow\!\!\! v_\ell v) , f_V({}^\leftarrow\!\!\! v' v_k) \big).\]
    \end{enumerate}
  \end{lemma}
  
  Then the following result immediately follows from Lemma~\ref{le:shadow statistic recursion}.
  \begin{corollary}\mbox{}
    \label{cor:remote shadows}
    \begin{enumerate}
      \item Suppose the Dyck path $D$ is vertically-adapted to $c$ and let $D'=\mu_{1,c}(D)$ with $D'_2=\{v'_1,\ldots,v'_{a_1}\}$.
        For $H\subset D_1$, define $V'\subset D'_2$ by $V'=D'_2\setminus \mu_{1,c}(H)$.
        Then, for $1 \le d < j \le a_1$, we have $|\rsh_D(H)_{j;d}|=|\rsh_{D'}(V')_{d;j}|$.
      \item Suppose the Dyck path $D$ is horizontally-adapted to $b$ and let $D''=\mu_{2,b}(D)$ with $D''_1=\{h''_1,\ldots,h''_{a_2}\}$.
        For $V\subset D_2$, define $H''\subset D''_1$ by $H''=D''_1\setminus \mu_{2,b}(V)$.
        Then, for $1 \le \ell < k \le a_2$, we have $|\rsh_D(V)_{k;\ell}|=|\rsh_{D''}(H'')_{\ell;k}$.
    \end{enumerate}
  \end{corollary}

  In the results below, we keep the notation and assumptions from Corollary~\ref{cor:remote shadows}.
  It follows that, for $H\subset D_1$ and $1 \le d < j \le a_1$, we may define an order-reversing bijection $\theta^H_{j;d} : \rsh_D(H)_{j;d} \to \rsh_{D'}(V')_{d;j}$.
  Similarly, for $V\subset D_2$ and $1 \le \ell < k \le a_2$, we may define an order-reversing bijection $\theta^V_{k;\ell} : \rsh_D(V)_{k;\ell} \to \rsh_{D''}(H'')_{\ell;k}$.
  \begin{definition}\mbox{}
    \begin{enumerate}
      \item For $H\subset D_1$ and $V \subset \rsh_D(H)$, define $\Omega_H(V) \subset \rsh_{D'}(V')$ by
        \[\Omega_H(V)=\{\theta^H_{j;d}(v) : v \in \rsh_D(H)_{j;d}, 1 \le d < j \le a_1\}.\]
      \item For $V\subset D_2$ and $H \subset \rsh_D(V)$, define $\Omega_V(H) \subset \rsh_{D''}(H'')$ by
        \[\Omega_V(H)=\{\theta^V_{k;\ell}(h) : h \in \rsh_D(V)_{k;\ell}, 1 \le \ell < k \le a_2\}.\]
    \end{enumerate}
  \end{definition}

  \begin{lemma}\mbox{}
    \begin{enumerate}
      \item Consider $H\subset D_1$ and $V \subset \rsh_D(H)$.
        Suppose $h'=\theta^H_{j;d}(v)$ for $v\in V$.
        Then $f_{\Omega_H(V)}(h'v'_d)=f_V(h_j v)$.
      \item Consider $V\subset D_2$ and $H \subset \rsh_D(V)$.
        Suppose $v''=\theta^V_{k;\ell}(h)$ for $h\in H$.
        Then $f_{\Omega_V(H)}(h''_\ell v'')=f_V(h v_k)$.
    \end{enumerate}
  \end{lemma}

  The following equivalence of compatibilities is the main result we need.
  \begin{corollary}\mbox{}
    \begin{enumerate}
      \item For $H \subset D_1$ and $V \subset \rsh_D(H)$, we have $V\in\cC_D(H)$ if and only if $\Omega_H(V)\in\cC_{D'}(V')$.
      \item For $V \subset D_2$ and $H \subset \rsh_D(V)$, we have $H\in\cC_D(V)$ if and only if $\Omega_V(H)\in\cC_{D''}(H'')$.
    \end{enumerate}
  \end{corollary}

  We associate to $D$ the element $x_D\in\ZZ_{\ge0}[x_1^{\pm1},x_2^{\pm1}]$ defined as follows
  \[x_D:=x_1^{-a_1} x_2^{-a_2} \sum_{H\sqcup V\in\cC_D} x_1^{b|V|} x_2^{c|H|}.\]

  \begin{theorem}[\cite{lee-li-zelevinsky}]
    For $(a_1,a_2)=(u_{m,+},u_{m-1,-})$ or $(u_{m-1,+},u_{m,-})$ for $m\ge1$, the element $x_{D^{max}[a_1,a_2]}$ is a non-initial cluster variable.
    Otherwise, the element $x_{D^{max}_\WA[a_1,a_2]}$ is an element of the greedy basis
  \end{theorem}

  \begin{lemma}
    Suppose there are two factorizations
    \[x_D=x_1^{-a_1} x_2^{-a_2} \sum_{t=0}^{a_2}\sum_{k=0}^{a_1} \nu_{k,t} x_1^{bt}(1+x_2^c)^{a_1-k}=x_1^{-a_1} x_2^{-a_2} \sum_{s=0}^{a_1}\sum_{\ell=0}^{a_2} \theta_{s,\ell} x_2^{cs}(1+x_1^b)^{a_2-\ell},\]
    where $\nu_{k,t}$ is the number of subsets $V\subseteq_t D_2$ with $|\sh_D(V)|=k$ and $\theta_{s,\ell}$ is the number of subsets $H\subseteq_s D_1$ with $|\sh_D(H)|=\ell$.
    Then 
    \[\sum_{k=0}^{a_1-s} {a_1-k\choose s} \nu_{k,t}=\sum_{\ell=0}^{a_2-t} {a_2-\ell\choose t} \theta_{s,\ell}\]
    for all possible $s$ and $t$.
  \end{lemma}
  \begin{question}
    How do such expressions behave under mutations?
  \end{question}

  \begin{lemma}
    Choose $a_1,a_2\in\ZZ_{\ge0}$.
    For $D^{st}:=D_1\sqcup D_2$ with $D_2=[1,a_2]$ and $D_1=[1,a_1]+a_2$, the element $x_{D^{st}}$ coincides with the corresponding standard monomial basis element.
  \end{lemma}

  \begin{definition}
    Suppose $D^{dom}[a_1,a_2]$ decomposes as $\overrightarrow{D}^s \sqcup \overleftarrow{D}^t$.
    Call a pair of subsets $H \sqcup V$ \emph{piecewise compatible} if the intersections with $\overrightarrow{D}^s$ and $\overleftarrow{D}^t$ are compatible on the subpaths.
  \end{definition}

  \begin{theorem}
    Suppose $H \sqcup V$ is a piecewise compatible subset of $D^{dom}[a_1,a_2]$.
    \begin{itemize}
      \item If the last vertical edge of $\overrightarrow{D}^s$ is contained in $V$, then $H \sqcup V$ is compatible.
      \item If the first horizontal edge of $\overleftarrow{D}^t$ is contained in $H$, then $H \sqcup V$ is compatible.
    \end{itemize}
  \end{theorem}

\end{document}
